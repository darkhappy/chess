\documentclass[12pt]{article}
\usepackage[margin=1in]{geometry}
\usepackage{float}
\usepackage[table,xcdraw]{xcolor}
\usepackage{graphicx}

\graphicspath{ {./images/} }

\title{Class Dictionary}
\author{Jean-Philippe Miguel-Gagnon, Jérémy Gaouette, Raphaël Rail}

\date{Thursday, 31th of march 2022}

\begin{document}

\begin{titlepage}
\maketitle
\includegraphics[width=\textwidth]{CHESS}
\begin{center}Presented to : Charles Jacob\end{center}
\end{titlepage}

\tableofcontents

\newpage

\section{Introduction}

The goal of this document is to inform the programmer about
classes used to create a C\#\ OOP Chess game.
\\

We'll go through this with the MVC model approch to make it
clearer for the programmer where to implement his code. 

\newpage

\section{Models}
%%%%%%%%%%%%%
%Piece Class%

\subsection{Classe Piece}

This is an abstract class for a base piece for the game.
\begin{table}[H]
    \begin{tabular}{|l|}
    \hline
    \cellcolor[HTML]{C0C0C0}\textbf{(Abstract) Piece}  \\ \hline
    \cellcolor[HTML]{EFEFEF}-\_colour : Colour         \\ \hline
    +CanCollide() : bool                               \\ \hline
    +ValidMove(int x1, int y1, int x2, int y2) : bool  \\ \hline
    +ToString() : string                               \\ \hline
    +CanPromote() : bool                               \\ \hline
    +IsEssential() : bool                              \\ \hline
    \end{tabular}
\end{table}

\subsubsection{Fields}

\begin{table}[H]
    \begin{tabular}{llllll}
    \hline
    \multicolumn{1}{|l|}{\cellcolor[HTML]{EFEFEF}\textbf{Field Name}} & \multicolumn{1}{l|}{\cellcolor[HTML]{EFEFEF}\textbf{Type}} & \multicolumn{1}{l|}{\cellcolor[HTML]{EFEFEF}\textbf{Visibility}} \\ \hline
    \multicolumn{1}{|l|}{\_colour}                                    & \multicolumn{1}{l|}{Colour}                                & \multicolumn{1}{l|}{Private}                                     \\ \hline
    \end{tabular}
\end{table}

\textbf{Description :} This field represent the colour a piece, wich can only be
black or white as it is for a regular chess game.


\subsubsection{Methods}

\begin{table}[H]
    \begin{tabular}{|l|l|l|l|}
    \hline
    \rowcolor[HTML]{EFEFEF} 
    \cellcolor[HTML]{EFEFEF}\textbf{Method Name} & \textbf{Parameters}  & \textbf{Returned Type} & \textbf{Visibility} \\ \hline
    ValidMove                          & x1, y1, x2, y2 : int & bool                   & Public              \\ \hline
    \end{tabular}
\end{table}

\textbf{Parameters :} x1 ans y1 represent the coordinates of the position before a possible move
and x2 and y2 are the coordinates of the position after the move. These are all of Integer(16) type.
\\
\textbf{Description :} The method returns true if the piece can move to the 2nd position.

\begin{table}[H]
    \begin{tabular}{|l|l|l|l|}
    \hline
    \rowcolor[HTML]{EFEFEF} 
    \cellcolor[HTML]{EFEFEF}\textbf{Method Name} & \textbf{Parameters}  & \textbf{Returned Type} & \textbf{Visibility} \\ \hline
    CanCollide                                   & None                 & bool                   & Public              \\ \hline
    \end{tabular}
\end{table}

\textbf{Description :} Returns true if the piece can't go over other pieces, otherwise
it returns false.

\begin{table}[H]
    \begin{tabular}{|l|l|l|l|}
    \hline
    \rowcolor[HTML]{EFEFEF} 
    \cellcolor[HTML]{EFEFEF}\textbf{Method Name} & \textbf{Parameters}  & \textbf{Returned Type} & \textbf{Visibility} \\ \hline
    ToString                                   & None                 & String                   & Public              \\ \hline
    \end{tabular}
\end{table}

\textbf{Description :} Empty method te be overrided by child classes.

\begin{table}[H]
    \begin{tabular}{|l|l|l|l|}
    \hline
    \rowcolor[HTML]{EFEFEF} 
    \cellcolor[HTML]{EFEFEF}\textbf{Method Name} & \textbf{Parameters}  & \textbf{Returned Type} & \textbf{Visibility} \\ \hline
    CanPromote                                   & None                 & bool                   & Public              \\ \hline
    \end{tabular}
\end{table}

\textbf{Description :} Returns true if the piece is promotable.

\begin{table}[H]
    \begin{tabular}{|l|l|l|l|}
    \hline
    \rowcolor[HTML]{EFEFEF} 
    \cellcolor[HTML]{EFEFEF}\textbf{Method Name} & \textbf{Parameters}  & \textbf{Returned Type} & \textbf{Visibility} \\ \hline
    IsEssential                                  & None                 & bool                   & Public              \\ \hline
    \end{tabular}
\end{table}

\textbf{Description :} Returns true if the piece is essential.

\subsubsection{Properties}

\begin{table}[H]
    \begin{tabular}{|l|l|l|l|}
    \hline
    \rowcolor[HTML]{EFEFEF} 
    \cellcolor[HTML]{EFEFEF}\textbf{Property Name} & \textbf{Parameters}  & \textbf{Returned Type} & \textbf{Visibility} \\ \hline
    Colour                                         & None                 & Colour                 & Public              \\ \hline
    \end{tabular}
\end{table}

\textbf{Description :} Gets the colour of a piece.
\newpage

%%%%%%%%%%%%%%%
%StartingPiece%

\subsection{Class StartingPiece : Piece}

Another abstract Class that inherits to the base Class Piece.
This Class is for pieces that have different behaviour after they
have mouved for the first time.

\begin{table}[H]
    \begin{tabular}{|l|}
    \hline
    \cellcolor[HTML]{C0C0C0}\textbf{(Abstract) StartingPiece}            \\ \hline
    \cellcolor[HTML]{EFEFEF}-\_hasMoved : bool        \\ \hline
    \end{tabular}
\end{table}

\subsubsection{Fields}

\begin{table}[H]
    \begin{tabular}{llllll}
    \hline
    \multicolumn{1}{|l|}{\cellcolor[HTML]{EFEFEF}\textbf{Field Name}} & \multicolumn{1}{l|}{\cellcolor[HTML]{EFEFEF}\textbf{Type}} & \multicolumn{1}{l|}{\cellcolor[HTML]{EFEFEF}\textbf{Visibility}} \\ \hline
    \multicolumn{1}{|l|}{\_hasMoved}                                  & \multicolumn{1}{l|}{bool}                                & \multicolumn{1}{l|}{Private}                                     \\ \hline
    \end{tabular}
\end{table}

\textbf{Description :} This field is used to tell if the piece has
already made its first move.

\subsubsection{Methods}
Same as Class Piece.
\subsubsection{Properties}

\begin{table}[H]
    \begin{tabular}{|l|l|l|l|}
    \hline
    \rowcolor[HTML]{EFEFEF} 
    \cellcolor[HTML]{EFEFEF}\textbf{Property Name} & \textbf{Parameters}  & \textbf{Returned Type} & \textbf{Visibility} \\ \hline
    HasMoved                                       & None                 & bool                   & Public              \\ \hline
    \end{tabular}
\end{table}

\textbf{Description :} Gets or Sets the property \_hasMoved of a piece to true or false.
\newpage

%%%%%%%%%%%%
%Pawn Class%

\subsection{Class Pawn : StartingPiece}

This Class inherits the Class StartingPiece. It represents the pawn 
piece of a regular chess game.
\begin{table}[H]
    \begin{tabular}{|l|}
    \hline
    \cellcolor[HTML]{C0C0C0}\textbf{Pawn} \\ \hline
    \cellcolor[HTML]{EFEFEF}                    \\ \hline
    +ValidMove(x1, y1, x2, y2) : bool           \\ \hline
    +CanPromote() : bool                        \\ \hline
    +ToString() : string                        \\ \hline
    \end{tabular}
\end{table}

\subsubsection{Fields}

\textbf{Description :} This Class doesn't provide a new field. It just
inherits the fields of its parent.

\subsubsection{Methods}

\begin{table}[H]
    \begin{tabular}{|l|l|l|l|}
    \hline
    \rowcolor[HTML]{EFEFEF} 
    \cellcolor[HTML]{EFEFEF}\textbf{Method Name} & \textbf{Parameters}  & \textbf{Returned Type} & \textbf{Visibility} \\ \hline
    ValidMove                          & x1, y1, x2, y2 : int & bool                   & Public              \\ \hline
    \end{tabular}
\end{table}

\textbf{Parameters :} x1 ans y1 represent the coordinates of the position before a possible move
and x2 and y2 are the coordinates of the position after the move. These are all of Integer(16) type.
\\
\textbf{Description :} This method checks
if the move provided is valid considering the basic moves of a pawn which
are 1 or 2 cells ahead, or 1 diagonal cell if an opponent piece is
present. In thoses cases it returns true, otherwise it returns false. 

\begin{table}[H]
    \begin{tabular}{|l|l|l|l|}
    \hline
    \rowcolor[HTML]{EFEFEF} 
    \cellcolor[HTML]{EFEFEF}\textbf{Method Name} & \textbf{Parameters}  & \textbf{Returned Type} & \textbf{Visibility} \\ \hline
    CanPromote                                   & None                 & bool                   & Public              \\ \hline
    \end{tabular}
\end{table}

\textbf{Description :} Returns true if the piece can promote.

\begin{table}[H]
    \begin{tabular}{|l|l|l|l|}
    \hline
    \rowcolor[HTML]{EFEFEF} 
    \cellcolor[HTML]{EFEFEF}\textbf{Method Name} & \textbf{Parameters}  & \textbf{Returned Type} & \textbf{Visibility} \\ \hline
    ToString                                   & None                 & String                   & Public              \\ \hline
    \end{tabular}
\end{table}

\textbf{Description :} This method overrides ToString virtual
method and returns minus "p" if the piece is white and upper "P"
if it is black.

\subsubsection{Properties}

Same as parent.
\newpage

%%%%%%%%%%%%
%Rook Class%

\subsection{Class Rook : StartingPiece}

This Class inherits the Class StartingPiece. It represents the rook 
piece of a regular chess game.
\begin{table}[H]
    \begin{tabular}{|l|}
    \hline
    \cellcolor[HTML]{C0C0C0}\textbf{Rook} \\ \hline
    \cellcolor[HTML]{EFEFEF}                    \\ \hline
    +ValidMove(x1, y1, x2, y2) : bool           \\ \hline
    +ToString() : string                        \\ \hline
    \end{tabular}
\end{table}

\subsubsection{Fields}

\textbf{Description :} This Class doesn't provide a new field. It just
inherits the fields of its parent.

\subsubsection{Methods}

\begin{table}[H]
    \begin{tabular}{|l|l|l|l|}
    \hline
    \rowcolor[HTML]{EFEFEF} 
    \cellcolor[HTML]{EFEFEF}\textbf{Method Name} & \textbf{Parameters}  & \textbf{Returned Type} & \textbf{Visibility} \\ \hline
    ValidMove                          & x1, y1, x2, y2 : int & bool                   & Public              \\ \hline
    \end{tabular}
\end{table}

\textbf{Parameters :} x1 ans y1 represent the coordinates of the position before a possible move
and x2 and y2 are the coordinates of the position after the move. These are all of Integer(16) type.
\\
\textbf{Description :} This method checks if the move provided is
valid considering the basic moves of a rook which are forward,
backward or sideways to any empty cell. In thoses cases it returns
true, otherwise it returns false. 

\begin{table}[H]
    \begin{tabular}{|l|l|l|l|}
    \hline
    \rowcolor[HTML]{EFEFEF} 
    \cellcolor[HTML]{EFEFEF}\textbf{Method Name} & \textbf{Parameters}  & \textbf{Returned Type} & \textbf{Visibility} \\ \hline
    ToString                                   & None                 & String                   & Public              \\ \hline
    \end{tabular}
\end{table}

\textbf{Description :} This method overrides ToString virtual
method and returns minus "r" if the piece is white and upper "R"
if it is black.

\subsubsection{Properties}

Same as parent.
\newpage

%%%%%%%%%%%%
%King Class%

\subsection{Class King : StartingPiece}

This Class inherits the Class StartingPiece. It represents the king 
piece of a regular chess game.
\begin{table}[H]
    \begin{tabular}{|l|}
    \hline
    \cellcolor[HTML]{C0C0C0}\textbf{King} \\ \hline
    \cellcolor[HTML]{EFEFEF}                    \\ \hline
    +ValidMove(x1, y1, x2, y2) : bool           \\ \hline
    +IsEssential() : bool                       \\ \hline
    +ToString() : string                        \\ \hline
    \end{tabular}
\end{table}

\subsubsection{Fields}

\textbf{Description :} This Class doesn't provide a new field. It just
inherits the fields of its parent.

\subsubsection{Methods}

\begin{table}[H]
    \begin{tabular}{|l|l|l|l|}
    \hline
    \rowcolor[HTML]{EFEFEF} 
    \cellcolor[HTML]{EFEFEF}\textbf{Method Name} & \textbf{Parameters}  & \textbf{Returned Type} & \textbf{Visibility} \\ \hline
    ValidMove                          & x1, y1, x2, y2 : int & bool                   & Public              \\ \hline
    \end{tabular}
\end{table}

\textbf{Parameters :} x1 ans y1 represent the coordinates of the position before a possible move
and x2 and y2 are the coordinates of the position after the move. These are all of Integer(16) type.
\\
\textbf{Description :} This method checks if the move provided is
valid considering the basic moves of a king which are one square
horizontally, vertically, or diagonally unless the square is already
occupied by a friendly piece or the move would place the king in
check. In thoses cases it returns true, otherwise it returns false.

\begin{table}[H]
    \begin{tabular}{|l|l|l|l|}
    \hline
    \rowcolor[HTML]{EFEFEF} 
    \cellcolor[HTML]{EFEFEF}\textbf{Method Name} & \textbf{Parameters}  & \textbf{Returned Type} & \textbf{Visibility} \\ \hline
    IsEssential                                  & None                 & bool                   & Public              \\ \hline
    \end{tabular}
\end{table}

\textbf{Description :} Returns true if the piece is essential.

\begin{table}[H]
    \begin{tabular}{|l|l|l|l|}
    \hline
    \rowcolor[HTML]{EFEFEF} 
    \cellcolor[HTML]{EFEFEF}\textbf{Method Name} & \textbf{Parameters}  & \textbf{Returned Type} & \textbf{Visibility} \\ \hline
    ToString                                   & None                 & String                   & Public              \\ \hline
    \end{tabular}
\end{table}

\textbf{Description :} This method overrides ToString virtual
method and returns minus "k" if the piece is white and upper "K"
if it is black.

\subsubsection{Properties}

Same as parent.
\newpage

%%%%%%%%%%%%%%
%Knight Class%

\subsection{Class Knight : Piece}

This Class inherits the Class Piece. It represents the knight 
piece of a regular chess game.
\begin{table}[H]
    \begin{tabular}{|l|}
    \hline
    \cellcolor[HTML]{C0C0C0}\textbf{Knight} \\ \hline
    \cellcolor[HTML]{EFEFEF}                    \\ \hline
    +ValidMove(x1, y1, x2, y2) : bool           \\ \hline
    +CanCollide() : bool                        \\ \hline
    +ToString() : string                        \\ \hline
    \end{tabular}
\end{table}

\subsubsection{Fields}

\textbf{Description :} This Class doesn't provide a new field. It just
inherits the fields of its parent.

\subsubsection{Methods}

\begin{table}[H]
    \begin{tabular}{|l|l|l|l|}
    \hline
    \rowcolor[HTML]{EFEFEF} 
    \cellcolor[HTML]{EFEFEF}\textbf{Method Name} & \textbf{Parameters}  & \textbf{Returned Type} & \textbf{Visibility} \\ \hline
    ValidMove                          & x1, y1, x2, y2 : int & bool                   & Public              \\ \hline
    \end{tabular}
\end{table}

\textbf{Parameters :} x1 ans y1 represent the coordinates of the position before a possible move
and x2 and y2 are the coordinates of the position after the move. These are all of Integer(16) type.
\\
\textbf{Description :} This method checks if the move provided is valid considering the basic moves of a knight which
are “L-shape”—that is, they can move two squares in any direction
vertically followed by one square horizontally, or two squares in any
direction horizontally followed by one square vertically. In thoses 
cases it returns true, otherwise it returns false. 

\begin{table}[H]
    \begin{tabular}{|l|l|l|l|}
    \hline
    \rowcolor[HTML]{EFEFEF} 
    \cellcolor[HTML]{EFEFEF}\textbf{Method Name} & \textbf{Parameters}  & \textbf{Returned Type} & \textbf{Visibility} \\ \hline
    CanCollide                                   & None                 & bool                   & Public              \\ \hline
    \end{tabular}
\end{table}

\textbf{Description :} Returns true if the piece can collide.

\begin{table}[H]
    \begin{tabular}{|l|l|l|l|}
    \hline
    \rowcolor[HTML]{EFEFEF} 
    \cellcolor[HTML]{EFEFEF}\textbf{Method Name} & \textbf{Parameters}  & \textbf{Returned Type} & \textbf{Visibility} \\ \hline
    ToString                                   & None                 & String                   & Public              \\ \hline
    \end{tabular}
\end{table}

\textbf{Description :} This method overrides ToString virtual
method and returns minus "n" if the piece is white and upper "N"
if it is black.

\subsubsection{Properties}

Same as parent.

\newpage

%Bishop Class%
%%%%%%%%%%%%%%

\subsection{Class Bishop : Piece}

This Class inherits the Class Piece. It represents the bishop 
piece of a regular chess game.
\begin{table}[H]
    \begin{tabular}{|l|}
    \hline
    \cellcolor[HTML]{C0C0C0}\textbf{Bishop} \\ \hline
    \cellcolor[HTML]{EFEFEF}                    \\ \hline
    +ValidMove(x1, y1, x2, y2) : bool           \\ \hline
    +ToString() : string                        \\ \hline
    \end{tabular}
\end{table}

\subsubsection{Fields}

\textbf{Description :} This Class doesn't provide a new field. It just
inherits the fields of its parent.

\subsubsection{Methods}

\begin{table}[H]
    \begin{tabular}{|l|l|l|l|}
    \hline
    \rowcolor[HTML]{EFEFEF} 
    \cellcolor[HTML]{EFEFEF}\textbf{Method Name} & \textbf{Parameters}  & \textbf{Returned Type} & \textbf{Visibility} \\ \hline
    ValidMove                          & x1, y1, x2, y2 : int & bool                   & Public              \\ \hline
    \end{tabular}
\end{table}

\textbf{Parameters :} x1 ans y1 represent the coordinates of the position before a possible move
and x2 and y2 are the coordinates of the position after the move. These are all of Integer(16) type.
\\
\textbf{Description :} This method checks if the move provided is valid considering the basic moves of a bishop which
are any direction diagonally with no limit of cells unless there is another piece obstructing its path. In thoses 
cases it returns true, otherwise it returns false. 

\begin{table}[H]
    \begin{tabular}{|l|l|l|l|}
    \hline
    \rowcolor[HTML]{EFEFEF} 
    \cellcolor[HTML]{EFEFEF}\textbf{Method Name} & \textbf{Parameters}  & \textbf{Returned Type} & \textbf{Visibility} \\ \hline
    ToString                                   & None                 & String                   & Public              \\ \hline
    \end{tabular}
\end{table}

\textbf{Description :} This method overrides ToString virtual
method and returns minus "b" if the piece is white and upper "B"
if it is black.

\subsubsection{Properties}

Same as parent.

\newpage

%Queen%
%%%%%%%

\subsection{Class Queen : Piece}

This Class inherits the Class Piece. It represents the queen 
piece of a regular chess game.
\begin{table}[H]
    \begin{tabular}{|l|}
    \hline
    \cellcolor[HTML]{C0C0C0}\textbf{Queen} \\ \hline
    \cellcolor[HTML]{EFEFEF}                    \\ \hline
    +ValidMove(x1, y1, x2, y2) : bool           \\ \hline
    +ToString() : string                        \\ \hline
    \end{tabular}
\end{table}

\subsubsection{Fields}

\textbf{Description :} This Class doesn't provide a new field. It just
inherits the fields of its parent.

\subsubsection{Methods}

\begin{table}[H]
    \begin{tabular}{|l|l|l|l|}
    \hline
    \rowcolor[HTML]{EFEFEF} 
    \cellcolor[HTML]{EFEFEF}\textbf{Method Name} & \textbf{Parameters}  & \textbf{Returned Type} & \textbf{Visibility} \\ \hline
    ValidMove                          & x1, y1, x2, y2 : int & bool                   & Public              \\ \hline
    \end{tabular}
\end{table}

\textbf{Parameters :} x1 ans y1 represent the coordinates of the position before a possible move
and x2 and y2 are the coordinates of the position after the move. These are all of Integer(16) type.
\\
\textbf{Description :} This method checks if the move provided is valid considering the basic moves of a bishop which
are any direction diagonally with no limit of cells unless there is another piece obstructing its path. In thoses 
cases it returns true, otherwise it returns false. 

\begin{table}[H]
    \begin{tabular}{|l|l|l|l|}
    \hline
    \rowcolor[HTML]{EFEFEF} 
    \cellcolor[HTML]{EFEFEF}\textbf{Method Name} & \textbf{Parameters}  & \textbf{Returned Type} & \textbf{Visibility} \\ \hline
    ToString                                   & None                 & String                   & Public              \\ \hline
    \end{tabular}
\end{table}

\textbf{Description :} This method overrides ToString virtual
method and returns minus "q" if the piece is white and upper "Q"
if it is black.

\subsubsection{Properties}

Same as parent.

\newpage

%Match Class%
%%%%%%%%%%%%%

\subsection{Class Match}

This Class represents the model of a chess match wich compose the
GameController. It will keep all changes that the game controller will return.

\begin{table}[H]
    \begin{tabular}{|l|}
    \hline
    \rowcolor[HTML]{C0C0C0} 
    \textbf{Match}                                    \\ \hline
    \rowcolor[HTML]{EFEFEF} 
    -\_board : Board                                  \\ \hline
    \rowcolor[HTML]{EFEFEF} 
    -\_current : Colour                               \\ \hline
    \rowcolor[HTML]{EFEFEF} 
    -\_history : string{[}{]}                         \\ \hline
    \rowcolor[HTML]{EFEFEF} 
    -\_turnNumber : int                               \\ \hline
    +ExportBoard() : string                           \\ \hline
    +ExportHistory() : string[]                       \\ \hline
    +ValidTurn(int origin, int target) : bool         \\ \hline
    +MakeTurn(int origin, int target) : void          \\ \hline
    +ValidSelection(int cell, bool firstClick) : bool \\ \hline
    +HasPromotable(int target) : bool                 \\ \hline
    +Check() : bool                                   \\ \hline
    +Checkmate() : bool                               \\ \hline
    +Stalemate() : bool                               \\ \hline
    +Castle(int origin, int target) : void            \\ \hline
    +GetAssailants(Colour) : List\textless{} int \textgreater{}                \\ \hline
\end{tabular}
\end{table}

\subsubsection{Fields}

\begin{table}[H]
    \begin{tabular}{llllll}
    \hline
    \multicolumn{1}{|l|}{\cellcolor[HTML]{EFEFEF}\textbf{Field Name}} & \multicolumn{1}{l|}{\cellcolor[HTML]{EFEFEF}\textbf{Type}} & \multicolumn{1}{l|}{\cellcolor[HTML]{EFEFEF}\textbf{Visibility}} \\ \hline
    \multicolumn{1}{|l|}{\_board}                                     & \multicolumn{1}{l|}{Board}                                 & \multicolumn{1}{l|}{Private}                                     \\ \hline
    \end{tabular}
\end{table}

\textbf{Description :} This field represent the board of a match wich compose
the match. Its type is Board wich is the next class to discuss.

\begin{table}[H]
    \begin{tabular}{llllll}
    \hline
    \multicolumn{1}{|l|}{\cellcolor[HTML]{EFEFEF}\textbf{Field Name}} & \multicolumn{1}{l|}{\cellcolor[HTML]{EFEFEF}\textbf{Type}} & \multicolumn{1}{l|}{\cellcolor[HTML]{EFEFEF}\textbf{Visibility}} \\ \hline
    \multicolumn{1}{|l|}{\_current}                                     & \multicolumn{1}{l|}{Colour}                                 & \multicolumn{1}{l|}{Private}                                     \\ \hline
    \end{tabular}
\end{table}

\textbf{Description :} This field tells us wich piece colour is
currently playing, white or black.

\begin{table}[H]
    \begin{tabular}{llllll}
    \hline
    \multicolumn{1}{|l|}{\cellcolor[HTML]{EFEFEF}\textbf{Field Name}} & \multicolumn{1}{l|}{\cellcolor[HTML]{EFEFEF}\textbf{Type}} & \multicolumn{1}{l|}{\cellcolor[HTML]{EFEFEF}\textbf{Visibility}} \\ \hline
    \multicolumn{1}{|l|}{\_history}                                     & \multicolumn{1}{l|}{string[]}                            & \multicolumn{1}{l|}{Private}                                     \\ \hline
    \end{tabular}
\end{table}

\textbf{Description :} This field is a table that contains strings that
represent previous board states to keep track of what has been played. For exemple,
the first string would look like that :
\\"RNBKQBNRPPPPPPPP.................................pppppppprnkqbnr". 

\begin{table}[H]
    \begin{tabular}{llllll}
    \hline
    \multicolumn{1}{|l|}{\cellcolor[HTML]{EFEFEF}\textbf{Field Name}} & \multicolumn{1}{l|}{\cellcolor[HTML]{EFEFEF}\textbf{Type}} & \multicolumn{1}{l|}{\cellcolor[HTML]{EFEFEF}\textbf{Visibility}} \\ \hline
    \multicolumn{1}{|l|}{\_turnNumber}                                & \multicolumn{1}{l|}{int}                                   & \multicolumn{1}{l|}{Private}                                     \\ \hline
    \end{tabular}
\end{table}

\textbf{Description :} This field tracks the number of turns that have been played.

\subsubsection{Methods}

\begin{table}[H]
    \begin{tabular}{|l|l|l|l|}
    \hline
    \rowcolor[HTML]{EFEFEF} 
    \cellcolor[HTML]{EFEFEF}\textbf{Method Name} & \textbf{Parameters}  & \textbf{Returned Type} & \textbf{Visibility} \\ \hline
    ExportBoard                                  & none                 & string                 & Public              \\ \hline
    \end{tabular}
\end{table}

\textbf{Description :} The method returns the the board as a string of 64 char.

\begin{table}[H]
    \begin{tabular}{|l|l|l|l|}
    \hline
    \rowcolor[HTML]{EFEFEF} 
    \cellcolor[HTML]{EFEFEF}\textbf{Method Name} & \textbf{Parameters}  & \textbf{Returned Type} & \textbf{Visibility} \\ \hline
    ExportHistory                                & none                 & string[]                 & Public            \\ \hline
    \end{tabular}
\end{table}

\textbf{Description :} The method returns the table \_history.

\begin{table}[H]
    \begin{tabular}{|l|l|l|l|}
    \hline
    \rowcolor[HTML]{EFEFEF} 
    \cellcolor[HTML]{EFEFEF}\textbf{Method Name} & \textbf{Parameters}    & \textbf{Returned Type} & \textbf{Visibility} \\ \hline
    ValidTurn                                    & int origin, int target & bool                   & Public              \\ \hline
    \end{tabular}
\end{table}

\textbf{Parameters :} Parameter origin represents the cell
where the piece is before the move and target is the targeted cell.
\\

\textbf{Description :} The method returns true if the turn is valid.

\begin{table}[H]
    \begin{tabular}{|l|l|l|l|}
    \hline
    \rowcolor[HTML]{EFEFEF} 
    \cellcolor[HTML]{EFEFEF}\textbf{Method Name} & \textbf{Parameters}    & \textbf{Returned Type} & \textbf{Visibility} \\ \hline
    MakeTurn                                     & int origin, int target & void                   & Public              \\ \hline
    \end{tabular}
\end{table}

\textbf{Parameters :} Parameter origin represents the cell
where the piece is before the move and target is the targeted cell.
\\

\textbf{Description :} The method makes all changes to the board in order to make the turn.

\begin{table}[H]
    \begin{tabular}{|l|l|l|l|}
    \hline
    \rowcolor[HTML]{EFEFEF} 
    \cellcolor[HTML]{EFEFEF}\textbf{Method Name} & \textbf{Parameters}       & \textbf{Returned Type} & \textbf{Visibility} \\ \hline
    ValidSelection                               & int cell, bool firstClick & bool                   & Public              \\ \hline
    \end{tabular}
\end{table}

\textbf{Parameters :} Parameter cell represents the cell
where the player clicks and the parameter firstClick is a bool
that returns true if its the first click.
\\

\textbf{Description :} The method checks if the selection made
in the board is valid.

\begin{table}[H]
    \begin{tabular}{|l|l|l|l|}
    \hline
    \rowcolor[HTML]{EFEFEF} 
    \cellcolor[HTML]{EFEFEF}\textbf{Method Name} & \textbf{Parameters}     & \textbf{Returned Type} & \textbf{Visibility} \\ \hline
    HasPromotable                                & int target                    & bool                   & Public              \\ \hline
    \end{tabular}
\end{table}


\textbf{Parameters :} The parameter target represents the targeted cell for a piece move.
\textbf{Description :} This method returns true if the cell contains a promotable piece.

\begin{table}[H]
    \begin{tabular}{|l|l|l|l|}
    \hline
    \rowcolor[HTML]{EFEFEF} 
    \cellcolor[HTML]{EFEFEF}\textbf{Method Name} & \textbf{Parameters}     & \textbf{Returned Type} & \textbf{Visibility} \\ \hline
    Check                                        & none                    & bool                   & Public              \\ \hline
    \end{tabular}
\end{table}

\textbf{Description :} This method returns true if the current turn makes a check.

\begin{table}[H]
    \begin{tabular}{|l|l|l|l|}
    \hline
    \rowcolor[HTML]{EFEFEF} 
    \cellcolor[HTML]{EFEFEF}\textbf{Method Name} & \textbf{Parameters}     & \textbf{Returned Type} & \textbf{Visibility} \\ \hline
    Checkmate                                        & none                    & bool                   & Public              \\ \hline
    \end{tabular}
\end{table}

\textbf{Description :} This method returns true if the current turn makes a checkmate.

\begin{table}[H]
    \begin{tabular}{|l|l|l|l|}
    \hline
    \rowcolor[HTML]{EFEFEF} 
    \cellcolor[HTML]{EFEFEF}\textbf{Method Name} & \textbf{Parameters}     & \textbf{Returned Type} & \textbf{Visibility} \\ \hline
    Stalemate                                    & none                    & bool                   & Public              \\ \hline
    \end{tabular}
\end{table}

\textbf{Description :} This method returns true if the current turn makes a Stalemate.

\begin{table}[H]
    \begin{tabular}{|l|l|l|l|}
    \hline
    \rowcolor[HTML]{EFEFEF} 
    \cellcolor[HTML]{EFEFEF}\textbf{Method Name} & \textbf{Parameters}      & \textbf{Returned Type} & \textbf{Visibility} \\ \hline
    Castle                                       & int origin, int target   & bool                   & Public              \\ \hline
    \end{tabular}
\end{table}

\textbf{Description :} This method returns true if the move will make a castle.

\begin{table}[H]
    \begin{tabular}{|l|l|l|l|}
    \hline
    \rowcolor[HTML]{EFEFEF} 
    \cellcolor[HTML]{EFEFEF}\textbf{Method Name} & \textbf{Parameters}     & \textbf{Returned Type} & \textbf{Visibility} \\ \hline
    GetAsaillants                                & Colour colour           & List\textless{}int\textgreater{}                   & Public              \\ \hline
    \end{tabular}
\end{table}

\textbf{Parameters :} The parameter colour represents the piece colour specified.
\textbf{Description :} This method returns a list of all possible asaillants by the index of their cell.

\subsubsection{Properties}

\begin{table}[H]
    \begin{tabular}{|l|l|l|l|}
    \hline
    \rowcolor[HTML]{EFEFEF} 
    \cellcolor[HTML]{EFEFEF}\textbf{Property Name} & \textbf{Parameters}  & \textbf{Returned Type} & \textbf{Visibility} \\ \hline
    Current                                        & None                 & Colour                 & Public              \\ \hline
    \end{tabular}
\end{table}

\textbf{Description :} Gets or Sets the property \_current of a match (Colour white or black).

\newpage

%Board Class%
%%%%%%%%%%%%%

\subsection{Class Board}

This Class represents the model of a chess board wich compose the
match. It will return all changes to the match.

\begin{table}[H]
    \begin{tabular}{|l|}
    \hline
    \rowcolor[HTML]{C0C0C0} 
    \textbf{Board}                                             \\ \hline
    \rowcolor[HTML]{EFEFEF}                                    
    -\_cells : Cell[]                                          \\ \hline
    +ToString() : string                                       \\ \hline
    +Collision(int origin, int target) : bool                  \\ \hline
    +SameColour (int cell, Colour colour) : bool               \\ \hline
    +ValidMove(int origin, int target) : bool                  \\ \hline
    +MoveCellTo(int origin, int target) : void                 \\ \hline
    +GenerateBoard(string board) : void                        \\ \hline
    +IsEssetialExposed(Colour colour) : bool                   \\ \hline
    +HasPromotable(int target) : bool                          \\ \hline
    +GetAssaillants(Colour colour) : List\textless{}int\textgreater{}                 \\ \hline
    -GetEssentialPiece(Colour colour) : int                    \\ \hline
    -GetAttackingPieces(Colour colour, int target) : List\textless{}int\textgreater{} \\ \hline
    -HasAttackersAroundEssential(Colour colour) : bool          \\ \hline
    \end{tabular}
\end{table}

\subsubsection{Fields}

\begin{table}[H]
    \begin{tabular}{llllll}
    \hline
    \multicolumn{1}{|l|}{\cellcolor[HTML]{EFEFEF}\textbf{Field Name}} & \multicolumn{1}{l|}{\cellcolor[HTML]{EFEFEF}\textbf{Type}} & \multicolumn{1}{l|}{\cellcolor[HTML]{EFEFEF}\textbf{Visibility}} \\ \hline
    \multicolumn{1}{|l|}{\_cells}                                     & \multicolumn{1}{l|}{Cell[]}                                & \multicolumn{1}{l|}{Private}                                     \\ \hline
    \end{tabular}
\end{table}

\textbf{Description :} This field represent all 64 cells on a chess board. Each cell will contain a piece or be empty.

\subsubsection{Methods}

\begin{table}[H]
    \begin{tabular}{|l|l|l|l|}
    \hline
    \rowcolor[HTML]{EFEFEF} 
    \cellcolor[HTML]{EFEFEF}\textbf{Method Name} & \textbf{Parameters}    & \textbf{Returned Type} & \textbf{Visibility} \\ \hline
    Collision                                    & int origin, int target & bool                   & Public              \\ \hline
    \end{tabular}
\end{table}

\textbf{Parameters :} Parameter origin represents the cell
where the piece is before the move and target is the targeted
cell.
\\

\textbf{Description :} This method returns true if it detects a
possible collision in the path to the targeted cell.

\begin{table}[H]
    \begin{tabular}{|l|l|l|l|}
    \hline
    \rowcolor[HTML]{EFEFEF} 
    \cellcolor[HTML]{EFEFEF}\textbf{Method Name} & \textbf{Parameters}     & \textbf{Returned Type} & \textbf{Visibility} \\ \hline
    SameColour                                   & int cell, Colour colour & bool                   & Public              \\ \hline
    \end{tabular}
\end{table}

\textbf{Parameters :} Parameter cell represents the index of one of
the 64 cells that is selected and colour is the colour of the selected cell.
\\

\textbf{Description :} This method returns true the cell selected
contains a piece at the same colour of the current turn.

\begin{table}[H]
    \begin{tabular}{|l|l|l|l|}
    \hline
    \rowcolor[HTML]{EFEFEF} 
    \cellcolor[HTML]{EFEFEF}\textbf{Method Name} & \textbf{Parameters}     & \textbf{Returned Type} & \textbf{Visibility} \\ \hline
    ValidMove                                   & int origin, int target   & bool                   & Public              \\ \hline
    \end{tabular}
\end{table}

\textbf{Parameters :} Parameter origin represents the index of
the origin cell and the target parameter is the cell targeted.
\\

\textbf{Description :} This method returns true if the move is valid.

\begin{table}[H]
    \begin{tabular}{|l|l|l|l|}
    \hline
    \rowcolor[HTML]{EFEFEF} 
    \cellcolor[HTML]{EFEFEF}\textbf{Method Name} & \textbf{Parameters}     & \textbf{Returned Type} & \textbf{Visibility} \\ \hline
    MoveCellTo                                   & int origin, int target  & bool                   & Public              \\ \hline
    \end{tabular}
\end{table}

\textbf{Parameters :} Parameter origin represents the index of
the origin cell and the target parameter is the cell targeted.
\\

\textbf{Description :} This method swaps cells origin and target.

\begin{table}[H]
    \begin{tabular}{|l|l|l|l|}
    \hline
    \rowcolor[HTML]{EFEFEF} 
    \cellcolor[HTML]{EFEFEF}\textbf{Method Name} & \textbf{Parameters}     & \textbf{Returned Type} & \textbf{Visibility} \\ \hline
    GenerateBoard                              & string board            & void                   & Public              \\ \hline
    \end{tabular}
\end{table}

\textbf{Parameters :} Board parameter is a 64 char string that represent
a board state(tells where the pieces are supposed to be).
\\

\textbf{Description :} This method takes a string as board and transforms
each of the 64 char as the content of each of the 64 cells.

\begin{table}[H]
    \begin{tabular}{|l|l|l|l|}
    \hline
    \rowcolor[HTML]{EFEFEF} 
    \cellcolor[HTML]{EFEFEF}\textbf{Method Name} & \textbf{Parameters}     & \textbf{Returned Type} & \textbf{Visibility} \\ \hline
    IsEssentialExposed                           & Colour colour           & bool                   & Public              \\ \hline
    \end{tabular}
\end{table}

\textbf{Parameters :} The parameter colour represent the colour
to test.
\\

\textbf{Description :} Checks if the essential piece of the colour white or black
is exposed.

\begin{table}[H]
    \begin{tabular}{|l|l|l|l|}
    \hline
    \rowcolor[HTML]{EFEFEF} 
    \cellcolor[HTML]{EFEFEF}\textbf{Method Name} & \textbf{Parameters}     & \textbf{Returned Type} & \textbf{Visibility} \\ \hline
    HasPromotable                                & int target              & bool                   & Public              \\ \hline
    \end{tabular}
\end{table}

\textbf{Parameters :} The parameter target represent the targeted cell.
\\

\textbf{Description :} Checks if the targeted cell has the promotable attribute.

\begin{table}[H]
    \begin{tabular}{|l|l|l|l|}
    \hline
    \rowcolor[HTML]{EFEFEF} 
    \cellcolor[HTML]{EFEFEF}\textbf{Method Name} & \textbf{Parameters}     & \textbf{Returned Type} & \textbf{Visibility} \\ \hline
    GetAsaillants                                & Colour colour           & List\textless{}int\textgreater{}                   & Public              \\ \hline
    \end{tabular}
\end{table}

\textbf{Parameters :} The parameter colour represents the piece colour specified.
\textbf{Description :} This method returns a list of all possible asaillants by the index of their cell.

\begin{table}[H]
    \begin{tabular}{|l|l|l|l|}
    \hline
    \rowcolor[HTML]{EFEFEF} 
    \cellcolor[HTML]{EFEFEF}\textbf{Method Name} & \textbf{Parameters}     & \textbf{Returned Type} & \textbf{Visibility} \\ \hline
    GetEssentialPiece                            & Colour colour           & int                   & Public              \\ \hline
    \end{tabular}
\end{table}

\textbf{Parameters :} The parameter colour represents the piece colour specified.
\textbf{Description :} This method returns the index of the cell that contains the essential piece.

\begin{table}[H]
    \begin{tabular}{|l|l|l|l|}
    \hline
    \rowcolor[HTML]{EFEFEF} 
    \cellcolor[HTML]{EFEFEF}\textbf{Method Name} & \textbf{Parameters}        & \textbf{Returned Type} & \textbf{Visibility} \\ \hline
    GetAttackingPieces                           & Colour colour, int target  & List\textless{}int\textgreater{}                   & Public              \\ \hline
    \end{tabular}
\end{table}

\textbf{Parameters :} The parameter colour represents the piece colour specified and target is the targeted piece.
\textbf{Description :} This method returns a list of all attacking pieces.

\begin{table}[H]
    \begin{tabular}{|l|l|l|l|}
    \hline
    \rowcolor[HTML]{EFEFEF} 
    \cellcolor[HTML]{EFEFEF}\textbf{Method Name} & \textbf{Parameters}     & \textbf{Returned Type} & \textbf{Visibility} \\ \hline
    HasAttackersAroundEssential                  & Colour colour           & bool                   & Public              \\ \hline
    \end{tabular}
\end{table}

\textbf{Parameters :} The parameter colour represents the piece colour specified.
\textbf{Description :} This method returns true if there's attackers around the essential piece.

\newpage

%Cell Class%
%%%%%%%%%%%%%

\subsection{Class Cell}

This Class represents the model of a chess cell and its content.

\begin{table}[H]
    \begin{tabular}{|l|}
    \hline
    \rowcolor[HTML]{C0C0C0} 
    \textbf{Cell}                                    \\ \hline
    \rowcolor[HTML]{EFEFEF}
    -\_piece : Nullable\textless{}Piece\textgreater{}                        \\ \hline
    +IsEmpty() : bool                                 \\ \hline
    +HasCollision() : bool                            \\ \hline
    +HasPromotable() : bool                           \\ \hline
    +HasEssetial() : bool                             \\ \hline
    +ValidMove(int x1, int y1, int x2, int y2) : bool \\ \hline
    +Colour() : Colour                                \\ \hline
    \end{tabular}
\end{table}

\subsubsection{Fields}

\begin{table}[H]
    \begin{tabular}{llllll}
    \hline
    \multicolumn{1}{|l|}{\cellcolor[HTML]{EFEFEF}\textbf{Field Name}} & \multicolumn{1}{l|}{\cellcolor[HTML]{EFEFEF}\textbf{Type}} & \multicolumn{1}{l|}{\cellcolor[HTML]{EFEFEF}\textbf{Visibility}} \\ \hline
    \multicolumn{1}{|l|}{\_piece}                                     & \multicolumn{1}{l|}{Nullable\textless{}Piece\textgreater{}}                       & \multicolumn{1}{l|}{Private}                                     \\ \hline
    \end{tabular}
\end{table}

\textbf{Description :} This field represents de content of a cell if it is a piece
or null if it's empty.

\subsubsection{Methods}

\begin{table}[H]
    \begin{tabular}{|l|l|l|l|}
    \hline
    \rowcolor[HTML]{EFEFEF} 
    \cellcolor[HTML]{EFEFEF}\textbf{Method Name} & \textbf{Parameters}    & \textbf{Returned Type} & \textbf{Visibility} \\ \hline
    IsEmpty                                      & none                   & bool                   & Public              \\ \hline
    \end{tabular}
\end{table}

\textbf{Description :} This method returns true if the \_piece is null, so it's empty.

\begin{table}[H]
    \begin{tabular}{|l|l|l|l|}
    \hline
    \rowcolor[HTML]{EFEFEF} 
    \cellcolor[HTML]{EFEFEF}\textbf{Method Name} & \textbf{Parameters}     & \textbf{Returned Type} & \textbf{Visibility} \\ \hline
    HasCollision                                 & none                    & bool                   & Public              \\ \hline
    \end{tabular}
\end{table}

\textbf{Description :} This method returns true if the cell contains a piece that can collide.

\begin{table}[H]
    \begin{tabular}{|l|l|l|l|}
    \hline
    \rowcolor[HTML]{EFEFEF} 
    \cellcolor[HTML]{EFEFEF}\textbf{Method Name} & \textbf{Parameters}     & \textbf{Returned Type} & \textbf{Visibility} \\ \hline
    HasPromotable                                & none                    & bool                   & Public              \\ \hline
    \end{tabular}
\end{table}

\textbf{Description :} This method returns true if the cell contains a promotable piece.

\begin{table}[H]
    \begin{tabular}{|l|l|l|l|}
    \hline
    \rowcolor[HTML]{EFEFEF} 
    \cellcolor[HTML]{EFEFEF}\textbf{Method Name} & \textbf{Parameters}     & \textbf{Returned Type} & \textbf{Visibility} \\ \hline
    HasEssential                                 & none                    & bool                   & Public              \\ \hline
    \end{tabular}
\end{table}

\textbf{Description :} This method returns true if the cell contains a piece that is essential.

\begin{table}[H]
    \begin{tabular}{|l|l|l|l|}
    \hline
    \rowcolor[HTML]{EFEFEF} 
    \cellcolor[HTML]{EFEFEF}\textbf{Method Name} & \textbf{Parameters}            & \textbf{Returned Type} & \textbf{Visibility} \\ \hline
    ValidMove                                    & int x1, int y1, int x2, int y2 & bool                   & Public              \\ \hline
    \end{tabular}
\end{table}

\textbf{Parameters :} x1 ans y1 represent the coordinates of the position before a possible move
and x2 and y2 are the coordinates of the position after the move. These are all of Integer(16) type.
\\

\textbf{Description :} This method returns true if the move is valid.

\begin{table}[H]
    \begin{tabular}{|l|l|l|l|}
    \hline
    \rowcolor[HTML]{EFEFEF} 
    \cellcolor[HTML]{EFEFEF}\textbf{Method Name} & \textbf{Parameters}     & \textbf{Returned Type} & \textbf{Visibility} \\ \hline
    Colour                                       & none                    & Colour                   & Public              \\ \hline
    \end{tabular}
\end{table}

\textbf{Description :} This method returns the colour if the field \_piece is not null.

\newpage

%Player Class%
%%%%%%%%%%%%%%

\subsection{Class Player}

This Class represents a player in a chess game.

\begin{table}[H]
    \begin{tabular}{|l|}
    \hline
    \rowcolor[HTML]{C0C0C0} 
    \textbf{Player}          \\ \hline
    \rowcolor[HTML]{EFEFEF}
    -\_name : string        \\ \hline
    -\_points : int         \\ \hline
    \end{tabular}
\end{table}

\subsubsection{Fields}

\begin{table}[H]
    \begin{tabular}{llllll}
    \hline
    \multicolumn{1}{|l|}{\cellcolor[HTML]{EFEFEF}\textbf{Field Name}} & \multicolumn{1}{l|}{\cellcolor[HTML]{EFEFEF}\textbf{Type}} & \multicolumn{1}{l|}{\cellcolor[HTML]{EFEFEF}\textbf{Visibility}} \\ \hline
    \multicolumn{1}{|l|}{\_name}                                      & \multicolumn{1}{l|}{string}                                & \multicolumn{1}{l|}{Private}                                     \\ \hline
    \end{tabular}
\end{table}

\textbf{Description :} This field is the name of the player.

\begin{table}[H]
    \begin{tabular}{llllll}
    \hline
    \multicolumn{1}{|l|}{\cellcolor[HTML]{EFEFEF}\textbf{Field Name}} & \multicolumn{1}{l|}{\cellcolor[HTML]{EFEFEF}\textbf{Type}} & \multicolumn{1}{l|}{\cellcolor[HTML]{EFEFEF}\textbf{Visibility}} \\ \hline
    \multicolumn{1}{|l|}{\_points}                                      & \multicolumn{1}{l|}{int}                                 & \multicolumn{1}{l|}{Private}                                   \\ \hline
    \end{tabular}
\end{table}

\textbf{Description :} This field is the points that the player collects during a game.

\newpage

%NEW SECTION CONTROLLERS%
%%%%%%%%%%%%%%%%%%%%%%%%%

\section{Controllers}

%Chess Class%
%%%%%%%%%%%%%

\subsection{Class Chess}

\begin{table}[H]
    \begin{tabular}{|l|}
    \hline
    \rowcolor[HTML]{C0C0C0} 
    \textbf{Chess}                                             \\ \hline
    \rowcolor[HTML]{EFEFEF} 
    -\_listGames : List\textless{}{}GameController\textgreater{}{} \\ \hline
    +main() : void                                             \\ \hline
    +NewGame() : void                                          \\ \hline
    +StartGame(Player{[}2{]} players) : void                   \\ \hline
    +ManagePlayers() : void                                    \\ \hline
    +Exit() : void                                             \\ \hline
    \end{tabular}
\end{table}

\subsubsection{Fields}

\begin{table}[H]
    \begin{tabular}{llllll}
    \hline
    \multicolumn{1}{|l|}{\cellcolor[HTML]{EFEFEF}\textbf{Field Name}} & \multicolumn{1}{l|}{\cellcolor[HTML]{EFEFEF}\textbf{Type}} & \multicolumn{1}{l|}{\cellcolor[HTML]{EFEFEF}\textbf{Visibility}} \\ \hline
    \multicolumn{1}{|l|}{\_listGames}                                 & \multicolumn{1}{l|}{List\textless{}GameController\textgreater{}}                                & \multicolumn{1}{l|}{Private}                                     \\ \hline
    \end{tabular}
\end{table}

\textbf{Description :} This field is a list of the active games.

\subsubsection{Methods}

\begin{table}[H]
    \begin{tabular}{|l|l|l|l|}
    \hline
    \rowcolor[HTML]{EFEFEF} 
    \cellcolor[HTML]{EFEFEF}\textbf{Method Name} & \textbf{Parameters}    & \textbf{Returned Type} & \textbf{Visibility} \\ \hline
    main                                         & none                   & void                   & Public              \\ \hline
    \end{tabular}
\end{table}

\textbf{Description :} This is the entry point of the program.

\begin{table}[H]
    \begin{tabular}{|l|l|l|l|}
    \hline
    \rowcolor[HTML]{EFEFEF} 
    \cellcolor[HTML]{EFEFEF}\textbf{Method Name} & \textbf{Parameters}    & \textbf{Returned Type} & \textbf{Visibility} \\ \hline
    NewGame                                      & none                   & void                   & Public              \\ \hline
    \end{tabular}
\end{table}

\textbf{Description :} This method creates a new game with its own GameController, Match, Board etc\dots

\begin{table}[H]
    \begin{tabular}{|l|l|l|l|}
    \hline
    \rowcolor[HTML]{EFEFEF} 
    \cellcolor[HTML]{EFEFEF}\textbf{Method Name} & \textbf{Parameters}    & \textbf{Returned Type} & \textbf{Visibility} \\ \hline
    StartGame                                    & Player[2] players      & void                   & Public              \\ \hline
    \end{tabular}
\end{table}

\textbf{Description :} This method starts the game with the players given.

\begin{table}[H]
    \begin{tabular}{|l|l|l|l|}
    \hline
    \rowcolor[HTML]{EFEFEF} 
    \cellcolor[HTML]{EFEFEF}\textbf{Method Name} & \textbf{Parameters}    & \textbf{Returned Type} & \textbf{Visibility} \\ \hline
    ManagePlayers                                & none                   & void                   & Public              \\ \hline
    \end{tabular}
\end{table}

\textbf{Description :} This method makes the link between FormMenu and FormLeaderboard.

\begin{table}[H]
    \begin{tabular}{|l|l|l|l|}
    \hline
    \rowcolor[HTML]{EFEFEF} 
    \cellcolor[HTML]{EFEFEF}\textbf{Method Name} & \textbf{Parameters}    & \textbf{Returned Type} & \textbf{Visibility} \\ \hline
    Exit                                         & none                   & void                   & Public              \\ \hline
    \end{tabular}
\end{table}

\textbf{Description :} This is the exit point method.

\newpage

%GameController Class%
%%%%%%%%%%%%%%%%%%%%%%

\subsection{Class GameController}

\begin{table}[H]
    \begin{tabular}{|l|}
    \hline
    \rowcolor[HTML]{C0C0C0} 
    \textbf{GameController}              \\ \hline
    \rowcolor[HTML]{EFEFEF} 
    -\_main : Chess                      \\ \hline
    \rowcolor[HTML]{EFEFEF} 
    -\_selected : int                    \\ \hline
    \rowcolor[HTML]{EFEFEF} 
    -\_match : Match                     \\ \hline
    \rowcolor[HTML]{EFEFEF} 
    -\_playerA : Player                  \\ \hline
    \rowcolor[HTML]{EFEFEF} 
    -\_playerB : Player                  \\ \hline
    \rowcolor[HTML]{EFEFEF} 
    -\_tieCounter : int                  \\ \hline
    \rowcolor[HTML]{EFEFEF} 
    -\_view : FormMatch                  \\ \hline
    -Rules(int origin, int target) : void\\ \hline
    -Check() : bool                      \\ \hline
    -Checkmate() : bool                  \\ \hline
    -Castle() : bool                     \\ \hline
    -Stalemate() : bool                  \\ \hline
    -FiftyTurns() : bool                 \\ \hline
    -SameBoard() : bool                  \\ \hline
    -Turn(int origin, int target) : void \\ \hline
    +Selection(int cell) : void          \\ \hline
    +Resign() : void                     \\ \hline
    \end{tabular}
\end{table}

\subsubsection{Fields}

\begin{table}[H]
    \begin{tabular}{llllll}
    \hline
    \multicolumn{1}{|l|}{\cellcolor[HTML]{EFEFEF}\textbf{Field Name}} & \multicolumn{1}{l|}{\cellcolor[HTML]{EFEFEF}\textbf{Type}} & \multicolumn{1}{l|}{\cellcolor[HTML]{EFEFEF}\textbf{Visibility}} \\ \hline
    \multicolumn{1}{|l|}{\_main}                                      & \multicolumn{1}{l|}{Chess}                                 & \multicolumn{1}{l|}{Private}                                     \\ \hline
    \end{tabular}
\end{table}

\textbf{Description :} This field represent the chess which the game controller will talk to.

\begin{table}[H]
    \begin{tabular}{llllll}
    \hline
    \multicolumn{1}{|l|}{\cellcolor[HTML]{EFEFEF}\textbf{Field Name}} & \multicolumn{1}{l|}{\cellcolor[HTML]{EFEFEF}\textbf{Type}} & \multicolumn{1}{l|}{\cellcolor[HTML]{EFEFEF}\textbf{Visibility}} \\ \hline
    \multicolumn{1}{|l|}{\_selected}                                      & \multicolumn{1}{l|}{int}                                 & \multicolumn{1}{l|}{Private}                                     \\ \hline
    \end{tabular}
\end{table}

\textbf{Description :} This field represent the selected cell.

\begin{table}[H]
    \begin{tabular}{llllll}
    \hline
    \multicolumn{1}{|l|}{\cellcolor[HTML]{EFEFEF}\textbf{Field Name}} & \multicolumn{1}{l|}{\cellcolor[HTML]{EFEFEF}\textbf{Type}} & \multicolumn{1}{l|}{\cellcolor[HTML]{EFEFEF}\textbf{Visibility}} \\ \hline
    \multicolumn{1}{|l|}{\_match}                                      & \multicolumn{1}{l|}{Match}                                 & \multicolumn{1}{l|}{Private}                                     \\ \hline
    \end{tabular}
\end{table}

\textbf{Description :} This field represent the match of a game.

\begin{table}[H]
    \begin{tabular}{llllll}
    \hline
    \multicolumn{1}{|l|}{\cellcolor[HTML]{EFEFEF}\textbf{Field Name}} & \multicolumn{1}{l|}{\cellcolor[HTML]{EFEFEF}\textbf{Type}} & \multicolumn{1}{l|}{\cellcolor[HTML]{EFEFEF}\textbf{Visibility}} \\ \hline
    \multicolumn{1}{|l|}{\_playerA}                                      & \multicolumn{1}{l|}{Player}                                 & \multicolumn{1}{l|}{Private}                                     \\ \hline
    \end{tabular}
\end{table}

\textbf{Description :} This field represent the first player in the game.

\begin{table}[H]
    \begin{tabular}{llllll}
    \hline
    \multicolumn{1}{|l|}{\cellcolor[HTML]{EFEFEF}\textbf{Field Name}} & \multicolumn{1}{l|}{\cellcolor[HTML]{EFEFEF}\textbf{Type}} & \multicolumn{1}{l|}{\cellcolor[HTML]{EFEFEF}\textbf{Visibility}} \\ \hline
    \multicolumn{1}{|l|}{\_playerB}                                      & \multicolumn{1}{l|}{Player}                                 & \multicolumn{1}{l|}{Private}                                     \\ \hline
    \end{tabular}
\end{table}

\textbf{Description :} This field represent the second player in the game.

\begin{table}[H]
    \begin{tabular}{llllll}
    \hline
    \multicolumn{1}{|l|}{\cellcolor[HTML]{EFEFEF}\textbf{Field Name}} & \multicolumn{1}{l|}{\cellcolor[HTML]{EFEFEF}\textbf{Type}} & \multicolumn{1}{l|}{\cellcolor[HTML]{EFEFEF}\textbf{Visibility}} \\ \hline
    \multicolumn{1}{|l|}{\_tieCounter}                                      & \multicolumn{1}{l|}{int}                                 & \multicolumn{1}{l|}{Private}                                     \\ \hline
    \end{tabular}
\end{table}

\textbf{Description :} This field counts the number of ties in the game.

\begin{table}[H]
    \begin{tabular}{llllll}
    \hline
    \multicolumn{1}{|l|}{\cellcolor[HTML]{EFEFEF}\textbf{Field Name}} & \multicolumn{1}{l|}{\cellcolor[HTML]{EFEFEF}\textbf{Type}} & \multicolumn{1}{l|}{\cellcolor[HTML]{EFEFEF}\textbf{Visibility}} \\ \hline
    \multicolumn{1}{|l|}{\_view}                                      & \multicolumn{1}{l|}{FormMatch}                                 & \multicolumn{1}{l|}{Private}                                     \\ \hline
    \end{tabular}
\end{table}

\textbf{Description :} This field represent the view to paint in a match.

\subsubsection{Methods}

\begin{table}[H]
    \begin{tabular}{|l|l|l|l|}
    \hline
    \rowcolor[HTML]{EFEFEF} 
    \cellcolor[HTML]{EFEFEF}\textbf{Method Name} & \textbf{Parameters}    & \textbf{Returned Type} & \textbf{Visibility} \\ \hline
    Rules                                        & int origin, int target & void                   & Private             \\ \hline
    \end{tabular}
\end{table}

\textbf{Parameters :} Origin and target represent the index of the origin cell and the targeted cell for the path.
\textbf{Description :} This method calls Check(), Checkmate(), Castle() and Stalemate() to check the state of the game if the move is valid.


\begin{table}[H]
    \begin{tabular}{|l|l|l|l|}
    \hline
    \rowcolor[HTML]{EFEFEF} 
    \cellcolor[HTML]{EFEFEF}\textbf{Method Name} & \textbf{Parameters}    & \textbf{Returned Type} & \textbf{Visibility} \\ \hline
    Check                                        & none                   & bool                   & Public              \\ \hline
    \end{tabular}
\end{table}

\textbf{Description :} This method returns true if the match has a check.

\begin{table}[H]
    \begin{tabular}{|l|l|l|l|}
    \hline
    \rowcolor[HTML]{EFEFEF} 
    \cellcolor[HTML]{EFEFEF}\textbf{Method Name} & \textbf{Parameters}    & \textbf{Returned Type} & \textbf{Visibility} \\ \hline
    Checkmate                                    & None                   & bool                   & Private              \\ \hline
    \end{tabular}
\end{table}

\textbf{Description :} This method returns true if the match has a checkmate.

\begin{table}[H]
    \begin{tabular}{|l|l|l|l|}
    \hline
    \rowcolor[HTML]{EFEFEF} 
    \cellcolor[HTML]{EFEFEF}\textbf{Method Name} & \textbf{Parameters}    & \textbf{Returned Type} & \textbf{Visibility} \\ \hline
    Castle                                       & None                   & bool                   & Private              \\ \hline
    \end{tabular}
\end{table}

\textbf{Description :} This method returns true if the match has a Castle.

\begin{table}[H]
    \begin{tabular}{|l|l|l|l|}
    \hline
    \rowcolor[HTML]{EFEFEF} 
    \cellcolor[HTML]{EFEFEF}\textbf{Method Name} & \textbf{Parameters}    & \textbf{Returned Type} & \textbf{Visibility} \\ \hline
    Stalemate                                    & None                   & bool                   & Private              \\ \hline
    \end{tabular}
\end{table}

\textbf{Description :} This method returns true if the match has a stalemate.

\begin{table}[H]
    \begin{tabular}{|l|l|l|l|}
    \hline
    \rowcolor[HTML]{EFEFEF} 
    \cellcolor[HTML]{EFEFEF}\textbf{Method Name} & \textbf{Parameters}    & \textbf{Returned Type} & \textbf{Visibility} \\ \hline
    FiftyTurns                                   & None                   & bool                   & Private              \\ \hline
    \end{tabular}
\end{table}

\textbf{Description :} This method returns true if the match has 50 turns to count.

\begin{table}[H]
    \begin{tabular}{|l|l|l|l|}
    \hline
    \rowcolor[HTML]{EFEFEF} 
    \cellcolor[HTML]{EFEFEF}\textbf{Method Name} & \textbf{Parameters}    & \textbf{Returned Type} & \textbf{Visibility} \\ \hline
    SameBoard                                    & None                   & bool                   & Private              \\ \hline
    \end{tabular}
\end{table}

\textbf{Description :} This method returns true if history list of boards has 3 same boards.

\begin{table}[H]
    \begin{tabular}{|l|l|l|l|}
    \hline
    \rowcolor[HTML]{EFEFEF} 
    \cellcolor[HTML]{EFEFEF}\textbf{Method Name} & \textbf{Parameters}    & \textbf{Returned Type} & \textbf{Visibility} \\ \hline
    Turn                                         & int origin, int target & void                   & Private             \\ \hline
    \end{tabular}
\end{table}

\textbf{Parameters :} Origin and target represent the index of the origin cell and the targeted cell for the path.
\textbf{Description :} Makes the turn, and send it to the view.

\begin{table}[H]
    \begin{tabular}{|l|l|l|l|}
    \hline
    \rowcolor[HTML]{EFEFEF} 
    \cellcolor[HTML]{EFEFEF}\textbf{Method Name} & \textbf{Parameters}    & \textbf{Returned Type} & \textbf{Visibility} \\ \hline
    Selection                                    & int cell               & void                   & Public             \\ \hline
    \end{tabular}
\end{table}

\textbf{Parameters :} Parameeter cell is the index of the cell to be selected.
\textbf{Description :} Selects the cell.

\begin{table}[H]
    \begin{tabular}{|l|l|l|l|}
    \hline
    \rowcolor[HTML]{EFEFEF} 
    \cellcolor[HTML]{EFEFEF}\textbf{Method Name} & \textbf{Parameters}    & \textbf{Returned Type} & \textbf{Visibility} \\ \hline
    Resign                                       & none                   & void                   & Public             \\ \hline
    \end{tabular}
\end{table}

\textbf{Description :} Make the end and exit the game.

\newpage

%PlayerController Class%
%%%%%%%%%%%%%%%%%%%%%%%%

\subsection{Class PlayerController}

\begin{table}[H]
    \begin{tabular}{|l|}
    \hline
    \rowcolor[HTML]{C0C0C0} 
    \textbf{PlayerController}                       \\ \hline
    \rowcolor[HTML]{EFEFEF} 
    -\_main : Chess                               \\ \hline
    \rowcolor[HTML]{EFEFEF} 
    -\_list : List\textless{}{}Player\textgreater{}{} \\ \hline
    \rowcolor[HTML]{FFFFFF} 
    +Add() : void                                 \\ \hline
    \rowcolor[HTML]{FFFFFF} 
    +Remove() : void                              \\ \hline
    \end{tabular}
\end{table}

\subsubsection{Fields}

\begin{table}[H]
    \begin{tabular}{llllll}
    \hline
    \multicolumn{1}{|l|}{\cellcolor[HTML]{EFEFEF}\textbf{Field Name}} & \multicolumn{1}{l|}{\cellcolor[HTML]{EFEFEF}\textbf{Type}} & \multicolumn{1}{l|}{\cellcolor[HTML]{EFEFEF}\textbf{Visibility}} \\ \hline
    \multicolumn{1}{|l|}{\_main}                                      & \multicolumn{1}{l|}{Chess}                                 & \multicolumn{1}{l|}{Private}                                     \\ \hline
    \end{tabular}
\end{table}

\textbf{Description :} This field represent the chess which the game controller will talk to.

\begin{table}[H]
    \begin{tabular}{llllll}
    \hline
    \multicolumn{1}{|l|}{\cellcolor[HTML]{EFEFEF}\textbf{Field Name}} & \multicolumn{1}{l|}{\cellcolor[HTML]{EFEFEF}\textbf{Type}} & \multicolumn{1}{l|}{\cellcolor[HTML]{EFEFEF}\textbf{Visibility}} \\ \hline
    \multicolumn{1}{|l|}{\_list}                                      & \multicolumn{1}{l|}{List\textless{}Player\textgreater{}}                                 & \multicolumn{1}{l|}{Private}                                     \\ \hline
    \end{tabular}
\end{table}

\textbf{Description :} This field represent the list of all players.

\subsubsection{Methods}

\begin{table}[H]
    \begin{tabular}{|l|l|l|l|}
    \hline
    \rowcolor[HTML]{EFEFEF} 
    \cellcolor[HTML]{EFEFEF}\textbf{Method Name} & \textbf{Parameters}    & \textbf{Returned Type} & \textbf{Visibility} \\ \hline
    Add                                          & none                   & void                   & Public             \\ \hline
    \end{tabular}
\end{table}

\textbf{Description :} This method adds the player to the list.

\begin{table}[H]
    \begin{tabular}{|l|l|l|l|}
    \hline
    \rowcolor[HTML]{EFEFEF} 
    \cellcolor[HTML]{EFEFEF}\textbf{Method Name} & \textbf{Parameters}    & \textbf{Returned Type} & \textbf{Visibility} \\ \hline
    Remove                                       & none                   & void                   & Public             \\ \hline
    \end{tabular}
\end{table}

\textbf{Description :} This method removes the player to the list.

\newpage

%NEW SECTION VIEWS%
%%%%%%%%%%%%%%%%%%%

\section{Views}

%FormSelection Class%
%%%%%%%%%%%%%%%%%%%%%

\subsection{Class FormSelection}

\begin{table}[H]
    \begin{tabular}{|l|}
    \hline
    \rowcolor[HTML]{C0C0C0} 
    \textbf{FormSelection}    \\ \hline
    \rowcolor[HTML]{EFEFEF} 
    -\_controller : Chess      \\ \hline
    \rowcolor[HTML]{FFFFFF} 
    +OpenLeaderboard() : void \\ \hline
    \rowcolor[HTML]{FFFFFF} 
    +Start() : Player{[}2{]}  \\ \hline
    \rowcolor[HTML]{FFFFFF} 
    +Cancel() : void          \\ \hline
    \end{tabular}
\end{table}

\subsubsection{Fields}

\begin{table}[H]
    \begin{tabular}{llllll}
    \hline
    \multicolumn{1}{|l|}{\cellcolor[HTML]{EFEFEF}\textbf{Field Name}} & \multicolumn{1}{l|}{\cellcolor[HTML]{EFEFEF}\textbf{Type}} & \multicolumn{1}{l|}{\cellcolor[HTML]{EFEFEF}\textbf{Visibility}} \\ \hline
    \multicolumn{1}{|l|}{\_controller}                                & \multicolumn{1}{l|}{Chess}                                & \multicolumn{1}{l|}{Private}                                     \\ \hline
    \end{tabular}
\end{table}

\textbf{Description :} This field is the controller of the selection menu, it helps to access the list of players.

\subsubsection{Methods}

\begin{table}[H]
    \begin{tabular}{|l|l|l|l|}
    \hline
    \rowcolor[HTML]{EFEFEF} 
    \cellcolor[HTML]{EFEFEF}\textbf{Method Name} & \textbf{Parameters}    & \textbf{Returned Type} & \textbf{Visibility} \\ \hline
    OpenLeaderboard                              & none                   & void                   & Public             \\ \hline
    \end{tabular}
\end{table}

\textbf{Description :} This method opens a form listing all the best players.

\begin{table}[H]
    \begin{tabular}{|l|l|l|l|}
    \hline
    \rowcolor[HTML]{EFEFEF} 
    \cellcolor[HTML]{EFEFEF}\textbf{Method Name} & \textbf{Parameters}    & \textbf{Returned Type} & \textbf{Visibility} \\ \hline
    Start                                        & none                   & void                   & Public             \\ \hline
    \end{tabular}
\end{table}

\textbf{Description :} This method returns a table of 2 players and starts the game.

\begin{table}[H]
    \begin{tabular}{|l|l|l|l|}
    \hline
    \rowcolor[HTML]{EFEFEF} 
    \cellcolor[HTML]{EFEFEF}\textbf{Method Name} & \textbf{Parameters}    & \textbf{Returned Type} & \textbf{Visibility} \\ \hline
    Cancel                                        & none                   & void                   & Public             \\ \hline
    \end{tabular}
\end{table}

\textbf{Description :} This method cancels the player selection.

\newpage

%FormMenu Class%
%%%%%%%%%%%%%%%%

\subsection{Class FormMenu}

\begin{table}[H]
    \begin{tabular}{|l|}
    \hline
    \rowcolor[HTML]{C0C0C0} 
    \textbf{FormMenu}                                   \\ \hline
    \rowcolor[HTML]{EFEFEF} 
    -\_main : Chess                                          \\ \hline
    \rowcolor[HTML]{FFFFFF} 
    +Start(object sender, System.EventArgs e) : void         \\ \hline
    \rowcolor[HTML]{FFFFFF} 
    +Exit(object sender, System.EventArgs e) : void          \\ \hline
    \rowcolor[HTML]{FFFFFF} 
    +ManagePlayers(object sender, System.EventArgs e) : void \\ \hline
    \end{tabular}
\end{table}

\subsubsection{Fields}

\begin{table}[H]
    \begin{tabular}{llllll}
    \hline
    \multicolumn{1}{|l|}{\cellcolor[HTML]{EFEFEF}\textbf{Field Name}} & \multicolumn{1}{l|}{\cellcolor[HTML]{EFEFEF}\textbf{Type}} & \multicolumn{1}{l|}{\cellcolor[HTML]{EFEFEF}\textbf{Visibility}} \\ \hline
    \multicolumn{1}{|l|}{\_main}                                      & \multicolumn{1}{l|}{Chess}                                & \multicolumn{1}{l|}{Private}                                     \\ \hline
    \end{tabular}
\end{table}

\textbf{Description :} This field makes the link between the menu and the chess game.

\subsubsection{Methods}

\begin{table}[H]
    \begin{tabular}{|l|l|l|l|}
    \hline
    \rowcolor[HTML]{EFEFEF} 
    \cellcolor[HTML]{EFEFEF}\textbf{Method Name} & \textbf{Parameters}                    & \textbf{Returned Type} & \textbf{Visibility} \\ \hline
    Start                                        & object sender, System.EventArgs e      & void                   & Public             \\ \hline
    \end{tabular}
\end{table}

\textbf{Description :} Parameter sender is the button and e is the event listener.

\textbf{Description :} This method is for the start button. Starts the game.

\begin{table}[H]
    \begin{tabular}{|l|l|l|l|}
    \hline
    \rowcolor[HTML]{EFEFEF} 
    \cellcolor[HTML]{EFEFEF}\textbf{Method Name} & \textbf{Parameters}                    & \textbf{Returned Type} & \textbf{Visibility} \\ \hline
    Exit                                        & object sender, System.EventArgs e      & void                   & Public             \\ \hline
    \end{tabular}
\end{table}

\textbf{Description :} Parameter sender is the button and e is the event listener.

\textbf{Description :} This method is for the exit button. Stops the game.

\begin{table}[H]
    \begin{tabular}{|l|l|l|l|}
    \hline
    \rowcolor[HTML]{EFEFEF} 
    \cellcolor[HTML]{EFEFEF}\textbf{Method Name} & \textbf{Parameters}                    & \textbf{Returned Type} & \textbf{Visibility} \\ \hline
    ManagePlayers                                        & object sender, System.EventArgs e      & void                   & Public             \\ \hline
    \end{tabular}
\end{table}

\textbf{Description :} Parameter sender is the button and e is the event listener.

\textbf{Description :} This method is to manage players, add or remove.

\newpage

%FormPromotion Class%
%%%%%%%%%%%%%%%%%%%%%

\subsection{Class FormPromotion}

\begin{table}[H]
    \begin{tabular}{|l|}
    \hline
    \rowcolor[HTML]{C0C0C0} 
    \textbf{FormPromotion}                            \\ \hline
    \rowcolor[HTML]{EFEFEF} 
    -\_controller : GameController                    \\ \hline
    \rowcolor[HTML]{FFFFFF} 
    +Submit(object sender, System.EventArgs e) : void \\ \hline
    \end{tabular}
\end{table}

\subsubsection{Fields}

\begin{table}[H]
    \begin{tabular}{llllll}
    \hline
    \multicolumn{1}{|l|}{\cellcolor[HTML]{EFEFEF}\textbf{Field Name}} & \multicolumn{1}{l|}{\cellcolor[HTML]{EFEFEF}\textbf{Type}} & \multicolumn{1}{l|}{\cellcolor[HTML]{EFEFEF}\textbf{Visibility}} \\ \hline
    \multicolumn{1}{|l|}{\_controller}                                & \multicolumn{1}{l|}{GameController}                        & \multicolumn{1}{l|}{Private}                                     \\ \hline
    \end{tabular}
\end{table}

\textbf{Description :} This field makes the link between the promotion form and the players.

\subsubsection{Methods}

\begin{table}[H]
    \begin{tabular}{|l|l|l|l|}
    \hline
    \rowcolor[HTML]{EFEFEF} 
    \cellcolor[HTML]{EFEFEF}\textbf{Method Name} & \textbf{Parameters}                    & \textbf{Returned Type} & \textbf{Visibility} \\ \hline
    ManagePlayers                                        & object sender, System.EventArgs e      & void                   & Public             \\ \hline
    \end{tabular}
\end{table}

\textbf{Description :} Parameter sender is the button and e is the event listener.

\textbf{Description :} This method is to submit changes to players.

\newpage

%FormMatch Class%
%%%%%%%%%%%%%%%%%

\subsection{Class FormMatch}

\begin{table}[H]
    \begin{tabular}{|l|}
    \hline
    \rowcolor[HTML]{C0C0C0} 
    \textbf{FormMatch}                                   \\ \hline
    \rowcolor[HTML]{EFEFEF} 
    -\_controller : GameController                       \\ \hline
    \rowcolor[HTML]{FFFFFF} 
    +GridClick(object sender, System.EventArgs e) : void \\ \hline
    +DrawBoard(string board) : void                      \\ \hline
    +DrawSelection(int cell) : void                      \\ \hline
    +ShowMessage(string message) : void                  \\ \hline
    +VictoryMessage() : void                             \\ \hline
    \end{tabular}
\end{table}

\subsubsection{Fields}

\begin{table}[H]
    \begin{tabular}{llllll}
    \hline
    \multicolumn{1}{|l|}{\cellcolor[HTML]{EFEFEF}\textbf{Field Name}} & \multicolumn{1}{l|}{\cellcolor[HTML]{EFEFEF}\textbf{Type}} & \multicolumn{1}{l|}{\cellcolor[HTML]{EFEFEF}\textbf{Visibility}} \\ \hline
    \multicolumn{1}{|l|}{\_controller}                                & \multicolumn{1}{l|}{GameController}                        & \multicolumn{1}{l|}{Private}                                     \\ \hline
    \end{tabular}
\end{table}

\textbf{Description :} This field makes the link between the match form and the chess game.

\subsubsection{Methods}

\begin{table}[H]
    \begin{tabular}{|l|l|l|l|}
    \hline
    \rowcolor[HTML]{EFEFEF} 
    \cellcolor[HTML]{EFEFEF}\textbf{Method Name} & \textbf{Parameters}                    & \textbf{Returned Type} & \textbf{Visibility} \\ \hline
    GridClick                                    & object sender, System.EventArgs e      & void                   & Public             \\ \hline
    \end{tabular}
\end{table}

\textbf{Description :} Parameter sender is the button and e is the event listener.

\textbf{Description :} This method sends the coordinates of the cell clicked on the match board to the game controller.

\begin{table}[H]
    \begin{tabular}{|l|l|l|l|}
    \hline
    \rowcolor[HTML]{EFEFEF} 
    \cellcolor[HTML]{EFEFEF}\textbf{Method Name} & \textbf{Parameters}                    & \textbf{Returned Type} & \textbf{Visibility} \\ \hline
    DrawBoard                                    & string board                           & void                   & Public             \\ \hline
    \end{tabular}
\end{table}

\textbf{Description :} Parameter board is a string that represents a board as each char represents a cell.

\textbf{Description :} This method paints the board with all pieces at the right places.

\begin{table}[H]
    \begin{tabular}{|l|l|l|l|}
    \hline
    \rowcolor[HTML]{EFEFEF} 
    \cellcolor[HTML]{EFEFEF}\textbf{Method Name} & \textbf{Parameters}                    & \textbf{Returned Type} & \textbf{Visibility} \\ \hline
    DrawSelection                                    & int cell                           & void                   & Public             \\ \hline
    \end{tabular}
\end{table}

\textbf{Description :} Parameter cell is a index of a cell that represents the selected cell.

\textbf{Description :} This method paints the selected cell as selected.

\begin{table}[H]
    \begin{tabular}{|l|l|l|l|}
    \hline
    \rowcolor[HTML]{EFEFEF} 
    \cellcolor[HTML]{EFEFEF}\textbf{Method Name} & \textbf{Parameters}                    & \textbf{Returned Type} & \textbf{Visibility} \\ \hline
    ShowMessage                                    & string message                       & void                   & Public             \\ \hline
    \end{tabular}
\end{table}

\textbf{Description :} Parameter message is a string that represents a specific message depending on the situation.

\textbf{Description :} This method paints the message.

\begin{table}[H]
    \begin{tabular}{|l|l|l|l|}
    \hline
    \rowcolor[HTML]{EFEFEF} 
    \cellcolor[HTML]{EFEFEF}\textbf{Method Name} & \textbf{Parameters}                    & \textbf{Returned Type} & \textbf{Visibility} \\ \hline
    VictoryMessage                               & none                                   & void                   & Public             \\ \hline
    \end{tabular}
\end{table}

\textbf{Description :} Parameter message is a string that represents a specific message depending on the situation.

\textbf{Description :} This method paints the victory message.

\newpage

%FormLeaderboard Class%
%%%%%%%%%%%%%%%%%%%%%%%

\subsection{Class FormLeaderboard}

\begin{table}[H]
    \begin{tabular}{|l|}
    \hline
    \rowcolor[HTML]{C0C0C0} 
    \textbf{FormLeaderboard}                                    \\ \hline
    \rowcolor[HTML]{EFEFEF} 
    -\_controller : PlayerController                      \\ \hline
    \rowcolor[HTML]{FFFFFF} 
    +ShowList(List\textless{}{}Player\textgreater{}{}) : void \\ \hline
    +Add() : void                                         \\ \hline
    +Remove() : void                                      \\ \hline
    +Back() : void                                        \\ \hline
    \end{tabular}
\end{table}

\subsubsection{Fields}

\begin{table}[H]
    \begin{tabular}{llllll}
    \hline
    \multicolumn{1}{|l|}{\cellcolor[HTML]{EFEFEF}\textbf{Field Name}} & \multicolumn{1}{l|}{\cellcolor[HTML]{EFEFEF}\textbf{Type}} & \multicolumn{1}{l|}{\cellcolor[HTML]{EFEFEF}\textbf{Visibility}} \\ \hline
    \multicolumn{1}{|l|}{\_controller}                                & \multicolumn{1}{l|}{PlayerController}                        & \multicolumn{1}{l|}{Private}                                     \\ \hline
    \end{tabular}
\end{table}

\textbf{Description :} This field is the controller for the leader board.

\subsubsection{Methods}

\begin{table}[H]
    \begin{tabular}{|l|l|l|l|}
    \hline
    \rowcolor[HTML]{EFEFEF} 
    \cellcolor[HTML]{EFEFEF}\textbf{Method Name} & \textbf{Parameters}                    & \textbf{Returned Type} & \textbf{Visibility} \\ \hline
    ShowList                                     & List\textless{}Player\textgreater{}    & void                   & Public             \\ \hline
    \end{tabular}
\end{table}

\textbf{Description :} Parameter List is a list of players.

\textbf{Description :} This method paints the list of all players.

\begin{table}[H]
    \begin{tabular}{|l|l|l|l|}
    \hline
    \rowcolor[HTML]{EFEFEF} 
    \cellcolor[HTML]{EFEFEF}\textbf{Method Name} & \textbf{Parameters}                    & \textbf{Returned Type} & \textbf{Visibility} \\ \hline
    Add                                          & none                                   & void                   & Public             \\ \hline
    \end{tabular}
\end{table}

\textbf{Description :} Adds a player to list.

\begin{table}[H]
    \begin{tabular}{|l|l|l|l|}
    \hline
    \rowcolor[HTML]{EFEFEF} 
    \cellcolor[HTML]{EFEFEF}\textbf{Method Name} & \textbf{Parameters}                    & \textbf{Returned Type} & \textbf{Visibility} \\ \hline
    Remove                                       & none                                   & void                   & Public             \\ \hline
    \end{tabular}
\end{table}

\textbf{Description :} Removes a player from list.

\begin{table}[H]
    \begin{tabular}{|l|l|l|l|}
    \hline
    \rowcolor[HTML]{EFEFEF} 
    \cellcolor[HTML]{EFEFEF}\textbf{Method Name} & \textbf{Parameters}                    & \textbf{Returned Type} & \textbf{Visibility} \\ \hline
    Back                                         & none                                   & void                   & Public             \\ \hline
    \end{tabular}
\end{table}

\textbf{Description :} Gets back to main menu.

\newpage

\end{document}