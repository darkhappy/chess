\documentclass[12pt]{article}
\usepackage[T1]{fontenc}
\usepackage{booktabs}
\usepackage{float}
\usepackage[table,xcdraw]{xcolor}
\usepackage{graphicx}

\graphicspath{ {./images/} }

\title{Class Dictionary}
\author{Jean-Philippe Miguel-Gagnon, Jérémy Gaouette, Raphaël Rail}

\date{Thursday, 31th of march 2022}

\begin{document}

\begin{titlepage}
\maketitle
\includegraphics[width=\textwidth]{CHESS}
\begin{center}Presented to : Charles Jacob\end{center}
\end{titlepage}

\tableofcontents

\newpage

\section{Introduction}

The goal of this document is to inform the programmer about
classes used to create a C\#\ OOP Chess game.
\\

We'll go through this with the MVC model approch to make it
clearer for the programmer where to implement his code. 

\newpage

\section{Models}
%%%%%%%%%%%%%
%Piece Class%

\subsection{Classe Piece}

This is an abstract class for a base piece for the game.
\begin{table}[H]
    \begin{tabular}{|l|}
    \hline
    \cellcolor[HTML]{C0C0C0}\textbf{(Abstract) Piece}            \\ \hline
    \cellcolor[HTML]{EFEFEF}-\_colour : Colour        \\ \hline
    +CanCollide() : bool                                \\ \hline
    +ValidMove(int x1, int y1, int x2, int y2) : bool \\ \hline
    +ToString()                                       \\ \hline
    \end{tabular}
\end{table}

\subsubsection{Fields}

\begin{table}[H]
    \begin{tabular}{llllll}
    \hline
    \multicolumn{1}{|l|}{\cellcolor[HTML]{EFEFEF}\textbf{Field Name}} & \multicolumn{1}{l|}{\cellcolor[HTML]{EFEFEF}\textbf{Type}} & \multicolumn{1}{l|}{\cellcolor[HTML]{EFEFEF}\textbf{Visibility}} \\ \hline
    \multicolumn{1}{|l|}{\_colour}                                    & \multicolumn{1}{l|}{Colour}                                & \multicolumn{1}{l|}{Private}                                     \\ \hline
    \end{tabular}
\end{table}

\textbf{Description :} This field represent the colour a piece, wich can only be
black or white as it is for a regular chess game.


\subsubsection{Methods}

\begin{table}[H]
    \begin{tabular}{|l|l|l|l|}
    \hline
    \rowcolor[HTML]{EFEFEF} 
    \cellcolor[HTML]{EFEFEF}\textbf{Method Name} & \textbf{Parameters}  & \textbf{Returned Type} & \textbf{Visibility} \\ \hline
    ValidMove                          & x1, y1, x2, y2 : int & bool                   & Public              \\ \hline
    \end{tabular}
\end{table}

\textbf{Parameters :} x1 ans y1 represent the coordinates of the position before a possible move
and x2 and y2 are the coordinates of the position after the move. These are all of Integer(16) type.
\\
\textbf{Description :} The method returns true if the piece can move to the 2nd position.

\begin{table}[H]
    \begin{tabular}{|l|l|l|l|}
    \hline
    \rowcolor[HTML]{EFEFEF} 
    \cellcolor[HTML]{EFEFEF}\textbf{Method Name} & \textbf{Parameters}  & \textbf{Returned Type} & \textbf{Visibility} \\ \hline
    CanCollide                                   & None                 & bool                   & Public              \\ \hline
    \end{tabular}
\end{table}

\textbf{Description :} Returns true if the piece can't go over other pieces, otherwise
it returns false.

\begin{table}[H]
    \begin{tabular}{|l|l|l|l|}
    \hline
    \rowcolor[HTML]{EFEFEF} 
    \cellcolor[HTML]{EFEFEF}\textbf{Method Name} & \textbf{Parameters}  & \textbf{Returned Type} & \textbf{Visibility} \\ \hline
    ToString                                   & None                 & String                   & Public              \\ \hline
    \end{tabular}
\end{table}

\textbf{Description :} Empty method te be overrided by child classes.

\subsubsection{Properties}

\begin{table}[H]
    \begin{tabular}{|l|l|l|l|}
    \hline
    \rowcolor[HTML]{EFEFEF} 
    \cellcolor[HTML]{EFEFEF}\textbf{Property Name} & \textbf{Parameters}  & \textbf{Returned Type} & \textbf{Visibility} \\ \hline
    Colour                                         & None                 & Colour                 & Public              \\ \hline
    \end{tabular}
\end{table}

\textbf{Description :} Returns the colour of a piece.
\newpage

%%%%%%%%%%%%%%%
%StartingPiece%

\subsection{Class StartingPiece : Piece}

Another abstract Class that inherits to the base Class Piece.
This Class is for pieces that have different behaviour after they
have mouved for the first time.

\begin{table}[H]
    \begin{tabular}{|l|}
    \hline
    \cellcolor[HTML]{C0C0C0}\textbf{(Abstract) StartingPiece}            \\ \hline
    \cellcolor[HTML]{EFEFEF}-\_hasMoved : bool        \\ \hline
    +ValidMove(int x1, int y1, int x2, int y2) : bool \\ \hline
    +ToString()                                       \\ \hline
    \end{tabular}
\end{table}

\subsubsection{Fields}

\begin{table}[H]
    \begin{tabular}{llllll}
    \hline
    \multicolumn{1}{|l|}{\cellcolor[HTML]{EFEFEF}\textbf{Field Name}} & \multicolumn{1}{l|}{\cellcolor[HTML]{EFEFEF}\textbf{Type}} & \multicolumn{1}{l|}{\cellcolor[HTML]{EFEFEF}\textbf{Visibility}} \\ \hline
    \multicolumn{1}{|l|}{\_hasMoved}                                  & \multicolumn{1}{l|}{bool}                                & \multicolumn{1}{l|}{Private}                                     \\ \hline
    \end{tabular}
\end{table}

\textbf{Description :} This field is used to tell if the piece has
already made its first move.

\subsubsection{Methods}
Same as Class Piece.
\subsubsection{Properties}

\begin{table}[H]
    \begin{tabular}{|l|l|l|l|}
    \hline
    \rowcolor[HTML]{EFEFEF} 
    \cellcolor[HTML]{EFEFEF}\textbf{Property Name} & \textbf{Parameters}  & \textbf{Returned Type} & \textbf{Visibility} \\ \hline
    HasMoved                                       & None                 & bool                   & Public              \\ \hline
    \end{tabular}
\end{table}

\textbf{Description :} Gets or Sets the property \_hasMoved of a piece to true or false.
\newpage

%%%%%%%%%%%%
%Pawn Class%

\subsection{Class Pawn : StartingPiece}

This Class inherits the Class StartingPiece. It represents the pawn 
piece of a regular chess game.
\begin{table}[H]
    \begin{tabular}{|l|}
    \hline
    \cellcolor[HTML]{C0C0C0}\textbf{Pawn} \\ \hline
    \cellcolor[HTML]{EFEFEF}                    \\ \hline
    +ValidMove(x1, y1, x2, y2) : bool           \\ \hline
    +ToString() : string                        \\ \hline
    \end{tabular}
\end{table}

\subsubsection{Fields}

\textbf{Description :} This Class doesn't provide a new field. It just
inherits the fields of its parent.

\subsubsection{Methods}

\begin{table}[H]
    \begin{tabular}{|l|l|l|l|}
    \hline
    \rowcolor[HTML]{EFEFEF} 
    \cellcolor[HTML]{EFEFEF}\textbf{Method Name} & \textbf{Parameters}  & \textbf{Returned Type} & \textbf{Visibility} \\ \hline
    ValidMove                          & x1, y1, x2, y2 : int & bool                   & Public              \\ \hline
    \end{tabular}
\end{table}

\textbf{Parameters :} x1 ans y1 represent the coordinates of the position before a possible move
and x2 and y2 are the coordinates of the position after the move. These are all of Integer(16) type.
\\
\textbf{Description :} This method checks
if the move provided is valid considering the basic moves of a pawn which
are 1 or 2 cells ahead, or 1 diagonal cell if an opponent piece is
present. In thoses cases it returns true, otherwise it returns false. 

\begin{table}[H]
    \begin{tabular}{|l|l|l|l|}
    \hline
    \rowcolor[HTML]{EFEFEF} 
    \cellcolor[HTML]{EFEFEF}\textbf{Method Name} & \textbf{Parameters}  & \textbf{Returned Type} & \textbf{Visibility} \\ \hline
    ToString                                   & None                 & String                   & Public              \\ \hline
    \end{tabular}
\end{table}

\textbf{Description :} This method overrides ToString virtual
method and returns minus "p" if the piece is white and upper "P"
if it is black.
\subsubsection{Properties}

Same as parent.
\newpage

%%%%%%%%%%%%
%Rook Class%

\subsection{Class Rook : StartingPiece}

This Class inherits the Class StartingPiece. It represents the rook 
piece of a regular chess game.
\begin{table}[H]
    \begin{tabular}{|l|}
    \hline
    \cellcolor[HTML]{C0C0C0}\textbf{Rook} \\ \hline
    \cellcolor[HTML]{EFEFEF}                    \\ \hline
    +ValidMove(x1, y1, x2, y2) : bool           \\ \hline
    +ToString() : string                        \\ \hline
    \end{tabular}
\end{table}

\subsubsection{Fields}

\textbf{Description :} This Class doesn't provide a new field. It just
inherits the fields of its parent.

\subsubsection{Methods}

\begin{table}[H]
    \begin{tabular}{|l|l|l|l|}
    \hline
    \rowcolor[HTML]{EFEFEF} 
    \cellcolor[HTML]{EFEFEF}\textbf{Method Name} & \textbf{Parameters}  & \textbf{Returned Type} & \textbf{Visibility} \\ \hline
    ValidMove                          & x1, y1, x2, y2 : int & bool                   & Public              \\ \hline
    \end{tabular}
\end{table}

\textbf{Parameters :} x1 ans y1 represent the coordinates of the position before a possible move
and x2 and y2 are the coordinates of the position after the move. These are all of Integer(16) type.
\\
\textbf{Description :} This method checks if the move provided is
valid considering the basic moves of a rook which are forward,
backward or sideways to any empty cell. In thoses cases it returns
true, otherwise it returns false. 

\begin{table}[H]
    \begin{tabular}{|l|l|l|l|}
    \hline
    \rowcolor[HTML]{EFEFEF} 
    \cellcolor[HTML]{EFEFEF}\textbf{Method Name} & \textbf{Parameters}  & \textbf{Returned Type} & \textbf{Visibility} \\ \hline
    ToString                                   & None                 & String                   & Public              \\ \hline
    \end{tabular}
\end{table}

\textbf{Description :} This method overrides ToString virtual
method and returns minus "r" if the piece is white and upper "R"
if it is black.

\subsubsection{Properties}

Same as parent.
\newpage

%%%%%%%%%%%%
%King Class%

\subsection{Class King : StartingPiece}

This Class inherits the Class StartingPiece. It represents the king 
piece of a regular chess game.
\begin{table}[H]
    \begin{tabular}{|l|}
    \hline
    \cellcolor[HTML]{C0C0C0}\textbf{King} \\ \hline
    \cellcolor[HTML]{EFEFEF}                    \\ \hline
    +ValidMove(x1, y1, x2, y2) : bool           \\ \hline
    +ToString() : string                        \\ \hline
    \end{tabular}
\end{table}

\subsubsection{Fields}

\textbf{Description :} This Class doesn't provide a new field. It just
inherits the fields of its parent.

\subsubsection{Methods}

\begin{table}[H]
    \begin{tabular}{|l|l|l|l|}
    \hline
    \rowcolor[HTML]{EFEFEF} 
    \cellcolor[HTML]{EFEFEF}\textbf{Method Name} & \textbf{Parameters}  & \textbf{Returned Type} & \textbf{Visibility} \\ \hline
    ValidMove                          & x1, y1, x2, y2 : int & bool                   & Public              \\ \hline
    \end{tabular}
\end{table}

\textbf{Parameters :} x1 ans y1 represent the coordinates of the position before a possible move
and x2 and y2 are the coordinates of the position after the move. These are all of Integer(16) type.
\\
\textbf{Description :} This method checks if the move provided is
valid considering the basic moves of a king which are one square
horizontally, vertically, or diagonally unless the square is already
occupied by a friendly piece or the move would place the king in
check. In thoses cases it returns true, otherwise it returns false.

\begin{table}[H]
    \begin{tabular}{|l|l|l|l|}
    \hline
    \rowcolor[HTML]{EFEFEF} 
    \cellcolor[HTML]{EFEFEF}\textbf{Method Name} & \textbf{Parameters}  & \textbf{Returned Type} & \textbf{Visibility} \\ \hline
    ToString                                   & None                 & String                   & Public              \\ \hline
    \end{tabular}
\end{table}

\textbf{Description :} This method overrides ToString virtual
method and returns minus "k" if the piece is white and upper "K"
if it is black.

\subsubsection{Properties}

Same as parent.
\newpage

%%%%%%%%%%%%%%
%Knight Class%

\subsection{Class Knight : Piece}

This Class inherits the Class Piece. It represents the knight 
piece of a regular chess game.
\begin{table}[H]
    \begin{tabular}{|l|}
    \hline
    \cellcolor[HTML]{C0C0C0}\textbf{Knight} \\ \hline
    \cellcolor[HTML]{EFEFEF}                    \\ \hline
    +ValidMove(x1, y1, x2, y2) : bool           \\ \hline
    +ToString() : string                        \\ \hline
    \end{tabular}
\end{table}

\subsubsection{Fields}

\textbf{Description :} This Class doesn't provide a new field. It just
inherits the fields of its parent.

\subsubsection{Methods}

\begin{table}[H]
    \begin{tabular}{|l|l|l|l|}
    \hline
    \rowcolor[HTML]{EFEFEF} 
    \cellcolor[HTML]{EFEFEF}\textbf{Method Name} & \textbf{Parameters}  & \textbf{Returned Type} & \textbf{Visibility} \\ \hline
    ValidMove                          & x1, y1, x2, y2 : int & bool                   & Public              \\ \hline
    \end{tabular}
\end{table}

\textbf{Parameters :} x1 ans y1 represent the coordinates of the position before a possible move
and x2 and y2 are the coordinates of the position after the move. These are all of Integer(16) type.
\\
\textbf{Description :} This method checks if the move provided is valid considering the basic moves of a knight which
are “L-shape”—that is, they can move two squares in any direction
vertically followed by one square horizontally, or two squares in any
direction horizontally followed by one square vertically. In thoses 
cases it returns true, otherwise it returns false. 

\begin{table}[H]
    \begin{tabular}{|l|l|l|l|}
    \hline
    \rowcolor[HTML]{EFEFEF} 
    \cellcolor[HTML]{EFEFEF}\textbf{Method Name} & \textbf{Parameters}  & \textbf{Returned Type} & \textbf{Visibility} \\ \hline
    ToString                                   & None                 & String                   & Public              \\ \hline
    \end{tabular}
\end{table}

\textbf{Description :} This method overrides ToString virtual
method and returns minus "n" if the piece is white and upper "N"
if it is black.

\subsubsection{Properties}

Same as parent.

\newpage

%Bishop Class%
%%%%%%%%%%%%%%

\subsection{Class Bishop : Piece}

This Class inherits the Class Piece. It represents the bishop 
piece of a regular chess game.
\begin{table}[H]
    \begin{tabular}{|l|}
    \hline
    \cellcolor[HTML]{C0C0C0}\textbf{Bishop} \\ \hline
    \cellcolor[HTML]{EFEFEF}                    \\ \hline
    +ValidMove(x1, y1, x2, y2) : bool           \\ \hline
    +ToString() : string                        \\ \hline
    \end{tabular}
\end{table}

\subsubsection{Fields}

\textbf{Description :} This Class doesn't provide a new field. It just
inherits the fields of its parent.

\subsubsection{Methods}

\begin{table}[H]
    \begin{tabular}{|l|l|l|l|}
    \hline
    \rowcolor[HTML]{EFEFEF} 
    \cellcolor[HTML]{EFEFEF}\textbf{Method Name} & \textbf{Parameters}  & \textbf{Returned Type} & \textbf{Visibility} \\ \hline
    ValidMove                          & x1, y1, x2, y2 : int & bool                   & Public              \\ \hline
    \end{tabular}
\end{table}

\textbf{Parameters :} x1 ans y1 represent the coordinates of the position before a possible move
and x2 and y2 are the coordinates of the position after the move. These are all of Integer(16) type.
\\
\textbf{Description :} This method checks if the move provided is valid considering the basic moves of a bishop which
are any direction diagonally with no limit of cells unless there is another piece obstructing its path. In thoses 
cases it returns true, otherwise it returns false. 

\begin{table}[H]
    \begin{tabular}{|l|l|l|l|}
    \hline
    \rowcolor[HTML]{EFEFEF} 
    \cellcolor[HTML]{EFEFEF}\textbf{Method Name} & \textbf{Parameters}  & \textbf{Returned Type} & \textbf{Visibility} \\ \hline
    ToString                                   & None                 & String                   & Public              \\ \hline
    \end{tabular}
\end{table}

\textbf{Description :} This method overrides ToString virtual
method and returns minus "b" if the piece is white and upper "B"
if it is black.

\subsubsection{Properties}

Same as parent.

\newpage

%Queen%
%%%%%%%

\subsection{Class Queen : Piece}

This Class inherits the Class Piece. It represents the queen 
piece of a regular chess game.
\begin{table}[H]
    \begin{tabular}{|l|}
    \hline
    \cellcolor[HTML]{C0C0C0}\textbf{Queen} \\ \hline
    \cellcolor[HTML]{EFEFEF}                    \\ \hline
    +ValidMove(x1, y1, x2, y2) : bool           \\ \hline
    +ToString() : string                        \\ \hline
    \end{tabular}
\end{table}

\subsubsection{Fields}

\textbf{Description :} This Class doesn't provide a new field. It just
inherits the fields of its parent.

\subsubsection{Methods}

\begin{table}[H]
    \begin{tabular}{|l|l|l|l|}
    \hline
    \rowcolor[HTML]{EFEFEF} 
    \cellcolor[HTML]{EFEFEF}\textbf{Method Name} & \textbf{Parameters}  & \textbf{Returned Type} & \textbf{Visibility} \\ \hline
    ValidMove                          & x1, y1, x2, y2 : int & bool                   & Public              \\ \hline
    \end{tabular}
\end{table}

\textbf{Parameters :} x1 ans y1 represent the coordinates of the position before a possible move
and x2 and y2 are the coordinates of the position after the move. These are all of Integer(16) type.
\\
\textbf{Description :} This method checks if the move provided is valid considering the basic moves of a bishop which
are any direction diagonally with no limit of cells unless there is another piece obstructing its path. In thoses 
cases it returns true, otherwise it returns false. 

\begin{table}[H]
    \begin{tabular}{|l|l|l|l|}
    \hline
    \rowcolor[HTML]{EFEFEF} 
    \cellcolor[HTML]{EFEFEF}\textbf{Method Name} & \textbf{Parameters}  & \textbf{Returned Type} & \textbf{Visibility} \\ \hline
    ToString                                   & None                 & String                   & Public              \\ \hline
    \end{tabular}
\end{table}

\textbf{Description :} This method overrides ToString virtual
method and returns minus "q" if the piece is white and upper "Q"
if it is black.

\subsubsection{Properties}

Same as parent.

\newpage

%Match Class%
%%%%%%%%%%%%%

\subsection{Class Match}

This Class represents the model of a chess match wich compose the
GameController. It will keep all changes that the game controller will return.

\begin{table}[H]
    \begin{tabular}{|l|}
    \hline
    \rowcolor[HTML]{C0C0C0} 
    \textbf{Match}                                    \\ \hline
    \rowcolor[HTML]{EFEFEF} 
    -\_board : Board                                  \\ \hline
    \rowcolor[HTML]{EFEFEF} 
    -\_current : Colour                               \\ \hline
    \rowcolor[HTML]{EFEFEF} 
    -\_history : string{[}{]}                         \\ \hline
    \rowcolor[HTML]{EFEFEF} 
    -\_turnNumber : int                               \\ \hline
    +Export() : string                                \\ \hline
    +ValidTurn(int origin, int target) : bool         \\ \hline
    +MakeTurn(int origin, int target) : void          \\ \hline
    +ValidSelection(int cell, bool firstClick) : bool \\ \hline
    \end{tabular}
\end{table}

\subsubsection{Fields}

\begin{table}[H]
    \begin{tabular}{llllll}
    \hline
    \multicolumn{1}{|l|}{\cellcolor[HTML]{EFEFEF}\textbf{Field Name}} & \multicolumn{1}{l|}{\cellcolor[HTML]{EFEFEF}\textbf{Type}} & \multicolumn{1}{l|}{\cellcolor[HTML]{EFEFEF}\textbf{Visibility}} \\ \hline
    \multicolumn{1}{|l|}{\_board}                                     & \multicolumn{1}{l|}{Board}                                 & \multicolumn{1}{l|}{Private}                                     \\ \hline
    \end{tabular}
\end{table}

\textbf{Description :} This field represent the board of a match wich compose
the match. Its type is Board wich is the next class to discuss.

\begin{table}[H]
    \begin{tabular}{llllll}
    \hline
    \multicolumn{1}{|l|}{\cellcolor[HTML]{EFEFEF}\textbf{Field Name}} & \multicolumn{1}{l|}{\cellcolor[HTML]{EFEFEF}\textbf{Type}} & \multicolumn{1}{l|}{\cellcolor[HTML]{EFEFEF}\textbf{Visibility}} \\ \hline
    \multicolumn{1}{|l|}{\_current}                                     & \multicolumn{1}{l|}{Colour}                                 & \multicolumn{1}{l|}{Private}                                     \\ \hline
    \end{tabular}
\end{table}

\textbf{Description :} This field tells us wich piece colour is
currently playing, white or black.

\begin{table}[H]
    \begin{tabular}{llllll}
    \hline
    \multicolumn{1}{|l|}{\cellcolor[HTML]{EFEFEF}\textbf{Field Name}} & \multicolumn{1}{l|}{\cellcolor[HTML]{EFEFEF}\textbf{Type}} & \multicolumn{1}{l|}{\cellcolor[HTML]{EFEFEF}\textbf{Visibility}} \\ \hline
    \multicolumn{1}{|l|}{\_history}                                     & \multicolumn{1}{l|}{string[]}                            & \multicolumn{1}{l|}{Private}                                     \\ \hline
    \end{tabular}
\end{table}

\textbf{Description :} This field is a table that contains strings that
represent previous board states to keep track of what has been played. For exemple,
the first string would look like that :
\\"RNBKQBNRPPPPPPPP.................................pppppppprnkqbnr". 

\begin{table}[H]
    \begin{tabular}{llllll}
    \hline
    \multicolumn{1}{|l|}{\cellcolor[HTML]{EFEFEF}\textbf{Field Name}} & \multicolumn{1}{l|}{\cellcolor[HTML]{EFEFEF}\textbf{Type}} & \multicolumn{1}{l|}{\cellcolor[HTML]{EFEFEF}\textbf{Visibility}} \\ \hline
    \multicolumn{1}{|l|}{\_turnNumber}                                & \multicolumn{1}{l|}{int}                                   & \multicolumn{1}{l|}{Private}                                     \\ \hline
    \end{tabular}
\end{table}

\textbf{Description :} This field tracks the number of turns that have been played.

\subsubsection{Methods}

\begin{table}[H]
    \begin{tabular}{|l|l|l|l|}
    \hline
    \rowcolor[HTML]{EFEFEF} 
    \cellcolor[HTML]{EFEFEF}\textbf{Method Name} & \textbf{Parameters}  & \textbf{Returned Type} & \textbf{Visibility} \\ \hline
    Export                                       & none                 & string                 & Public              \\ \hline
    \end{tabular}
\end{table}

\textbf{Description :} The method returns the last string added to the field \_history.

\subsubsection{Properties}

None.
\newpage

\end{document}