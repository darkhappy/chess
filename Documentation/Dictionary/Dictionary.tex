\documentclass[12pt, letterpaper, twoside]{article}
\usepackage[T1]{fontenc}
\usepackage{booktabs}
\usepackage{float}
\usepackage[table,xcdraw]{xcolor}
\usepackage{graphicx}

\graphicspath{ {./images/} }

\title{Class Dictionary}
\author{Jean-Philippe Miguel-Gagnon, Jérémy Gaouette, Raphaël Rail}

\date{Thursday, 31th of march 2022}

\begin{document}

\begin{titlepage}
\maketitle
\includegraphics[width=\textwidth]{CHESS}
\begin{center}Presented to : Charles Jacob\end{center}
\end{titlepage}

\tableofcontents

\newpage

\section{Introduction}

The goal of this document is to inform the programmer about
classes used to create a C\#\ OOP Chess game.
\\

We'll go through this with the MVC model approch to make it
clearer for the programmer where to implement his code. 

\newpage

\section{Models}

\subsection{Classe `Piece'}

This is an abstract class for a base piece for the game.
\begin{table}[H]
    \begin{tabular}{|
    >{\columncolor[HTML]{F5F5F5}}l |lllll}
    \cline{1-1}
    \cellcolor[HTML]{A9A9A9}\textbf{Piece}            & \textbf{} & \textbf{} & \textbf{} &  &  \\ \cline{1-1}
    \cellcolor[HTML]{E8E8E8}-\_colour : Colour          &           &           &           &  &  \\ \cline{1-1}
    +Get et Set =\textgreater \_colour                 &           &           &           &  &  \\ \cline{1-1}
    +CanCollide : bool                                &           &           &           &  &  \\ \cline{1-1}
    +ValidMove(int x1, int y1, int x2, int y2) : bool &           &           &           &  &  \\ \cline{1-1}
    +ToString()                                       &           &           &           &  &  \\ \cline{1-1}
    \end{tabular}
\end{table}

\subsection{Fields}

\begin{table}[H]
    \begin{tabular}{llllll}
    \cline{1-3}
    \multicolumn{1}{|l|}{\cellcolor[HTML]{E8E8E8}\textbf{Field Name}} & \multicolumn{1}{l|}{\cellcolor[HTML]{E8E8E8}\textbf{Type}} & \multicolumn{1}{l|}{\cellcolor[HTML]{E8E8E8}\textbf{Visibility}} & \textbf{} &  &  \\ \cline{1-3}
    \multicolumn{1}{|l|}{\_colour}                                     & \multicolumn{1}{l|}{Colour}                                        & \multicolumn{1}{l|}{Private}                                 &           &  &  \\ \cline{1-3}
    \end{tabular}
\end{table}

\textbf{Description :} This field represent the colour a piece, wich can only be
black or white as it is for a regular chess game.


\subsection{Methods}

\begin{table}[H]
    \begin{tabular}{|l|l|l|l|}
    \hline
    \rowcolor[HTML]{EFEFEF} 
    \cellcolor[HTML]{EFEFEF}\textbf{Method Name} & \textbf{Parameters}  & \textbf{Returned Type} & \textbf{Visibility} \\ \hline
    ValidMove                          & x1, y1, x2, y2 : int & bool                   & Public              \\ \hline
    \end{tabular}
\end{table}

\textbf{Parameters :} x1 ans y1 represent the coordinates of the position before a possible move
and x2 and y2 are the coordinates of the position after the move. These are all of Integer(16) type.
\\
\textbf{Description :} The method calculates if a move is valid and returns true in this case.

\begin{table}[H]
    \begin{tabular}{|l|l|l|l|}
    \hline
    \rowcolor[HTML]{EFEFEF} 
    \cellcolor[HTML]{EFEFEF}\textbf{Method Name} & \textbf{Parameters}  & \textbf{Returned Type} & \textbf{Visibility} \\ \hline
    CanCollide                                   & None                 & bool                   & Public              \\ \hline
    \end{tabular}
\end{table}

\textbf{Parameters :} x1 ans y1 represent the coordinates of the position before a possible move
and x2 and y2 are the coordinates of the position after the 
\textbf{Description :} 

\newpage

Descrptions : 
\begin{table}[H]
    \begin{tabular}{|l|l|}
    \hline
    \rowcolor[HTML]{EFEFEF} 
    \textbf{Method Name} & \textbf{Description}                                                                                                    \\ \hline
    GetColor             & Getter for the color of a piece                                                                                         \\ \hline
    SetColor             & Setter for the color of a piece                                                                                         \\ \hline
    CanColide            & Returns true if the piece can collide                                                                                   \\ \hline
    ValidMove            & Returns true if the piece is asked to do a valid move                                                                   \\ \hline
    ToString             & \begin{tabular}[c]{@{}l@{}}Methode to be override by children to show all properties of\\ a specific piece\end{tabular} \\ \hline
    \end{tabular}
\end{table}

    
\end{document}