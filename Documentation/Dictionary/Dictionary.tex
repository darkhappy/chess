\documentclass[12pt]{article}
\usepackage[margin=1in]{geometry}
\usepackage{float}
\usepackage[table,xcdraw]{xcolor}
\usepackage{graphicx}

\graphicspath{ {./images/} }

\title{Class Dictionary}
\author{Jean-Philippe Miguel-Gagnon, Jérémy Gaouette, Raphaël Rail}

\date{Thursday, 31th of march 2022}

\begin{document}

\begin{titlepage}
\maketitle
\includegraphics[width=\textwidth]{CHESS}
\begin{center}Presented to : Charles Jacob\end{center}
\end{titlepage}

\tableofcontents

\newpage

\section{Introduction}

The goal of this document is to inform the programmer about
classes used to create a C\#\ OOP Chess game.
\\

We'll go through this with the MVC model approch to make it
clearer for the programmer where to implement his code. 

\newpage

\section{Models}
%%%%%%%%%%%%%
%Piece Class%

\subsection{Classe Piece}

This is an abstract class for a base piece for the game.
\begin{table}[H]
    \begin{tabular}{|l|}
    \hline
    \cellcolor[HTML]{C0C0C0}\textbf{(Abstract) Piece}  \\ \hline
    \cellcolor[HTML]{EFEFEF}-\_colour : Colour         \\ \hline
    +CanCollide() : bool                               \\ \hline
    +ValidMove(int x1, int y1, int x2, int y2) : bool  \\ \hline
    +ToString() : string                               \\ \hline
    +CanPromote() : bool                               \\ \hline
    +IsEssential() : bool                              \\ \hline
    \end{tabular}
\end{table}

\subsubsection{Fields}

\begin{table}[H]
    \begin{tabular}{llllll}
    \hline
    \multicolumn{1}{|l|}{\cellcolor[HTML]{EFEFEF}\textbf{Field Name}} & \multicolumn{1}{l|}{\cellcolor[HTML]{EFEFEF}\textbf{Type}} & \multicolumn{1}{l|}{\cellcolor[HTML]{EFEFEF}\textbf{Visibility}} \\ \hline
    \multicolumn{1}{|l|}{\_colour}                                    & \multicolumn{1}{l|}{Colour}                                & \multicolumn{1}{l|}{Private}                                     \\ \hline
    \end{tabular}
\end{table}

    \textbf{Description :} Represents the colour of the piece, which is either white or black.

\subsubsection{Methods}

\begin{table}[H]
    \begin{tabular}{|l|l|l|l|}
    \hline
    \rowcolor[HTML]{EFEFEF} 
    \cellcolor[HTML]{EFEFEF}\textbf{Method Name} & \textbf{Parameters}  & \textbf{Returned Type} & \textbf{Visibility} \\ \hline
    ValidMove                          & x1, y1, x2, y2 : int & bool                   & Public              \\ \hline
    \end{tabular}
\end{table}

    \textbf{Parameters :} x1 and y1 are the coordinates of the piece.
    x2 and y2 are the coordinates of the target location.
\\
    \textbf{Description :} This method is used to check if the piece can move from position (x1, y1) to position (x2, y2).

\begin{table}[H]
    \begin{tabular}{|l|l|l|l|}
    \hline
    \rowcolor[HTML]{EFEFEF} 
    \cellcolor[HTML]{EFEFEF}\textbf{Method Name} & \textbf{Parameters}  & \textbf{Returned Type} & \textbf{Visibility} \\ \hline
    CanCollide                                   & None                 & bool                   & Public              \\ \hline
    \end{tabular}
\end{table}

    \textbf{Description :} By default, this method returns true, meaning that the piece can collide with other pieces.

\begin{table}[H]
    \begin{tabular}{|l|l|l|l|}
    \hline
    \rowcolor[HTML]{EFEFEF} 
    \cellcolor[HTML]{EFEFEF}\textbf{Method Name} & \textbf{Parameters}  & \textbf{Returned Type} & \textbf{Visibility} \\ \hline
    ToString                                   & None                 & String                   & Public              \\ \hline
    \end{tabular}
\end{table}

    \textbf{Description :} This method returns the string representation of the piece.
    This method gets overridden by the subclasses.

\begin{table}[H]
    \begin{tabular}{|l|l|l|l|}
    \hline
    \rowcolor[HTML]{EFEFEF} 
    \cellcolor[HTML]{EFEFEF}\textbf{Method Name} & \textbf{Parameters}  & \textbf{Returned Type} & \textbf{Visibility} \\ \hline
    CanPromote                                   & None                 & bool                   & Public              \\ \hline
    \end{tabular}
\end{table}

    \textbf{Description :} By default, this method returns false, meaning that the piece cannot be promoted.

\begin{table}[H]
    \begin{tabular}{|l|l|l|l|}
    \hline
    \rowcolor[HTML]{EFEFEF} 
    \cellcolor[HTML]{EFEFEF}\textbf{Method Name} & \textbf{Parameters}  & \textbf{Returned Type} & \textbf{Visibility} \\ \hline
    IsEssential                                  & None                 & bool                   & Public              \\ \hline
    \end{tabular}
\end{table}

    \textbf{Description :} By default, this method returns false, meaning that the piece is not essential.

\subsubsection{Properties}

\begin{table}[H]
    \begin{tabular}{|l|l|l|l|}
    \hline
    \rowcolor[HTML]{EFEFEF} 
    \cellcolor[HTML]{EFEFEF}\textbf{Property Name} & \textbf{Parameters}  & \textbf{Returned Type} & \textbf{Visibility} \\ \hline
    Colour                                         & None                 & Colour                 & Public              \\ \hline
    \end{tabular}
\end{table}

    \textbf{Description :} Gets the colour of the piece.
\newpage

%%%%%%%%%%%%%%%
%StartingPiece%

\subsection{Class StartingPiece : Piece}

    Extended from the Piece class, the StartingPiece class represents a piece that keeps track of whether it has been moved or not.

\begin{table}[H]
    \begin{tabular}{|l|}
    \hline
    \cellcolor[HTML]{C0C0C0}\textbf{(Abstract) StartingPiece}            \\ \hline
    \cellcolor[HTML]{EFEFEF}-\_hasMoved : bool        \\ \hline
    \end{tabular}
\end{table}

\subsubsection{Fields}

\begin{table}[H]
    \begin{tabular}{llllll}
    \hline
    \multicolumn{1}{|l|}{\cellcolor[HTML]{EFEFEF}\textbf{Field Name}} & \multicolumn{1}{l|}{\cellcolor[HTML]{EFEFEF}\textbf{Type}} & \multicolumn{1}{l|}{\cellcolor[HTML]{EFEFEF}\textbf{Visibility}} \\ \hline
    \multicolumn{1}{|l|}{\_hasMoved}                                  & \multicolumn{1}{l|}{bool}                                & \multicolumn{1}{l|}{Private}                                     \\ \hline
    \end{tabular}
\end{table}

\textbf{Description :} Represents whether the piece has been moved or not.

\subsubsection{Methods}
    Inherited from the Piece class.
\subsubsection{Properties}

\begin{table}[H]
    \begin{tabular}{|l|l|l|l|}
    \hline
    \rowcolor[HTML]{EFEFEF} 
    \cellcolor[HTML]{EFEFEF}\textbf{Property Name} & \textbf{Parameters}  & \textbf{Returned Type} & \textbf{Visibility} \\ \hline
    HasMoved                                       & None                 & bool                   & Public              \\ \hline
    \end{tabular}
\end{table}

    \textbf{Description :} Gets or Sets whether the piece has been moved or not.
\newpage

%%%%%%%%%%%%
%Pawn Class%

\subsection{Class Pawn : StartingPiece}

    Extended from the StartingPiece class, the Pawn class represents a pawn.
\begin{table}[H]
    \begin{tabular}{|l|}
    \hline
    \cellcolor[HTML]{C0C0C0}\textbf{Pawn} \\ \hline
    \cellcolor[HTML]{EFEFEF}                    \\ \hline
    +ValidMove(x1, y1, x2, y2) : bool           \\ \hline
    +CanPromote() : bool                        \\ \hline
    +ToString() : string                        \\ \hline
    \end{tabular}
\end{table}

\subsubsection{Fields}

    Inherited from the StartingPiece class.

\subsubsection{Methods}

\begin{table}[H]
    \begin{tabular}{|l|l|l|l|}
    \hline
    \rowcolor[HTML]{EFEFEF} 
    \cellcolor[HTML]{EFEFEF}\textbf{Method Name} & \textbf{Parameters}  & \textbf{Returned Type} & \textbf{Visibility} \\ \hline
    ValidMove                          & x1, y1, x2, y2 : int & bool                   & Public              \\ \hline
    \end{tabular}
\end{table}

    \textbf{Parameters :} x1 and y1 are the coordinates of the piece's current position.
    x2 and y2 are the coordinates of the destination.
\\
    \textbf{Description :} This method checks if the move is valid for the pawn. 
    Valid moves are forward by one, forward by two if the pawn has not moved, and forward diagonally if the pawn is attacking.
    In these cases, the method returns true.
    Otherwise, the method returns false.

\begin{table}[H]
    \begin{tabular}{|l|l|l|l|}
    \hline
    \rowcolor[HTML]{EFEFEF} 
    \cellcolor[HTML]{EFEFEF}\textbf{Method Name} & \textbf{Parameters}  & \textbf{Returned Type} & \textbf{Visibility} \\ \hline
    CanPromote                                   & None                 & bool                   & Public              \\ \hline
    \end{tabular}
\end{table}

    \textbf{Description :} Overrides the CanPromote method in the Piece class.
    Returns true. 

\begin{table}[H]
    \begin{tabular}{|l|l|l|l|}
    \hline
    \rowcolor[HTML]{EFEFEF} 
    \cellcolor[HTML]{EFEFEF}\textbf{Method Name} & \textbf{Parameters}  & \textbf{Returned Type} & \textbf{Visibility} \\ \hline
    ToString                                   & None                 & String                   & Public              \\ \hline
    \end{tabular}
\end{table}

    \textbf{Description :} Overrides the ToString method in the Piece class.
    Returns the string "p" if the piece is white, and "P" if the piece is black. 

\subsubsection{Properties}

    Inherited from the StartingPiece class.
\newpage

%%%%%%%%%%%%
%Rook Class%

\subsection{Class Rook : StartingPiece}

    Extended from the StartingPiece class, the Rook class represents a rook.
\begin{table}[H]
    \begin{tabular}{|l|}
    \hline
    \cellcolor[HTML]{C0C0C0}\textbf{Rook} \\ \hline
    \cellcolor[HTML]{EFEFEF}                    \\ \hline
    +ValidMove(x1, y1, x2, y2) : bool           \\ \hline
    +ToString() : string                        \\ \hline
    \end{tabular}
\end{table}

\subsubsection{Fields}

    Inherited from the StartingPiece class.

\subsubsection{Methods}

\begin{table}[H]
    \begin{tabular}{|l|l|l|l|}
    \hline
    \rowcolor[HTML]{EFEFEF} 
    \cellcolor[HTML]{EFEFEF}\textbf{Method Name} & \textbf{Parameters}  & \textbf{Returned Type} & \textbf{Visibility} \\ \hline
    ValidMove                          & x1, y1, x2, y2 : int & bool                   & Public              \\ \hline
    \end{tabular}
\end{table}

    \textbf{Parameters :} x1 and y1 are the coordinates of the piece's current position. x2 and y2 are the coordinates of the destination. 
\\
    \textbf{Description :} This method checks if the move is valid for the rook. 
    Valid moves are horizontal and vertical directions.
    In these cases, the method returns true.
    Otherwise, the method returns false.
\begin{table}[H]
    \begin{tabular}{|l|l|l|l|}
    \hline
    \rowcolor[HTML]{EFEFEF} 
    \cellcolor[HTML]{EFEFEF}\textbf{Method Name} & \textbf{Parameters}  & \textbf{Returned Type} & \textbf{Visibility} \\ \hline
    ToString                                   & None                 & String                   & Public              \\ \hline
    \end{tabular}
\end{table}

    \textbf{Description :} Overrides the ToString method in the Piece class.
    Returns the string "r" if the piece is white, and "R" if the piece is black. 

\subsubsection{Properties}

    Inherited from the StartingPiece class.
\newpage

%%%%%%%%%%%%
%King Class%

\subsection{Class King : StartingPiece}

    Extended from the StartingPiece class, the King class represents a king.
\begin{table}[H]
    \begin{tabular}{|l|}
    \hline
    \cellcolor[HTML]{C0C0C0}\textbf{King} \\ \hline
    \cellcolor[HTML]{EFEFEF}                    \\ \hline
    +ValidMove(x1, y1, x2, y2) : bool           \\ \hline
    +IsEssential() : bool                       \\ \hline
    +ToString() : string                        \\ \hline
    \end{tabular}
\end{table}

\subsubsection{Fields}

    Inherited from the StartingPiece class.

\subsubsection{Methods}

\begin{table}[H]
    \begin{tabular}{|l|l|l|l|}
    \hline
    \rowcolor[HTML]{EFEFEF} 
    \cellcolor[HTML]{EFEFEF}\textbf{Method Name} & \textbf{Parameters}  & \textbf{Returned Type} & \textbf{Visibility} \\ \hline
    ValidMove                          & x1, y1, x2, y2 : int & bool                   & Public              \\ \hline
    \end{tabular}
\end{table}

    \textbf{Parameters :} x1 and y1 are the coordinates of the piece's current position.
    x2 and y2 are the coordinates of the destination.  
\\
    \textbf{Description :} This method checks if the move is valid for the king. 
    Valid moves are horizontally, vertically and diagonally by one square.
    Additionally, the king can move by two squares on the left and right if it has not moved, called castling.
    In these cases, the method returns true.
    Otherwise, the method returns false.

\begin{table}[H]
    \begin{tabular}{|l|l|l|l|}
    \hline
    \rowcolor[HTML]{EFEFEF} 
    \cellcolor[HTML]{EFEFEF}\textbf{Method Name} & \textbf{Parameters}  & \textbf{Returned Type} & \textbf{Visibility} \\ \hline
    IsEssential                                  & None                 & bool                   & Public              \\ \hline
    \end{tabular}
\end{table}

    \textbf{Description :} Overrides the IsEssential method in the Piece class.
    Returns true.

\begin{table}[H]
    \begin{tabular}{|l|l|l|l|}
    \hline
    \rowcolor[HTML]{EFEFEF} 
    \cellcolor[HTML]{EFEFEF}\textbf{Method Name} & \textbf{Parameters}  & \textbf{Returned Type} & \textbf{Visibility} \\ \hline
    ToString                                   & None                 & String                   & Public              \\ \hline
    \end{tabular}
\end{table}

    \textbf{Description :} Overrides the ToString method in the Piece class.
    Returns the string "k" if the piece is white, and "K" if the piece is black. 

\subsubsection{Properties}

    Inherited from the StartingPiece class.
\newpage

%%%%%%%%%%%%%%
%Knight Class%

\subsection{Class Knight : Piece}

    Extended from the Piece class, the Knight class represents a knight.
\begin{table}[H]
    \begin{tabular}{|l|}
    \hline
    \cellcolor[HTML]{C0C0C0}\textbf{Knight} \\ \hline
    \cellcolor[HTML]{EFEFEF}                    \\ \hline
    +ValidMove(x1, y1, x2, y2) : bool           \\ \hline
    +CanCollide() : bool                        \\ \hline
    +ToString() : string                        \\ \hline
    \end{tabular}
\end{table}

\subsubsection{Fields}

    Inherited from the Piece class.

\subsubsection{Methods}

\begin{table}[H]
    \begin{tabular}{|l|l|l|l|}
    \hline
    \rowcolor[HTML]{EFEFEF} 
    \cellcolor[HTML]{EFEFEF}\textbf{Method Name} & \textbf{Parameters}  & \textbf{Returned Type} & \textbf{Visibility} \\ \hline
    ValidMove                          & x1, y1, x2, y2 : int & bool                   & Public              \\ \hline
    \end{tabular}
\end{table}

    \textbf{Parameters :} x1 and y1 are the coordinates of the piece's current position.
    x2 and y2 are the coordinates of the destination.  
\\
    \textbf{Description :} This method checks if the move is valid for the knight. 
    Valid moves are moves in an L shape.
    An L shape is composed of two forward moves and one move to the left or right.
    Alternatively, it can be composed of two moves to the left or right and one forward move.
    The method returns true if the move is valid.
    Otherwise, the method returns false.

\begin{table}[H]
    \begin{tabular}{|l|l|l|l|}
    \hline
    \rowcolor[HTML]{EFEFEF} 
    \cellcolor[HTML]{EFEFEF}\textbf{Method Name} & \textbf{Parameters}  & \textbf{Returned Type} & \textbf{Visibility} \\ \hline
    CanCollide                                   & None                 & bool                   & Public              \\ \hline
    \end{tabular}
\end{table}

    \textbf{Description :} Overrides the CanCollide method in the Piece class.
    Returns false.

\begin{table}[H]
    \begin{tabular}{|l|l|l|l|}
    \hline
    \rowcolor[HTML]{EFEFEF} 
    \cellcolor[HTML]{EFEFEF}\textbf{Method Name} & \textbf{Parameters}  & \textbf{Returned Type} & \textbf{Visibility} \\ \hline
    ToString                                   & None                 & String                   & Public              \\ \hline
    \end{tabular}
\end{table}

    \textbf{Description :} Overrides the ToString method in the Piece class.
    Returns the string "n" if the piece is white, and "N" if the piece is black. 

\subsubsection{Properties}

    Inherited from the StartingPiece class.

\newpage

%Bishop Class%
%%%%%%%%%%%%%%

\subsection{Class Bishop : Piece}

    Extended from the Piece class, the Bishop class represents a bishop.
\begin{table}[H]
    \begin{tabular}{|l|}
    \hline
    \cellcolor[HTML]{C0C0C0}\textbf{Bishop} \\ \hline
    \cellcolor[HTML]{EFEFEF}                    \\ \hline
    +ValidMove(x1, y1, x2, y2) : bool           \\ \hline
    +ToString() : string                        \\ \hline
    \end{tabular}
\end{table}

\subsubsection{Fields}

    Inherited from the Piece class.

\subsubsection{Methods}

\begin{table}[H]
    \begin{tabular}{|l|l|l|l|}
    \hline
    \rowcolor[HTML]{EFEFEF} 
    \cellcolor[HTML]{EFEFEF}\textbf{Method Name} & \textbf{Parameters}  & \textbf{Returned Type} & \textbf{Visibility} \\ \hline
    ValidMove                          & x1, y1, x2, y2 : int & bool                   & Public              \\ \hline
    \end{tabular}
\end{table}

    \textbf{Parameters :} x1 and y1 are the coordinates of the piece's current position.
    x2 and y2 are the coordinates of the destination.  
\\
    \textbf{Description :} This method checks if the move is valid for the bishop. 
    Valid moves are moves in a diagonal direction.
    The method returns true if the move is valid.
    Otherwise, the method returns false.
    
\begin{table}[H]
    \begin{tabular}{|l|l|l|l|}
    \hline
    \rowcolor[HTML]{EFEFEF} 
    \cellcolor[HTML]{EFEFEF}\textbf{Method Name} & \textbf{Parameters}  & \textbf{Returned Type} & \textbf{Visibility} \\ \hline
    ToString                                   & None                 & String                   & Public              \\ \hline
    \end{tabular}
\end{table}

    \textbf{Description :} Overrides the ToString method in the Piece class.
    Returns the string "b" if the piece is white, and "B" if the piece is black. 

\subsubsection{Properties}

    Inherited from the StartingPiece class.

\newpage

%Queen%
%%%%%%%

\subsection{Class Queen : Piece}

    Extended from the Piece class, the Queen class represents a queen.
\begin{table}[H]
    \begin{tabular}{|l|}
    \hline
    \cellcolor[HTML]{C0C0C0}\textbf{Queen} \\ \hline
    \cellcolor[HTML]{EFEFEF}                    \\ \hline
    +ValidMove(x1, y1, x2, y2) : bool           \\ \hline
    +ToString() : string                        \\ \hline
    \end{tabular}
\end{table}

\subsubsection{Fields}

    Inherited from the Piece class.

\subsubsection{Methods}

\begin{table}[H]
    \begin{tabular}{|l|l|l|l|}
    \hline
    \rowcolor[HTML]{EFEFEF} 
    \cellcolor[HTML]{EFEFEF}\textbf{Method Name} & \textbf{Parameters}  & \textbf{Returned Type} & \textbf{Visibility} \\ \hline
    ValidMove                          & x1, y1, x2, y2 : int & bool                   & Public              \\ \hline
    \end{tabular}
\end{table}

    \textbf{Parameters :} x1 and y1 are the coordinates of the piece's current position.
    x2 and y2 are the coordinates of the destination.  
\\
    \textbf{Description :} This method checks if the move is valid for the queen. 
    Valid moves are moves in a horizontal, vertical, or diagonal direction.
    The method returns true if the move is valid.
    Otherwise, the method returns false.

\begin{table}[H]
    \begin{tabular}{|l|l|l|l|}
    \hline
    \rowcolor[HTML]{EFEFEF} 
    \cellcolor[HTML]{EFEFEF}\textbf{Method Name} & \textbf{Parameters}  & \textbf{Returned Type} & \textbf{Visibility} \\ \hline
    ToString                                   & None                 & String                   & Public              \\ \hline
    \end{tabular}
\end{table}

    \textbf{Description :} Overrides the ToString method in the Piece class.
    Returns the string "q" if the piece is white, and "Q" if the piece is black. 

\subsubsection{Properties}

    Inherited from the Piece class.

\newpage

%Match Class%
%%%%%%%%%%%%%

\subsection{Class Match}

The Match class represents a chess match. It contains the board, it's history, the current turn and has methods for the Game Controller.

\begin{table}[H]
    \begin{tabular}{|l|}
    \hline
    \rowcolor[HTML]{C0C0C0} 
    \textbf{Match}                                    \\ \hline
    \rowcolor[HTML]{EFEFEF} 
    -\_board : Board                                  \\ \hline
    \rowcolor[HTML]{EFEFEF} 
    -\_current : Colour                               \\ \hline
    \rowcolor[HTML]{EFEFEF} 
    -\_history : string{[}{]}                         \\ \hline
    \rowcolor[HTML]{EFEFEF} 
    -\_turnNumber : int                               \\ \hline
    +ExportBoard() : string                           \\ \hline
    +ExportHistory() : string[]                       \\ \hline
    +ValidTurn(int origin, int target) : bool         \\ \hline
    +MakeTurn(int origin, int target) : void          \\ \hline
    +ValidSelection(int cell, bool firstClick) : bool \\ \hline
    +HasPromotable(int target) : bool                 \\ \hline
    +Check() : bool                                   \\ \hline
    +Checkmate() : bool                               \\ \hline
    +Stalemate() : bool                               \\ \hline
    +Castle(int origin, int target) : void            \\ \hline
\end{tabular}
\end{table}

\subsubsection{Fields}

\begin{table}[H]
    \begin{tabular}{llllll}
    \hline
    \multicolumn{1}{|l|}{\cellcolor[HTML]{EFEFEF}\textbf{Field Name}} & \multicolumn{1}{l|}{\cellcolor[HTML]{EFEFEF}\textbf{Type}} & \multicolumn{1}{l|}{\cellcolor[HTML]{EFEFEF}\textbf{Visibility}} \\ \hline
    \multicolumn{1}{|l|}{\_board}                                     & \multicolumn{1}{l|}{Board}                                 & \multicolumn{1}{l|}{Private}                                     \\ \hline
    \end{tabular}
\end{table}

\textbf{Description :}  Represents the board of the match. 

\begin{table}[H]
    \begin{tabular}{llllll}
    \hline
    \multicolumn{1}{|l|}{\cellcolor[HTML]{EFEFEF}\textbf{Field Name}} & \multicolumn{1}{l|}{\cellcolor[HTML]{EFEFEF}\textbf{Type}} & \multicolumn{1}{l|}{\cellcolor[HTML]{EFEFEF}\textbf{Visibility}} \\ \hline
    \multicolumn{1}{|l|}{\_current}                                     & \multicolumn{1}{l|}{Colour}                                 & \multicolumn{1}{l|}{Private}                                     \\ \hline
    \end{tabular}
\end{table}

\textbf{Description :} Represents the current Colour that is playing, which is either white or black. 

\begin{table}[H]
    \begin{tabular}{llllll}
    \hline
    \multicolumn{1}{|l|}{\cellcolor[HTML]{EFEFEF}\textbf{Field Name}} & \multicolumn{1}{l|}{\cellcolor[HTML]{EFEFEF}\textbf{Type}} & \multicolumn{1}{l|}{\cellcolor[HTML]{EFEFEF}\textbf{Visibility}} \\ \hline
    \multicolumn{1}{|l|}{\_history}                                     & \multicolumn{1}{l|}{string[]}                            & \multicolumn{1}{l|}{Private}                                     \\ \hline
    \end{tabular}
\end{table}

\textbf{Description :} Represents the history of the match.
Each board is represented by a string, which is a concatenation of the piece's name.
For example, an initial board is represented as:
\\"RNBKQBNRPPPPPPPP.................................pppppppprnkqbnr". 

\begin{table}[H]
    \begin{tabular}{llllll}
    \hline
    \multicolumn{1}{|l|}{\cellcolor[HTML]{EFEFEF}\textbf{Field Name}} & \multicolumn{1}{l|}{\cellcolor[HTML]{EFEFEF}\textbf{Type}} & \multicolumn{1}{l|}{\cellcolor[HTML]{EFEFEF}\textbf{Visibility}} \\ \hline
    \multicolumn{1}{|l|}{\_turnNumber}                                & \multicolumn{1}{l|}{int}                                   & \multicolumn{1}{l|}{Private}                                     \\ \hline
    \end{tabular}
\end{table}

\textbf{Description :} Represents the number of turns that has been played. 

\subsubsection{Methods}

\begin{table}[H]
    \begin{tabular}{|l|l|l|l|}
    \hline
    \rowcolor[HTML]{EFEFEF} 
    \cellcolor[HTML]{EFEFEF}\textbf{Method Name} & \textbf{Parameters}  & \textbf{Returned Type} & \textbf{Visibility} \\ \hline
    ExportBoard                                  & none                 & string                 & Public              \\ \hline
    \end{tabular}
\end{table}

\textbf{Description :} Exports the board to a string. 

\begin{table}[H]
    \begin{tabular}{|l|l|l|l|}
    \hline
    \rowcolor[HTML]{EFEFEF} 
    \cellcolor[HTML]{EFEFEF}\textbf{Method Name} & \textbf{Parameters}  & \textbf{Returned Type} & \textbf{Visibility} \\ \hline
    ExportHistory                                & none                 & string[]                 & Public            \\ \hline
    \end{tabular}
\end{table}

\textbf{Description :} Exports the history to an array of strings.   

\begin{table}[H]
    \begin{tabular}{|l|l|l|l|}
    \hline
    \rowcolor[HTML]{EFEFEF} 
    \cellcolor[HTML]{EFEFEF}\textbf{Method Name} & \textbf{Parameters}    & \textbf{Returned Type} & \textbf{Visibility} \\ \hline
    ValidTurn                                    & int origin, int target & bool                   & Public              \\ \hline
    \end{tabular}
\end{table}

\textbf{Parameters :} Origin represents the piece's starting position, while target represents the piece's end position. 
\\

\textbf{Description :} Evaluates whether the turn is valid.  

\begin{table}[H]
    \begin{tabular}{|l|l|l|l|}
    \hline
    \rowcolor[HTML]{EFEFEF} 
    \cellcolor[HTML]{EFEFEF}\textbf{Method Name} & \textbf{Parameters}    & \textbf{Returned Type} & \textbf{Visibility} \\ \hline
    MakeTurn                                     & int origin, int target & void                   & Public              \\ \hline
    \end{tabular}
\end{table}

\textbf{Parameters :} Origin represents the piece's starting position, while target represents the piece's end position.  
\\

\textbf{Description :} Makes a turn, updating the board, history, turn number and current colour. 

\begin{table}[H]
    \begin{tabular}{|l|l|l|l|}
    \hline
    \rowcolor[HTML]{EFEFEF} 
    \cellcolor[HTML]{EFEFEF}\textbf{Method Name} & \textbf{Parameters}       & \textbf{Returned Type} & \textbf{Visibility} \\ \hline
    ValidSelection                               & int cell, bool firstClick & bool                   & Public              \\ \hline
    \end{tabular}
\end{table}

\textbf{Parameters :} Cell represents the cell's position, while firstClick represents whether this is the first click of the turn. 
\\

\textbf{Description :} Evaluates whether the selection is valid.
On the first click, the method checks if the cell has a piece of the current colour.
On the second click, the method checks if the cell is empty, or has a piece of the other colour.

\begin{table}[H]
    \begin{tabular}{|l|l|l|l|}
    \hline
    \rowcolor[HTML]{EFEFEF} 
    \cellcolor[HTML]{EFEFEF}\textbf{Method Name} & \textbf{Parameters}     & \textbf{Returned Type} & \textbf{Visibility} \\ \hline
    HasPromotable                                & int target                    & bool                   & Public              \\ \hline
    \end{tabular}
\end{table}


\textbf{Parameters :} Target represents the cell's position.  
\textbf{Description :} Evaluates whether the cell has a promotable piece. 

\begin{table}[H]
    \begin{tabular}{|l|l|l|l|}
    \hline
    \rowcolor[HTML]{EFEFEF} 
    \cellcolor[HTML]{EFEFEF}\textbf{Method Name} & \textbf{Parameters}     & \textbf{Returned Type} & \textbf{Visibility} \\ \hline
    Check                                        & none                    & bool                   & Public              \\ \hline
    \end{tabular}
\end{table}

\textbf{Description :} Checks if the essential piece is under attack.  

\begin{table}[H]
    \begin{tabular}{|l|l|l|l|}
    \hline
    \rowcolor[HTML]{EFEFEF} 
    \cellcolor[HTML]{EFEFEF}\textbf{Method Name} & \textbf{Parameters}     & \textbf{Returned Type} & \textbf{Visibility} \\ \hline
    Checkmate                                        & none                    & bool                   & Public              \\ \hline
    \end{tabular}
\end{table}

\textbf{Description :} Checks if the essential piece is under attack and there are no valid moves.    

\begin{table}[H]
    \begin{tabular}{|l|l|l|l|}
    \hline
    \rowcolor[HTML]{EFEFEF} 
    \cellcolor[HTML]{EFEFEF}\textbf{Method Name} & \textbf{Parameters}     & \textbf{Returned Type} & \textbf{Visibility} \\ \hline
    Stalemate                                    & none                    & bool                   & Public              \\ \hline
    \end{tabular}
\end{table}

\textbf{Description :} Checks if there are no valid moves for the current colour.    

\begin{table}[H]
    \begin{tabular}{|l|l|l|l|}
    \hline
    \rowcolor[HTML]{EFEFEF} 
    \cellcolor[HTML]{EFEFEF}\textbf{Method Name} & \textbf{Parameters}      & \textbf{Returned Type} & \textbf{Visibility} \\ \hline
    Castle                                       & int origin, int target   & bool                   & Public              \\ \hline
    \end{tabular}
\end{table}

\textbf{Description :} Checks if the move is a castle.   


\subsubsection{Properties}

\begin{table}[H]
    \begin{tabular}{|l|l|l|l|}
    \hline
    \rowcolor[HTML]{EFEFEF} 
    \cellcolor[HTML]{EFEFEF}\textbf{Property Name} & \textbf{Parameters}  & \textbf{Returned Type} & \textbf{Visibility} \\ \hline
    Current                                        & None                 & Colour                 & Public              \\ \hline
    \end{tabular}
\end{table}

\textbf{Description :} Gets or Sets the current colour. 

\newpage

%Board Class%
%%%%%%%%%%%%%

\subsection{Class Board}

The Board class represents the board of the game.
It contains all 64 cells, as well as methods to manipulate them.

\begin{table}[H]
    \begin{tabular}{|l|}
    \hline
    \rowcolor[HTML]{C0C0C0} 
    \textbf{Board}                                             \\ \hline
    \rowcolor[HTML]{EFEFEF}                                    
    -\_cells : Cell[]                                          \\ \hline
    +ToString() : string                                       \\ \hline
    +Collision(int origin, int target) : bool                  \\ \hline
    +SameColour (int cell, Colour colour) : bool               \\ \hline
    +ValidMove(int origin, int target) : bool                  \\ \hline
    +MoveCellTo(int origin, int target) : void                 \\ \hline
    +GenerateBoard(string board) : void                        \\ \hline
    +IsEssetialExposed(Colour colour) : bool                   \\ \hline
    +HasPromotable(int target) : bool                          \\ \hline
    +GetAssaillants(Colour colour) : List\textless{}int\textgreater{}                 \\ \hline
    -GetEssentialPiece(Colour colour) : int                    \\ \hline
    -GetAttackingPieces(Colour colour, int target) : List\textless{}int\textgreater{} \\ \hline
    -HasAttackersAroundEssential(Colour colour) : bool          \\ \hline
    \end{tabular}
\end{table}

\subsubsection{Fields}

\begin{table}[H]
    \begin{tabular}{llllll}
    \hline
    \multicolumn{1}{|l|}{\cellcolor[HTML]{EFEFEF}\textbf{Field Name}} & \multicolumn{1}{l|}{\cellcolor[HTML]{EFEFEF}\textbf{Type}} & \multicolumn{1}{l|}{\cellcolor[HTML]{EFEFEF}\textbf{Visibility}} \\ \hline
    \multicolumn{1}{|l|}{\_cells}                                     & \multicolumn{1}{l|}{Cell[]}                                & \multicolumn{1}{l|}{Private}                                     \\ \hline
    \end{tabular}
\end{table}

\textbf{Description :}  Represents the board. 

\subsubsection{Methods}

\begin{table}[H]
    \begin{tabular}{|l|l|l|l|}
    \hline
    \rowcolor[HTML]{EFEFEF} 
    \cellcolor[HTML]{EFEFEF}\textbf{Method Name} & \textbf{Parameters}    & \textbf{Returned Type} & \textbf{Visibility} \\ \hline
    Collision                                    & int origin, int target & bool                   & Public              \\ \hline
    \end{tabular}
\end{table}

\textbf{Parameters :} Origin represents the starting cell, target represents the target cell. 
\\

\textbf{Description :} Evaluates if the move is valid by calculating collisions between the origin and the target. 

\begin{table}[H]
    \begin{tabular}{|l|l|l|l|}
    \hline
    \rowcolor[HTML]{EFEFEF} 
    \cellcolor[HTML]{EFEFEF}\textbf{Method Name} & \textbf{Parameters}     & \textbf{Returned Type} & \textbf{Visibility} \\ \hline
    SameColour                                   & int cell, Colour colour & bool                   & Public              \\ \hline
    \end{tabular}
\end{table}

\textbf{Parameters :} Cell represents the cell to be evaluated, colour represents the colour to be evaluated. 
\\

\textbf{Description :} Evaluates if the cell is of the same colour as the parameter colour. 

\begin{table}[H]
    \begin{tabular}{|l|l|l|l|}
    \hline
    \rowcolor[HTML]{EFEFEF} 
    \cellcolor[HTML]{EFEFEF}\textbf{Method Name} & \textbf{Parameters}     & \textbf{Returned Type} & \textbf{Visibility} \\ \hline
    ValidMove                                   & int origin, int target   & bool                   & Public              \\ \hline
    \end{tabular}
\end{table}

\textbf{Parameters :} Origin represents the starting cell, target represents the target cell.  
\\

\textbf{Description :} Evaluates if the move is valid.

\begin{table}[H]
    \begin{tabular}{|l|l|l|l|}
    \hline
    \rowcolor[HTML]{EFEFEF} 
    \cellcolor[HTML]{EFEFEF}\textbf{Method Name} & \textbf{Parameters}     & \textbf{Returned Type} & \textbf{Visibility} \\ \hline
    MoveCellTo                                   & int origin, int target  & bool                   & Public              \\ \hline
    \end{tabular}
\end{table}

\textbf{Parameters :} Origin represents the starting cell, target represents the target cell. 
\\

\textbf{Description :} Moves the cell from the origin to the target, and sets the origin cell to an empty cell. 

\begin{table}[H]
    \begin{tabular}{|l|l|l|l|}
    \hline
    \rowcolor[HTML]{EFEFEF} 
    \cellcolor[HTML]{EFEFEF}\textbf{Method Name} & \textbf{Parameters}     & \textbf{Returned Type} & \textbf{Visibility} \\ \hline
    GenerateBoard                              & string board            & void                   & Public              \\ \hline
    \end{tabular}
\end{table}

\textbf{Parameters :} Board represents a string to be parsed. 
\\

\textbf{Description :} Generates a new board from a string.

\begin{table}[H]
    \begin{tabular}{|l|l|l|l|}
    \hline
    \rowcolor[HTML]{EFEFEF} 
    \cellcolor[HTML]{EFEFEF}\textbf{Method Name} & \textbf{Parameters}     & \textbf{Returned Type} & \textbf{Visibility} \\ \hline
    IsEssentialExposed                           & Colour colour           & bool                   & Public              \\ \hline
    \end{tabular}
\end{table}

\textbf{Parameters :} Colour represents the colour of the essential piece to be evaluated. 
\\

\textbf{Description :} Evaluates if the essential piece is exposed to attackers.

\begin{table}[H]
    \begin{tabular}{|l|l|l|l|}
    \hline
    \rowcolor[HTML]{EFEFEF} 
    \cellcolor[HTML]{EFEFEF}\textbf{Method Name} & \textbf{Parameters}     & \textbf{Returned Type} & \textbf{Visibility} \\ \hline
    HasPromotable                                & int target              & bool                   & Public              \\ \hline
    \end{tabular}
\end{table}

\textbf{Parameters :} Target represents the target cell.  
\\

\textbf{Description :} Evaluates if the target cell has a promotable piece. 

\begin{table}[H]
    \begin{tabular}{|l|l|l|l|}
    \hline
    \rowcolor[HTML]{EFEFEF} 
    \cellcolor[HTML]{EFEFEF}\textbf{Method Name} & \textbf{Parameters}     & \textbf{Returned Type} & \textbf{Visibility} \\ \hline
    GetAsaillants                                & Colour colour           & List\textless{}int\textgreater{}                   & Public              \\ \hline
    \end{tabular}
\end{table}

\textbf{Parameters :} Colour represents the colour of the assailant pieces. 
\textbf{Description :} Returns a list of pieces of the specified colour that can attack the essential piece.  

\begin{table}[H]
    \begin{tabular}{|l|l|l|l|}
    \hline
    \rowcolor[HTML]{EFEFEF} 
    \cellcolor[HTML]{EFEFEF}\textbf{Method Name} & \textbf{Parameters}     & \textbf{Returned Type} & \textbf{Visibility} \\ \hline
    GetEssentialPiece                            & Colour colour           & int                   & Public              \\ \hline
    \end{tabular}
\end{table}

\textbf{Parameters :} Colour represents the colour of the essential piece to be evaluated.  
\textbf{Description :} Returns the index of the essential piece.  

\begin{table}[H]
    \begin{tabular}{|l|l|l|l|}
    \hline
    \rowcolor[HTML]{EFEFEF} 
    \cellcolor[HTML]{EFEFEF}\textbf{Method Name} & \textbf{Parameters}        & \textbf{Returned Type} & \textbf{Visibility} \\ \hline
    GetAttackingPieces                           & Colour colour, int target  & List\textless{}int\textgreater{}                   & Public              \\ \hline
    \end{tabular}
\end{table}

\textbf{Parameters :} Colour represents the colour of the assailant pieces, while target represents the target cell.   
\textbf{Description :} Returns a list of pieces of the specified colour that can attack the target cell. 

\begin{table}[H]
    \begin{tabular}{|l|l|l|l|}
    \hline
    \rowcolor[HTML]{EFEFEF} 
    \cellcolor[HTML]{EFEFEF}\textbf{Method Name} & \textbf{Parameters}     & \textbf{Returned Type} & \textbf{Visibility} \\ \hline
    HasAttackersAroundEssential                  & Colour colour           & bool                   & Public              \\ \hline
    \end{tabular}
\end{table}

\textbf{Parameters :}  Colour represents the colour of the assailant pieces.  
\textbf{Description :} Evaluates if the specified colour has pieces that can attack the essential piece, as well as all eight cells around it. 

\newpage

%Cell Class%
%%%%%%%%%%%%%

\subsection{Class Cell}

The Cell class represents a single cell on the board.

\begin{table}[H]
    \begin{tabular}{|l|}
    \hline
    \rowcolor[HTML]{C0C0C0} 
    \textbf{Cell}                                    \\ \hline
    \rowcolor[HTML]{EFEFEF}
    -\_piece : Nullable\textless{}Piece\textgreater{}                        \\ \hline
    +IsEmpty() : bool                                 \\ \hline
    +HasCollision() : bool                            \\ \hline
    +HasPromotable() : bool                           \\ \hline
    +HasEssetial() : bool                             \\ \hline
    +ValidMove(int x1, int y1, int x2, int y2) : bool \\ \hline
    +Colour() : Colour                                \\ \hline
    \end{tabular}
\end{table}

\subsubsection{Fields}

\begin{table}[H]
    \begin{tabular}{llllll}
    \hline
    \multicolumn{1}{|l|}{\cellcolor[HTML]{EFEFEF}\textbf{Field Name}} & \multicolumn{1}{l|}{\cellcolor[HTML]{EFEFEF}\textbf{Type}} & \multicolumn{1}{l|}{\cellcolor[HTML]{EFEFEF}\textbf{Visibility}} \\ \hline
    \multicolumn{1}{|l|}{\_piece}                                     & \multicolumn{1}{l|}{Nullable\textless{}Piece\textgreater{}}                       & \multicolumn{1}{l|}{Private}                                     \\ \hline
    \end{tabular}
\end{table}

\textbf{Description :} Represents the piece that is currently in the cell. Can be null. 

\subsubsection{Methods}

\begin{table}[H]
    \begin{tabular}{|l|l|l|l|}
    \hline
    \rowcolor[HTML]{EFEFEF} 
    \cellcolor[HTML]{EFEFEF}\textbf{Method Name} & \textbf{Parameters}    & \textbf{Returned Type} & \textbf{Visibility} \\ \hline
    IsEmpty                                      & none                   & bool                   & Public              \\ \hline
    \end{tabular}
\end{table}

\textbf{Description :} Returns true if the cell is empty, false otherwise.  

\begin{table}[H]
    \begin{tabular}{|l|l|l|l|}
    \hline
    \rowcolor[HTML]{EFEFEF} 
    \cellcolor[HTML]{EFEFEF}\textbf{Method Name} & \textbf{Parameters}     & \textbf{Returned Type} & \textbf{Visibility} \\ \hline
    HasCollision                                 & none                    & bool                   & Public              \\ \hline
    \end{tabular}
\end{table}

\textbf{Description :} Returns true if the cell has a piece that can collide with others, false otherwise.   

\begin{table}[H]
    \begin{tabular}{|l|l|l|l|}
    \hline
    \rowcolor[HTML]{EFEFEF} 
    \cellcolor[HTML]{EFEFEF}\textbf{Method Name} & \textbf{Parameters}     & \textbf{Returned Type} & \textbf{Visibility} \\ \hline
    HasPromotable                                & none                    & bool                   & Public              \\ \hline
    \end{tabular}
\end{table}

\textbf{Description :} Returns true if the cell has a piece that can be promoted, false otherwise.  

\begin{table}[H]
    \begin{tabular}{|l|l|l|l|}
    \hline
    \rowcolor[HTML]{EFEFEF} 
    \cellcolor[HTML]{EFEFEF}\textbf{Method Name} & \textbf{Parameters}     & \textbf{Returned Type} & \textbf{Visibility} \\ \hline
    HasEssential                                 & none                    & bool                   & Public              \\ \hline
    \end{tabular}
\end{table}

\textbf{Description :} Returns true if the cell has a piece that is essential, false otherwise.   

\begin{table}[H]
    \begin{tabular}{|l|l|l|l|}
    \hline
    \rowcolor[HTML]{EFEFEF} 
    \cellcolor[HTML]{EFEFEF}\textbf{Method Name} & \textbf{Parameters}            & \textbf{Returned Type} & \textbf{Visibility} \\ \hline
    ValidMove                                    & int x1, int y1, int x2, int y2 & bool                   & Public              \\ \hline
    \end{tabular}
\end{table}

\textbf{Parameters :} x1 and y1 are the coordinates of the piece.
x2 and y2 are the coordinates of the target location. 
\\

\textbf{Description :} Evaluates whether the piece can move to the target location. 

\begin{table}[H]
    \begin{tabular}{|l|l|l|l|}
    \hline
    \rowcolor[HTML]{EFEFEF} 
    \cellcolor[HTML]{EFEFEF}\textbf{Method Name} & \textbf{Parameters}     & \textbf{Returned Type} & \textbf{Visibility} \\ \hline
    Colour                                       & none                    & Colour                   & Public              \\ \hline
    \end{tabular}
\end{table}

\textbf{Description :} Returns the colour of the piece.  

\newpage

%Player Class%
%%%%%%%%%%%%%%

\subsection{Class Player}

The Player class represents a player that can play the game. 

\begin{table}[H]
    \begin{tabular}{|l|}
    \hline
    \rowcolor[HTML]{C0C0C0} 
    \textbf{Player}          \\ \hline
    \rowcolor[HTML]{EFEFEF}
    -\_name : string        \\ \hline
    -\_points : int         \\ \hline
    \end{tabular}
\end{table}

\subsubsection{Fields}

\begin{table}[H]
    \begin{tabular}{llllll}
    \hline
    \multicolumn{1}{|l|}{\cellcolor[HTML]{EFEFEF}\textbf{Field Name}} & \multicolumn{1}{l|}{\cellcolor[HTML]{EFEFEF}\textbf{Type}} & \multicolumn{1}{l|}{\cellcolor[HTML]{EFEFEF}\textbf{Visibility}} \\ \hline
    \multicolumn{1}{|l|}{\_name}                                      & \multicolumn{1}{l|}{string}                                & \multicolumn{1}{l|}{Private}                                     \\ \hline
    \end{tabular}
\end{table}

\textbf{Description :} Represents the name of the player.  

\begin{table}[H]
    \begin{tabular}{llllll}
    \hline
    \multicolumn{1}{|l|}{\cellcolor[HTML]{EFEFEF}\textbf{Field Name}} & \multicolumn{1}{l|}{\cellcolor[HTML]{EFEFEF}\textbf{Type}} & \multicolumn{1}{l|}{\cellcolor[HTML]{EFEFEF}\textbf{Visibility}} \\ \hline
    \multicolumn{1}{|l|}{\_points}                                      & \multicolumn{1}{l|}{int}                                 & \multicolumn{1}{l|}{Private}                                   \\ \hline
    \end{tabular}
\end{table}

\textbf{Description :} Represents the amount of victory points the player has.  

\newpage

%NEW SECTION CONTROLLERS%
%%%%%%%%%%%%%%%%%%%%%%%%%

\section{Controllers}

%Chess Class%
%%%%%%%%%%%%%

\subsection{Class Chess}

\begin{table}[H]
    \begin{tabular}{|l|}
    \hline
    \rowcolor[HTML]{C0C0C0} 
    \textbf{Chess}                                             \\ \hline
    \rowcolor[HTML]{EFEFEF} 
    -\_listGames : List\textless{}{}GameController\textgreater{}{} \\ \hline
    +main() : void                                             \\ \hline
    +NewGame() : void                                          \\ \hline
    +StartGame(Player{[}2{]} players) : void                   \\ \hline
    +ManagePlayers() : void                                    \\ \hline
    +Exit() : void                                             \\ \hline
    \end{tabular}
\end{table}

\subsubsection{Fields}

\begin{table}[H]
    \begin{tabular}{llllll}
    \hline
    \multicolumn{1}{|l|}{\cellcolor[HTML]{EFEFEF}\textbf{Field Name}} & \multicolumn{1}{l|}{\cellcolor[HTML]{EFEFEF}\textbf{Type}} & \multicolumn{1}{l|}{\cellcolor[HTML]{EFEFEF}\textbf{Visibility}} \\ \hline
    \multicolumn{1}{|l|}{\_listGames}                                 & \multicolumn{1}{l|}{List\textless{}GameController\textgreater{}}                                & \multicolumn{1}{l|}{Private}                                     \\ \hline
    \end{tabular}
\end{table}

\textbf{Description :} This field is a list of the active games.

\subsubsection{Methods}

\begin{table}[H]
    \begin{tabular}{|l|l|l|l|}
    \hline
    \rowcolor[HTML]{EFEFEF} 
    \cellcolor[HTML]{EFEFEF}\textbf{Method Name} & \textbf{Parameters}    & \textbf{Returned Type} & \textbf{Visibility} \\ \hline
    main                                         & none                   & void                   & Public              \\ \hline
    \end{tabular}
\end{table}

\textbf{Description :} This is the entry point of the program.

\begin{table}[H]
    \begin{tabular}{|l|l|l|l|}
    \hline
    \rowcolor[HTML]{EFEFEF} 
    \cellcolor[HTML]{EFEFEF}\textbf{Method Name} & \textbf{Parameters}    & \textbf{Returned Type} & \textbf{Visibility} \\ \hline
    NewGame                                      & none                   & void                   & Public              \\ \hline
    \end{tabular}
\end{table}

\textbf{Description :} This method creates a new game with its own GameController, Match, Board etc\dots

\begin{table}[H]
    \begin{tabular}{|l|l|l|l|}
    \hline
    \rowcolor[HTML]{EFEFEF} 
    \cellcolor[HTML]{EFEFEF}\textbf{Method Name} & \textbf{Parameters}    & \textbf{Returned Type} & \textbf{Visibility} \\ \hline
    StartGame                                    & Player[2] players      & void                   & Public              \\ \hline
    \end{tabular}
\end{table}

\textbf{Description :} This method starts the game with the players given.

\begin{table}[H]
    \begin{tabular}{|l|l|l|l|}
    \hline
    \rowcolor[HTML]{EFEFEF} 
    \cellcolor[HTML]{EFEFEF}\textbf{Method Name} & \textbf{Parameters}    & \textbf{Returned Type} & \textbf{Visibility} \\ \hline
    ManagePlayers                                & none                   & void                   & Public              \\ \hline
    \end{tabular}
\end{table}

\textbf{Description :} This method makes the link between FormMenu and FormLeaderboard.

\begin{table}[H]
    \begin{tabular}{|l|l|l|l|}
    \hline
    \rowcolor[HTML]{EFEFEF} 
    \cellcolor[HTML]{EFEFEF}\textbf{Method Name} & \textbf{Parameters}    & \textbf{Returned Type} & \textbf{Visibility} \\ \hline
    Exit                                         & none                   & void                   & Public              \\ \hline
    \end{tabular}
\end{table}

\textbf{Description :} This is the exit point method.

\newpage

%GameController Class%
%%%%%%%%%%%%%%%%%%%%%%

\subsection{Class GameController}

\begin{table}[H]
    \begin{tabular}{|l|}
    \hline
    \rowcolor[HTML]{C0C0C0} 
    \textbf{GameController}              \\ \hline
    \rowcolor[HTML]{EFEFEF} 
    -\_main : Chess                      \\ \hline
    \rowcolor[HTML]{EFEFEF} 
    -\_selected : int                    \\ \hline
    \rowcolor[HTML]{EFEFEF} 
    -\_match : Match                     \\ \hline
    \rowcolor[HTML]{EFEFEF} 
    -\_playerA : Player                  \\ \hline
    \rowcolor[HTML]{EFEFEF} 
    -\_playerB : Player                  \\ \hline
    \rowcolor[HTML]{EFEFEF} 
    -\_tieCounter : int                  \\ \hline
    \rowcolor[HTML]{EFEFEF} 
    -\_view : FormMatch                  \\ \hline
    -Rules(int origin, int target) : void\\ \hline
    -Check() : bool                      \\ \hline
    -Checkmate() : bool                  \\ \hline
    -Castle() : bool                     \\ \hline
    -Stalemate() : bool                  \\ \hline
    -FiftyTurns() : bool                 \\ \hline
    -SameBoard() : bool                  \\ \hline
    -Turn(int origin, int target) : void \\ \hline
    +Selection(int cell) : void          \\ \hline
    +Resign() : void                     \\ \hline
    \end{tabular}
\end{table}

\subsubsection{Fields}

\begin{table}[H]
    \begin{tabular}{llllll}
    \hline
    \multicolumn{1}{|l|}{\cellcolor[HTML]{EFEFEF}\textbf{Field Name}} & \multicolumn{1}{l|}{\cellcolor[HTML]{EFEFEF}\textbf{Type}} & \multicolumn{1}{l|}{\cellcolor[HTML]{EFEFEF}\textbf{Visibility}} \\ \hline
    \multicolumn{1}{|l|}{\_main}                                      & \multicolumn{1}{l|}{Chess}                                 & \multicolumn{1}{l|}{Private}                                     \\ \hline
    \end{tabular}
\end{table}

\textbf{Description :} This field represent the chess which the game controller will talk to.

\begin{table}[H]
    \begin{tabular}{llllll}
    \hline
    \multicolumn{1}{|l|}{\cellcolor[HTML]{EFEFEF}\textbf{Field Name}} & \multicolumn{1}{l|}{\cellcolor[HTML]{EFEFEF}\textbf{Type}} & \multicolumn{1}{l|}{\cellcolor[HTML]{EFEFEF}\textbf{Visibility}} \\ \hline
    \multicolumn{1}{|l|}{\_selected}                                      & \multicolumn{1}{l|}{int}                                 & \multicolumn{1}{l|}{Private}                                     \\ \hline
    \end{tabular}
\end{table}

\textbf{Description :} This field represent the selected cell.

\begin{table}[H]
    \begin{tabular}{llllll}
    \hline
    \multicolumn{1}{|l|}{\cellcolor[HTML]{EFEFEF}\textbf{Field Name}} & \multicolumn{1}{l|}{\cellcolor[HTML]{EFEFEF}\textbf{Type}} & \multicolumn{1}{l|}{\cellcolor[HTML]{EFEFEF}\textbf{Visibility}} \\ \hline
    \multicolumn{1}{|l|}{\_match}                                      & \multicolumn{1}{l|}{Match}                                 & \multicolumn{1}{l|}{Private}                                     \\ \hline
    \end{tabular}
\end{table}

\textbf{Description :} This field represent the match of a game.

\begin{table}[H]
    \begin{tabular}{llllll}
    \hline
    \multicolumn{1}{|l|}{\cellcolor[HTML]{EFEFEF}\textbf{Field Name}} & \multicolumn{1}{l|}{\cellcolor[HTML]{EFEFEF}\textbf{Type}} & \multicolumn{1}{l|}{\cellcolor[HTML]{EFEFEF}\textbf{Visibility}} \\ \hline
    \multicolumn{1}{|l|}{\_playerA}                                      & \multicolumn{1}{l|}{Player}                                 & \multicolumn{1}{l|}{Private}                                     \\ \hline
    \end{tabular}
\end{table}

\textbf{Description :} This field represent the first player in the game.

\begin{table}[H]
    \begin{tabular}{llllll}
    \hline
    \multicolumn{1}{|l|}{\cellcolor[HTML]{EFEFEF}\textbf{Field Name}} & \multicolumn{1}{l|}{\cellcolor[HTML]{EFEFEF}\textbf{Type}} & \multicolumn{1}{l|}{\cellcolor[HTML]{EFEFEF}\textbf{Visibility}} \\ \hline
    \multicolumn{1}{|l|}{\_playerB}                                      & \multicolumn{1}{l|}{Player}                                 & \multicolumn{1}{l|}{Private}                                     \\ \hline
    \end{tabular}
\end{table}

\textbf{Description :} This field represent the second player in the game.

\begin{table}[H]
    \begin{tabular}{llllll}
    \hline
    \multicolumn{1}{|l|}{\cellcolor[HTML]{EFEFEF}\textbf{Field Name}} & \multicolumn{1}{l|}{\cellcolor[HTML]{EFEFEF}\textbf{Type}} & \multicolumn{1}{l|}{\cellcolor[HTML]{EFEFEF}\textbf{Visibility}} \\ \hline
    \multicolumn{1}{|l|}{\_tieCounter}                                      & \multicolumn{1}{l|}{int}                                 & \multicolumn{1}{l|}{Private}                                     \\ \hline
    \end{tabular}
\end{table}

\textbf{Description :} This field counts the number of ties in the game.

\begin{table}[H]
    \begin{tabular}{llllll}
    \hline
    \multicolumn{1}{|l|}{\cellcolor[HTML]{EFEFEF}\textbf{Field Name}} & \multicolumn{1}{l|}{\cellcolor[HTML]{EFEFEF}\textbf{Type}} & \multicolumn{1}{l|}{\cellcolor[HTML]{EFEFEF}\textbf{Visibility}} \\ \hline
    \multicolumn{1}{|l|}{\_view}                                      & \multicolumn{1}{l|}{FormMatch}                                 & \multicolumn{1}{l|}{Private}                                     \\ \hline
    \end{tabular}
\end{table}

\textbf{Description :} This field represent the view to paint in a match.

\subsubsection{Methods}

\begin{table}[H]
    \begin{tabular}{|l|l|l|l|}
    \hline
    \rowcolor[HTML]{EFEFEF} 
    \cellcolor[HTML]{EFEFEF}\textbf{Method Name} & \textbf{Parameters}    & \textbf{Returned Type} & \textbf{Visibility} \\ \hline
    Rules                                        & int origin, int target & void                   & Private             \\ \hline
    \end{tabular}
\end{table}

\textbf{Parameters :} Origin and target represent the index of the origin cell and the targeted cell for the path.
\textbf{Description :} This method calls Check(), Checkmate(), Castle() and Stalemate() to check the state of the game if the move is valid.


\begin{table}[H]
    \begin{tabular}{|l|l|l|l|}
    \hline
    \rowcolor[HTML]{EFEFEF} 
    \cellcolor[HTML]{EFEFEF}\textbf{Method Name} & \textbf{Parameters}    & \textbf{Returned Type} & \textbf{Visibility} \\ \hline
    Check                                        & none                   & bool                   & Public              \\ \hline
    \end{tabular}
\end{table}

\textbf{Description :} This method returns true if the match has a check.

\begin{table}[H]
    \begin{tabular}{|l|l|l|l|}
    \hline
    \rowcolor[HTML]{EFEFEF} 
    \cellcolor[HTML]{EFEFEF}\textbf{Method Name} & \textbf{Parameters}    & \textbf{Returned Type} & \textbf{Visibility} \\ \hline
    Checkmate                                    & None                   & bool                   & Private              \\ \hline
    \end{tabular}
\end{table}

\textbf{Description :} This method returns true if the match has a checkmate.

\begin{table}[H]
    \begin{tabular}{|l|l|l|l|}
    \hline
    \rowcolor[HTML]{EFEFEF} 
    \cellcolor[HTML]{EFEFEF}\textbf{Method Name} & \textbf{Parameters}    & \textbf{Returned Type} & \textbf{Visibility} \\ \hline
    Castle                                       & None                   & bool                   & Private              \\ \hline
    \end{tabular}
\end{table}

\textbf{Description :} This method returns true if the match has a Castle.

\begin{table}[H]
    \begin{tabular}{|l|l|l|l|}
    \hline
    \rowcolor[HTML]{EFEFEF} 
    \cellcolor[HTML]{EFEFEF}\textbf{Method Name} & \textbf{Parameters}    & \textbf{Returned Type} & \textbf{Visibility} \\ \hline
    Stalemate                                    & None                   & bool                   & Private              \\ \hline
    \end{tabular}
\end{table}

\textbf{Description :} This method returns true if the match has a stalemate.

\begin{table}[H]
    \begin{tabular}{|l|l|l|l|}
    \hline
    \rowcolor[HTML]{EFEFEF} 
    \cellcolor[HTML]{EFEFEF}\textbf{Method Name} & \textbf{Parameters}    & \textbf{Returned Type} & \textbf{Visibility} \\ \hline
    FiftyTurns                                   & None                   & bool                   & Private              \\ \hline
    \end{tabular}
\end{table}

\textbf{Description :} This method returns true if the match has 50 turns to count.

\begin{table}[H]
    \begin{tabular}{|l|l|l|l|}
    \hline
    \rowcolor[HTML]{EFEFEF} 
    \cellcolor[HTML]{EFEFEF}\textbf{Method Name} & \textbf{Parameters}    & \textbf{Returned Type} & \textbf{Visibility} \\ \hline
    SameBoard                                    & None                   & bool                   & Private              \\ \hline
    \end{tabular}
\end{table}

\textbf{Description :} This method returns true if history list of boards has 3 same boards.

\begin{table}[H]
    \begin{tabular}{|l|l|l|l|}
    \hline
    \rowcolor[HTML]{EFEFEF} 
    \cellcolor[HTML]{EFEFEF}\textbf{Method Name} & \textbf{Parameters}    & \textbf{Returned Type} & \textbf{Visibility} \\ \hline
    Turn                                         & int origin, int target & void                   & Private             \\ \hline
    \end{tabular}
\end{table}

\textbf{Parameters :} Origin and target represent the index of the origin cell and the targeted cell for the path.
\textbf{Description :} Makes the turn, and send it to the view.

\begin{table}[H]
    \begin{tabular}{|l|l|l|l|}
    \hline
    \rowcolor[HTML]{EFEFEF} 
    \cellcolor[HTML]{EFEFEF}\textbf{Method Name} & \textbf{Parameters}    & \textbf{Returned Type} & \textbf{Visibility} \\ \hline
    Selection                                    & int cell               & void                   & Public             \\ \hline
    \end{tabular}
\end{table}

\textbf{Parameters :} Parameeter cell is the index of the cell to be selected.
\textbf{Description :} Selects the cell.

\begin{table}[H]
    \begin{tabular}{|l|l|l|l|}
    \hline
    \rowcolor[HTML]{EFEFEF} 
    \cellcolor[HTML]{EFEFEF}\textbf{Method Name} & \textbf{Parameters}    & \textbf{Returned Type} & \textbf{Visibility} \\ \hline
    Resign                                       & none                   & void                   & Public             \\ \hline
    \end{tabular}
\end{table}

\textbf{Description :} Make the end and exit the game.

\newpage

%PlayerController Class%
%%%%%%%%%%%%%%%%%%%%%%%%

\subsection{Class PlayerController}

\begin{table}[H]
    \begin{tabular}{|l|}
    \hline
    \rowcolor[HTML]{C0C0C0} 
    \textbf{PlayerController}                       \\ \hline
    \rowcolor[HTML]{EFEFEF} 
    -\_main : Chess                               \\ \hline
    \rowcolor[HTML]{EFEFEF} 
    -\_list : List\textless{}{}Player\textgreater{}{} \\ \hline
    \rowcolor[HTML]{FFFFFF} 
    +Add() : void                                 \\ \hline
    \rowcolor[HTML]{FFFFFF} 
    +Remove() : void                              \\ \hline
    \end{tabular}
\end{table}

\subsubsection{Fields}

\begin{table}[H]
    \begin{tabular}{llllll}
    \hline
    \multicolumn{1}{|l|}{\cellcolor[HTML]{EFEFEF}\textbf{Field Name}} & \multicolumn{1}{l|}{\cellcolor[HTML]{EFEFEF}\textbf{Type}} & \multicolumn{1}{l|}{\cellcolor[HTML]{EFEFEF}\textbf{Visibility}} \\ \hline
    \multicolumn{1}{|l|}{\_main}                                      & \multicolumn{1}{l|}{Chess}                                 & \multicolumn{1}{l|}{Private}                                     \\ \hline
    \end{tabular}
\end{table}

\textbf{Description :} This field represent the chess which the game controller will talk to.

\begin{table}[H]
    \begin{tabular}{llllll}
    \hline
    \multicolumn{1}{|l|}{\cellcolor[HTML]{EFEFEF}\textbf{Field Name}} & \multicolumn{1}{l|}{\cellcolor[HTML]{EFEFEF}\textbf{Type}} & \multicolumn{1}{l|}{\cellcolor[HTML]{EFEFEF}\textbf{Visibility}} \\ \hline
    \multicolumn{1}{|l|}{\_list}                                      & \multicolumn{1}{l|}{List\textless{}Player\textgreater{}}                                 & \multicolumn{1}{l|}{Private}                                     \\ \hline
    \end{tabular}
\end{table}

\textbf{Description :} This field represent the list of all players.

\subsubsection{Methods}

\begin{table}[H]
    \begin{tabular}{|l|l|l|l|}
    \hline
    \rowcolor[HTML]{EFEFEF} 
    \cellcolor[HTML]{EFEFEF}\textbf{Method Name} & \textbf{Parameters}    & \textbf{Returned Type} & \textbf{Visibility} \\ \hline
    Add                                          & none                   & void                   & Public             \\ \hline
    \end{tabular}
\end{table}

\textbf{Description :} This method adds the player to the list.

\begin{table}[H]
    \begin{tabular}{|l|l|l|l|}
    \hline
    \rowcolor[HTML]{EFEFEF} 
    \cellcolor[HTML]{EFEFEF}\textbf{Method Name} & \textbf{Parameters}    & \textbf{Returned Type} & \textbf{Visibility} \\ \hline
    Remove                                       & none                   & void                   & Public             \\ \hline
    \end{tabular}
\end{table}

\textbf{Description :} This method removes the player to the list.

\newpage

%NEW SECTION VIEWS%
%%%%%%%%%%%%%%%%%%%

\section{Views}

%FormSelection Class%
%%%%%%%%%%%%%%%%%%%%%

\subsection{Class FormSelection}

\begin{table}[H]
    \begin{tabular}{|l|}
    \hline
    \rowcolor[HTML]{C0C0C0} 
    \textbf{FormSelection}    \\ \hline
    \rowcolor[HTML]{EFEFEF} 
    -\_controller : Chess      \\ \hline
    \rowcolor[HTML]{FFFFFF} 
    +OpenLeaderboard() : void \\ \hline
    \rowcolor[HTML]{FFFFFF} 
    +Start() : Player{[}2{]}  \\ \hline
    \rowcolor[HTML]{FFFFFF} 
    +Cancel() : void          \\ \hline
    \end{tabular}
\end{table}



\newpage

%FormMenu Class%
%%%%%%%%%%%%%%%%

\subsection{Class FormMenu}

\begin{table}[H]
    \begin{tabular}{|l|}
    \hline
    \rowcolor[HTML]{C0C0C0} 
    \textbf{FormMenu}                                   \\ \hline
    \rowcolor[HTML]{EFEFEF} 
    -\_main : Chess                                          \\ \hline
    \rowcolor[HTML]{FFFFFF} 
    +Start(object sender, System.EventArgs e) : void         \\ \hline
    \rowcolor[HTML]{FFFFFF} 
    +Exit(object sender, System.EventArgs e) : void          \\ \hline
    \rowcolor[HTML]{FFFFFF} 
    +ManagePlayers(object sender, System.EventArgs e) : void \\ \hline
    \end{tabular}
\end{table}

\newpage

%FormPromotion Class%
%%%%%%%%%%%%%%%%%%%%%

\subsection{Class FormPromotion}

\begin{table}[H]
    \begin{tabular}{|l|}
    \hline
    \rowcolor[HTML]{C0C0C0} 
    \textbf{FormPromotion}                            \\ \hline
    \rowcolor[HTML]{EFEFEF} 
    -\_controller : GameController                    \\ \hline
    \rowcolor[HTML]{FFFFFF} 
    +Submit(object sender, System.EventArgs e) : void \\ \hline
    \end{tabular}
\end{table}



\newpage

%FormMatch Class%
%%%%%%%%%%%%%%%%%

\subsection{Class FormMatch}

\begin{table}[H]
    \begin{tabular}{|l|}
    \hline
    \rowcolor[HTML]{C0C0C0} 
    \textbf{FormMatch}                                   \\ \hline
    \rowcolor[HTML]{EFEFEF} 
    -\_controller : GameController                       \\ \hline
    \rowcolor[HTML]{FFFFFF} 
    +GridClick(object sender, System.EventArgs e) : void \\ \hline
    +DrawBoard(string board) : void                      \\ \hline
    +DrawSelection(int cell) : void                      \\ \hline
    +ShowMessage(string message) : void                  \\ \hline
    +VictoryMessage() : void                             \\ \hline
    \end{tabular}
\end{table}

\newpage

%FormLeaderboard Class%
%%%%%%%%%%%%%%%%%%%%%%%

\subsection{Class FormLeaderboard}

\begin{table}[H]
    \begin{tabular}{|l|}
    \hline
    \rowcolor[HTML]{C0C0C0} 
    \textbf{FormLeaderboard}                                    \\ \hline
    \rowcolor[HTML]{EFEFEF} 
    -\_controller : PlayerController                      \\ \hline
    \rowcolor[HTML]{FFFFFF} 
    +ShowList(List\textless{}{}Player\textgreater{}{}) : void \\ \hline
    +Add() : void                                         \\ \hline
    +Remove() : void                                      \\ \hline
    +Back() : void                                        \\ \hline
    \end{tabular}
\end{table}

\newpage

\end{document}