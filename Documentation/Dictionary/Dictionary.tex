\documentclass[12pt]{article}
\usepackage[margin=0.75in]{geometry}
\usepackage{float}
\usepackage[table,xcdraw]{xcolor}
\usepackage{graphicx}

\graphicspath{ {./images/} }

\title{Class Dictionary}
\author{Jean-Philippe Miguel-Gagnon, Jérémy Gaouette, Raphaël Rail}

\date{Thursday, 31th of march 2022}

\begin{document}

\begin{titlepage}
\maketitle
\includegraphics[width=\textwidth]{CHESS}
\begin{center}Presented to : Charles Jacob\end{center}
\end{titlepage}

\tableofcontents

\newpage

\section{Introduction}

The goal of this document is to inform the programmer about
classes used to create a C\#\ OOP Chess game.
\\

We'll go through this with the MVC model approch to make it
clearer for the programmer where to implement his code. 

\newpage

\section{Models}
%%%%%%%%%%%%%
%Piece Class%

\subsection{Class Piece}

    An abstract class representing a chess piece.
\begin{table}[H]
    \begin{tabular}{|l|}
    \hline
    \cellcolor[HTML]{C0C0C0}\textbf{(Abstract) Piece}  \\ \hline
    \cellcolor[HTML]{EFEFEF}-\_colour : Colour         \\ \hline
    +CanCollide() : bool                               \\ \hline
    +ValidMove(int x1, int y1, int x2, int y2) : bool  \\ \hline
    +ToString() : string                               \\ \hline
    +CanPromote() : bool                               \\ \hline
    +IsEssential() : bool                              \\ \hline
    \end{tabular}
\end{table}

\subsubsection{Fields}

\begin{table}[H]
    \begin{tabular}{llllll}
    \hline
    \multicolumn{1}{|l|}{\cellcolor[HTML]{EFEFEF}\textbf{Field Name}} & \multicolumn{1}{l|}{\cellcolor[HTML]{EFEFEF}\textbf{Type}} & \multicolumn{1}{l|}{\cellcolor[HTML]{EFEFEF}\textbf{Visibility}} \\ \hline
    \multicolumn{1}{|l|}{\_colour}                                    & \multicolumn{1}{l|}{Colour}                                & \multicolumn{1}{l|}{Private}                                     \\ \hline
    \end{tabular}
\end{table}

    \textbf{Description :} Represents the colour of the piece, which is either white or black.

\subsubsection{Methods}

\begin{table}[H]
    \begin{tabular}{|l|l|l|l|}
    \hline
    \rowcolor[HTML]{EFEFEF} 
    \cellcolor[HTML]{EFEFEF}\textbf{Method Name} & \textbf{Parameters}  & \textbf{Returned Type} & \textbf{Visibility} \\ \hline
    ValidMove                          & x1, y1, x2, y2 : int & bool                   & Public              \\ \hline
    \end{tabular}
\end{table}

    \textbf{Parameters :} x1 and y1 are the coordinates of the piece.
    x2 and y2 are the coordinates of the target location.
\\
    \textbf{Description :} This method is used to check if the piece can move from position (x1, y1) to position (x2, y2).

\begin{table}[H]
    \begin{tabular}{|l|l|l|l|}
    \hline
    \rowcolor[HTML]{EFEFEF} 
    \cellcolor[HTML]{EFEFEF}\textbf{Method Name} & \textbf{Parameters}  & \textbf{Returned Type} & \textbf{Visibility} \\ \hline
    CanCollide                                   & None                 & bool                   & Public              \\ \hline
    \end{tabular}
\end{table}

    \textbf{Description :} By default, this method returns true, meaning that the piece can collide with other pieces.

\begin{table}[H]
    \begin{tabular}{|l|l|l|l|}
    \hline
    \rowcolor[HTML]{EFEFEF} 
    \cellcolor[HTML]{EFEFEF}\textbf{Method Name} & \textbf{Parameters}  & \textbf{Returned Type} & \textbf{Visibility} \\ \hline
    ToString                                   & None                 & String                   & Public              \\ \hline
    \end{tabular}
\end{table}

    \textbf{Description :} This method returns the string representation of the piece.
    This method gets overridden by the subclasses.

\begin{table}[H]
    \begin{tabular}{|l|l|l|l|}
    \hline
    \rowcolor[HTML]{EFEFEF} 
    \cellcolor[HTML]{EFEFEF}\textbf{Method Name} & \textbf{Parameters}  & \textbf{Returned Type} & \textbf{Visibility} \\ \hline
    CanPromote                                   & None                 & bool                   & Public              \\ \hline
    \end{tabular}
\end{table}

    \textbf{Description :} By default, this method returns false, meaning that the piece cannot be promoted.

\begin{table}[H]
    \begin{tabular}{|l|l|l|l|}
    \hline
    \rowcolor[HTML]{EFEFEF} 
    \cellcolor[HTML]{EFEFEF}\textbf{Method Name} & \textbf{Parameters}  & \textbf{Returned Type} & \textbf{Visibility} \\ \hline
    IsEssential                                  & None                 & bool                   & Public              \\ \hline
    \end{tabular}
\end{table}

    \textbf{Description :} By default, this method returns false, meaning that the piece is not essential.

\subsubsection{Properties}

\begin{table}[H]
    \begin{tabular}{|l|l|l|l|}
    \hline
    \rowcolor[HTML]{EFEFEF} 
    \cellcolor[HTML]{EFEFEF}\textbf{Property Name} & \textbf{Parameters}  & \textbf{Returned Type} & \textbf{Visibility} \\ \hline
    Colour                                         & None                 & Colour                 & Public              \\ \hline
    \end{tabular}
\end{table}

    \textbf{Description :} Gets the colour of the piece.
\newpage

%%%%%%%%%%%%%%%
%StartingPiece%

\subsection{Class StartingPiece : Piece}

    Extended from the Piece class, the StartingPiece class represents a piece that keeps track of whether it has been moved or not.

\begin{table}[H]
    \begin{tabular}{|l|}
    \hline
    \cellcolor[HTML]{C0C0C0}\textbf{(Abstract) StartingPiece}            \\ \hline
    \cellcolor[HTML]{EFEFEF}-\_hasMoved : bool        \\ \hline
    \end{tabular}
\end{table}

\subsubsection{Fields}

\begin{table}[H]
    \begin{tabular}{llllll}
    \hline
    \multicolumn{1}{|l|}{\cellcolor[HTML]{EFEFEF}\textbf{Field Name}} & \multicolumn{1}{l|}{\cellcolor[HTML]{EFEFEF}\textbf{Type}} & \multicolumn{1}{l|}{\cellcolor[HTML]{EFEFEF}\textbf{Visibility}} \\ \hline
    \multicolumn{1}{|l|}{\_hasMoved}                                  & \multicolumn{1}{l|}{bool}                                & \multicolumn{1}{l|}{Private}                                     \\ \hline
    \end{tabular}
\end{table}

\textbf{Description :} Represents whether the piece has been moved or not.

\subsubsection{Methods}
    Inherited from the Piece class.
\subsubsection{Properties}

\begin{table}[H]
    \begin{tabular}{|l|l|l|l|}
    \hline
    \rowcolor[HTML]{EFEFEF} 
    \cellcolor[HTML]{EFEFEF}\textbf{Property Name} & \textbf{Parameters}  & \textbf{Returned Type} & \textbf{Visibility} \\ \hline
    HasMoved                                       & None                 & bool                   & Public              \\ \hline
    \end{tabular}
\end{table}

    \textbf{Description :} Gets or Sets whether the piece has been moved or not.
\newpage

%%%%%%%%%%%%
%Pawn Class%

\subsection{Class Pawn : StartingPiece}

    Extended from the StartingPiece class, the Pawn class represents a pawn.
\begin{table}[H]
    \begin{tabular}{|l|}
    \hline
    \cellcolor[HTML]{C0C0C0}\textbf{Pawn} \\ \hline
    \cellcolor[HTML]{EFEFEF}                    \\ \hline
    +ValidMove(x1, y1, x2, y2) : bool           \\ \hline
    +CanPromote() : bool                        \\ \hline
    +ToString() : string                        \\ \hline
    \end{tabular}
\end{table}

\subsubsection{Fields}

    Inherited from the StartingPiece class.

\subsubsection{Methods}

\begin{table}[H]
    \begin{tabular}{|l|l|l|l|}
    \hline
    \rowcolor[HTML]{EFEFEF} 
    \cellcolor[HTML]{EFEFEF}\textbf{Method Name} & \textbf{Parameters}  & \textbf{Returned Type} & \textbf{Visibility} \\ \hline
    ValidMove                          & x1, y1, x2, y2 : int & bool                   & Public              \\ \hline
    \end{tabular}
\end{table}

    \textbf{Parameters :} x1 and y1 are the coordinates of the piece's current position.
    x2 and y2 are the coordinates of the destination.
\\
    \textbf{Description :} This method checks if the move is valid for the pawn. 
    Valid moves are forward by one, forward by two if the pawn has not moved, and forward diagonally if the pawn is attacking.
    In these cases, the method returns true.
    Otherwise, the method returns false.

\begin{table}[H]
    \begin{tabular}{|l|l|l|l|}
    \hline
    \rowcolor[HTML]{EFEFEF} 
    \cellcolor[HTML]{EFEFEF}\textbf{Method Name} & \textbf{Parameters}  & \textbf{Returned Type} & \textbf{Visibility} \\ \hline
    CanPromote                                   & None                 & bool                   & Public              \\ \hline
    \end{tabular}
\end{table}

    \textbf{Description :} Overrides the CanPromote method in the Piece class.
    Returns true. 

\begin{table}[H]
    \begin{tabular}{|l|l|l|l|}
    \hline
    \rowcolor[HTML]{EFEFEF} 
    \cellcolor[HTML]{EFEFEF}\textbf{Method Name} & \textbf{Parameters}  & \textbf{Returned Type} & \textbf{Visibility} \\ \hline
    ToString                                   & None                 & String                   & Public              \\ \hline
    \end{tabular}
\end{table}

    \textbf{Description :} Overrides the ToString method in the Piece class.
    Returns the string "p" if the piece is white, and "P" if the piece is black. 

\subsubsection{Properties}

    Inherited from the StartingPiece class.
\newpage

%%%%%%%%%%%%
%Rook Class%

\subsection{Class Rook : StartingPiece}

    Extended from the StartingPiece class, the Rook class represents a rook.
\begin{table}[H]
    \begin{tabular}{|l|}
    \hline
    \cellcolor[HTML]{C0C0C0}\textbf{Rook} \\ \hline
    \cellcolor[HTML]{EFEFEF}                    \\ \hline
    +ValidMove(x1, y1, x2, y2) : bool           \\ \hline
    +ToString() : string                        \\ \hline
    \end{tabular}
\end{table}

\subsubsection{Fields}

    Inherited from the StartingPiece class.

\subsubsection{Methods}

\begin{table}[H]
    \begin{tabular}{|l|l|l|l|}
    \hline
    \rowcolor[HTML]{EFEFEF} 
    \cellcolor[HTML]{EFEFEF}\textbf{Method Name} & \textbf{Parameters}  & \textbf{Returned Type} & \textbf{Visibility} \\ \hline
    ValidMove                          & x1, y1, x2, y2 : int & bool                   & Public              \\ \hline
    \end{tabular}
\end{table}

    \textbf{Parameters :} x1 and y1 are the coordinates of the piece's current position. x2 and y2 are the coordinates of the destination. 
\\
    \textbf{Description :} This method checks if the move is valid for the rook. 
    Valid moves are horizontal and vertical directions.
    In these cases, the method returns true.
    Otherwise, the method returns false.
\begin{table}[H]
    \begin{tabular}{|l|l|l|l|}
    \hline
    \rowcolor[HTML]{EFEFEF} 
    \cellcolor[HTML]{EFEFEF}\textbf{Method Name} & \textbf{Parameters}  & \textbf{Returned Type} & \textbf{Visibility} \\ \hline
    ToString                                   & None                 & String                   & Public              \\ \hline
    \end{tabular}
\end{table}

    \textbf{Description :} Overrides the ToString method in the Piece class.
    Returns the string "r" if the piece is white, and "R" if the piece is black. 

\subsubsection{Properties}

    Inherited from the StartingPiece class.
\newpage

%%%%%%%%%%%%
%King Class%

\subsection{Class King : StartingPiece}

    Extended from the StartingPiece class, the King class represents a king.
\begin{table}[H]
    \begin{tabular}{|l|}
    \hline
    \cellcolor[HTML]{C0C0C0}\textbf{King} \\ \hline
    \cellcolor[HTML]{EFEFEF}                    \\ \hline
    +ValidMove(x1, y1, x2, y2) : bool           \\ \hline
    +IsEssential() : bool                       \\ \hline
    +ToString() : string                        \\ \hline
    \end{tabular}
\end{table}

\subsubsection{Fields}

    Inherited from the StartingPiece class.

\subsubsection{Methods}

\begin{table}[H]
    \begin{tabular}{|l|l|l|l|}
    \hline
    \rowcolor[HTML]{EFEFEF} 
    \cellcolor[HTML]{EFEFEF}\textbf{Method Name} & \textbf{Parameters}  & \textbf{Returned Type} & \textbf{Visibility} \\ \hline
    ValidMove                          & x1, y1, x2, y2 : int & bool                   & Public              \\ \hline
    \end{tabular}
\end{table}

    \textbf{Parameters :} x1 and y1 are the coordinates of the piece's current position.
    x2 and y2 are the coordinates of the destination.  
\\
    \textbf{Description :} This method checks if the move is valid for the king. 
    Valid moves are horizontally, vertically and diagonally by one square.
    Additionally, the king can move by two squares on the left and right if it has not moved, called castling.
    In these cases, the method returns true.
    Otherwise, the method returns false.

\begin{table}[H]
    \begin{tabular}{|l|l|l|l|}
    \hline
    \rowcolor[HTML]{EFEFEF} 
    \cellcolor[HTML]{EFEFEF}\textbf{Method Name} & \textbf{Parameters}  & \textbf{Returned Type} & \textbf{Visibility} \\ \hline
    IsEssential                                  & None                 & bool                   & Public              \\ \hline
    \end{tabular}
\end{table}

    \textbf{Description :} Overrides the IsEssential method in the Piece class.
    Returns true.

\begin{table}[H]
    \begin{tabular}{|l|l|l|l|}
    \hline
    \rowcolor[HTML]{EFEFEF} 
    \cellcolor[HTML]{EFEFEF}\textbf{Method Name} & \textbf{Parameters}  & \textbf{Returned Type} & \textbf{Visibility} \\ \hline
    ToString                                   & None                 & String                   & Public              \\ \hline
    \end{tabular}
\end{table}

    \textbf{Description :} Overrides the ToString method in the Piece class.
    Returns the string "k" if the piece is white, and "K" if the piece is black. 

\subsubsection{Properties}

    Inherited from the StartingPiece class.
\newpage

%%%%%%%%%%%%%%
%Knight Class%

\subsection{Class Knight : Piece}

    Extended from the Piece class, the Knight class represents a knight.
\begin{table}[H]
    \begin{tabular}{|l|}
    \hline
    \cellcolor[HTML]{C0C0C0}\textbf{Knight} \\ \hline
    \cellcolor[HTML]{EFEFEF}                    \\ \hline
    +ValidMove(x1, y1, x2, y2) : bool           \\ \hline
    +CanCollide() : bool                        \\ \hline
    +ToString() : string                        \\ \hline
    \end{tabular}
\end{table}

\subsubsection{Fields}

    Inherited from the Piece class.

\subsubsection{Methods}

\begin{table}[H]
    \begin{tabular}{|l|l|l|l|}
    \hline
    \rowcolor[HTML]{EFEFEF} 
    \cellcolor[HTML]{EFEFEF}\textbf{Method Name} & \textbf{Parameters}  & \textbf{Returned Type} & \textbf{Visibility} \\ \hline
    ValidMove                          & x1, y1, x2, y2 : int & bool                   & Public              \\ \hline
    \end{tabular}
\end{table}

    \textbf{Parameters :} x1 and y1 are the coordinates of the piece's current position.
    x2 and y2 are the coordinates of the destination.  
\\
    \textbf{Description :} This method checks if the move is valid for the knight. 
    Valid moves are moves in an L shape.
    An L shape is composed of two forward moves and one move to the left or right.
    Alternatively, it can be composed of two moves to the left or right and one forward move.
    The method returns true if the move is valid.
    Otherwise, the method returns false.

\begin{table}[H]
    \begin{tabular}{|l|l|l|l|}
    \hline
    \rowcolor[HTML]{EFEFEF} 
    \cellcolor[HTML]{EFEFEF}\textbf{Method Name} & \textbf{Parameters}  & \textbf{Returned Type} & \textbf{Visibility} \\ \hline
    CanCollide                                   & None                 & bool                   & Public              \\ \hline
    \end{tabular}
\end{table}

    \textbf{Description :} Overrides the CanCollide method in the Piece class.
    Returns false.

\begin{table}[H]
    \begin{tabular}{|l|l|l|l|}
    \hline
    \rowcolor[HTML]{EFEFEF} 
    \cellcolor[HTML]{EFEFEF}\textbf{Method Name} & \textbf{Parameters}  & \textbf{Returned Type} & \textbf{Visibility} \\ \hline
    ToString                                   & None                 & String                   & Public              \\ \hline
    \end{tabular}
\end{table}

    \textbf{Description :} Overrides the ToString method in the Piece class.
    Returns the string "n" if the piece is white, and "N" if the piece is black. 

\subsubsection{Properties}

    Inherited from the StartingPiece class.

\newpage

%Bishop Class%
%%%%%%%%%%%%%%

\subsection{Class Bishop : Piece}

    Extended from the Piece class, the Bishop class represents a bishop.
\begin{table}[H]
    \begin{tabular}{|l|}
    \hline
    \cellcolor[HTML]{C0C0C0}\textbf{Bishop} \\ \hline
    \cellcolor[HTML]{EFEFEF}                    \\ \hline
    +ValidMove(x1, y1, x2, y2) : bool           \\ \hline
    +ToString() : string                        \\ \hline
    \end{tabular}
\end{table}

\subsubsection{Fields}

    Inherited from the Piece class.

\subsubsection{Methods}

\begin{table}[H]
    \begin{tabular}{|l|l|l|l|}
    \hline
    \rowcolor[HTML]{EFEFEF} 
    \cellcolor[HTML]{EFEFEF}\textbf{Method Name} & \textbf{Parameters}  & \textbf{Returned Type} & \textbf{Visibility} \\ \hline
    ValidMove                          & x1, y1, x2, y2 : int & bool                   & Public              \\ \hline
    \end{tabular}
\end{table}

    \textbf{Parameters :} x1 and y1 are the coordinates of the piece's current position.
    x2 and y2 are the coordinates of the destination.  
\\
    \textbf{Description :} This method checks if the move is valid for the bishop. 
    Valid moves are moves in a diagonal direction.
    The method returns true if the move is valid.
    Otherwise, the method returns false.
    
\begin{table}[H]
    \begin{tabular}{|l|l|l|l|}
    \hline
    \rowcolor[HTML]{EFEFEF} 
    \cellcolor[HTML]{EFEFEF}\textbf{Method Name} & \textbf{Parameters}  & \textbf{Returned Type} & \textbf{Visibility} \\ \hline
    ToString                                   & None                 & String                   & Public              \\ \hline
    \end{tabular}
\end{table}

    \textbf{Description :} Overrides the ToString method in the Piece class.
    Returns the string "b" if the piece is white, and "B" if the piece is black. 

\subsubsection{Properties}

    Inherited from the StartingPiece class.

\newpage

%Queen%
%%%%%%%

\subsection{Class Queen : Piece}

    Extended from the Piece class, the Queen class represents a queen.
\begin{table}[H]
    \begin{tabular}{|l|}
    \hline
    \cellcolor[HTML]{C0C0C0}\textbf{Queen} \\ \hline
    \cellcolor[HTML]{EFEFEF}                    \\ \hline
    +ValidMove(x1, y1, x2, y2) : bool           \\ \hline
    +ToString() : string                        \\ \hline
    \end{tabular}
\end{table}

\subsubsection{Fields}

    Inherited from the Piece class.

\subsubsection{Methods}

\begin{table}[H]
    \begin{tabular}{|l|l|l|l|}
    \hline
    \rowcolor[HTML]{EFEFEF} 
    \cellcolor[HTML]{EFEFEF}\textbf{Method Name} & \textbf{Parameters}  & \textbf{Returned Type} & \textbf{Visibility} \\ \hline
    ValidMove                          & x1, y1, x2, y2 : int & bool                   & Public              \\ \hline
    \end{tabular}
\end{table}

    \textbf{Parameters :} x1 and y1 are the coordinates of the piece's current position.
    x2 and y2 are the coordinates of the destination.  
\\
    \textbf{Description :} This method checks if the move is valid for the queen. 
    Valid moves are moves in a horizontal, vertical, or diagonal direction.
    The method returns true if the move is valid.
    Otherwise, the method returns false.

\begin{table}[H]
    \begin{tabular}{|l|l|l|l|}
    \hline
    \rowcolor[HTML]{EFEFEF} 
    \cellcolor[HTML]{EFEFEF}\textbf{Method Name} & \textbf{Parameters}  & \textbf{Returned Type} & \textbf{Visibility} \\ \hline
    ToString                                   & None                 & String                   & Public              \\ \hline
    \end{tabular}
\end{table}

    \textbf{Description :} Overrides the ToString method in the Piece class.
    Returns the string "q" if the piece is white, and "Q" if the piece is black. 

\subsubsection{Properties}

    Inherited from the Piece class.

\newpage

%Match Class%
%%%%%%%%%%%%%

\subsection{Class Match}

This Class represents the model of a chess match wich compose the
GameController. It will keep all changes that the game controller will return.

\begin{table}[H]
    \begin{tabular}{|l|}
    \hline
    \rowcolor[HTML]{C0C0C0} 
    \textbf{Match}                                    \\ \hline
    \rowcolor[HTML]{EFEFEF} 
    -\_board : Board                                  \\ \hline
    \rowcolor[HTML]{EFEFEF} 
    -\_current : Colour                               \\ \hline
    \rowcolor[HTML]{EFEFEF} 
    -\_history : string{[}{]}                         \\ \hline
    \rowcolor[HTML]{EFEFEF} 
    -\_turnNumber : int                               \\ \hline
    +ExportBoard() : string                           \\ \hline
    +ExportHistory() : string[]                       \\ \hline
    +ValidTurn(int origin, int target) : bool         \\ \hline
    +MakeTurn(int origin, int target) : void          \\ \hline
    +ValidSelection(int cell, bool firstClick) : bool \\ \hline
    +HasPromotable(int target) : bool                 \\ \hline
    +Check() : bool                                   \\ \hline
    +Checkmate() : bool                               \\ \hline
    +Stalemate() : bool                               \\ \hline
    +Castle(int origin, int target) : void            \\ \hline
    +GetAssailants(Colour) : List<int>                \\ \hline
\end{tabular}
\end{table}

\subsubsection{Fields}

\begin{table}[H]
    \begin{tabular}{llllll}
    \hline
    \multicolumn{1}{|l|}{\cellcolor[HTML]{EFEFEF}\textbf{Field Name}} & \multicolumn{1}{l|}{\cellcolor[HTML]{EFEFEF}\textbf{Type}} & \multicolumn{1}{l|}{\cellcolor[HTML]{EFEFEF}\textbf{Visibility}} \\ \hline
    \multicolumn{1}{|l|}{\_board}                                     & \multicolumn{1}{l|}{Board}                                 & \multicolumn{1}{l|}{Private}                                     \\ \hline
    \end{tabular}
\end{table}

\textbf{Description :} This field represent the board of a match wich compose
the match. Its type is Board wich is the next class to discuss.

\begin{table}[H]
    \begin{tabular}{llllll}
    \hline
    \multicolumn{1}{|l|}{\cellcolor[HTML]{EFEFEF}\textbf{Field Name}} & \multicolumn{1}{l|}{\cellcolor[HTML]{EFEFEF}\textbf{Type}} & \multicolumn{1}{l|}{\cellcolor[HTML]{EFEFEF}\textbf{Visibility}} \\ \hline
    \multicolumn{1}{|l|}{\_current}                                     & \multicolumn{1}{l|}{Colour}                                 & \multicolumn{1}{l|}{Private}                                     \\ \hline
    \end{tabular}
\end{table}

\textbf{Description :} This field tells us wich piece colour is
currently playing, white or black.

\begin{table}[H]
    \begin{tabular}{llllll}
    \hline
    \multicolumn{1}{|l|}{\cellcolor[HTML]{EFEFEF}\textbf{Field Name}} & \multicolumn{1}{l|}{\cellcolor[HTML]{EFEFEF}\textbf{Type}} & \multicolumn{1}{l|}{\cellcolor[HTML]{EFEFEF}\textbf{Visibility}} \\ \hline
    \multicolumn{1}{|l|}{\_history}                                     & \multicolumn{1}{l|}{string[]}                            & \multicolumn{1}{l|}{Private}                                     \\ \hline
    \end{tabular}
\end{table}

\textbf{Description :} This field is a table that contains strings that
represent previous board states to keep track of what has been played. For exemple,
the first string would look like that :
\\"RNBKQBNRPPPPPPPP.................................pppppppprnkqbnr". 

\begin{table}[H]
    \begin{tabular}{llllll}
    \hline
    \multicolumn{1}{|l|}{\cellcolor[HTML]{EFEFEF}\textbf{Field Name}} & \multicolumn{1}{l|}{\cellcolor[HTML]{EFEFEF}\textbf{Type}} & \multicolumn{1}{l|}{\cellcolor[HTML]{EFEFEF}\textbf{Visibility}} \\ \hline
    \multicolumn{1}{|l|}{\_turnNumber}                                & \multicolumn{1}{l|}{int}                                   & \multicolumn{1}{l|}{Private}                                     \\ \hline
    \end{tabular}
\end{table}

\textbf{Description :} This field tracks the number of turns that have been played.

\subsubsection{Methods}

\begin{table}[H]
    \begin{tabular}{|l|l|l|l|}
    \hline
    \rowcolor[HTML]{EFEFEF} 
    \cellcolor[HTML]{EFEFEF}\textbf{Method Name} & \textbf{Parameters}  & \textbf{Returned Type} & \textbf{Visibility} \\ \hline
    ExportBoard                                  & none                 & string                 & Public              \\ \hline
    \end{tabular}
\end{table}

\textbf{Description :} The method returns the the board as a string of 64 char.

\begin{table}[H]
    \begin{tabular}{|l|l|l|l|}
    \hline
    \rowcolor[HTML]{EFEFEF} 
    \cellcolor[HTML]{EFEFEF}\textbf{Method Name} & \textbf{Parameters}  & \textbf{Returned Type} & \textbf{Visibility} \\ \hline
    ExportHistory                                & none                 & string[]                 & Public            \\ \hline
    \end{tabular}
\end{table}

\textbf{Description :} The method returns the table \_history.

\begin{table}[H]
    \begin{tabular}{|l|l|l|l|}
    \hline
    \rowcolor[HTML]{EFEFEF} 
    \cellcolor[HTML]{EFEFEF}\textbf{Method Name} & \textbf{Parameters}    & \textbf{Returned Type} & \textbf{Visibility} \\ \hline
    ValidTurn                                    & int origin, int target & bool                   & Public              \\ \hline
    \end{tabular}
\end{table}

\textbf{Parameters :} Parameter origin represents the cell
where the piece is before the move and target is the targeted cell.
\\

\textbf{Description :} The method returns true if the turn is valid.

\begin{table}[H]
    \begin{tabular}{|l|l|l|l|}
    \hline
    \rowcolor[HTML]{EFEFEF} 
    \cellcolor[HTML]{EFEFEF}\textbf{Method Name} & \textbf{Parameters}    & \textbf{Returned Type} & \textbf{Visibility} \\ \hline
    MakeTurn                                     & int origin, int target & void                   & Public              \\ \hline
    \end{tabular}
\end{table}

\textbf{Parameters :} Parameter origin represents the cell
where the piece is before the move and target is the targeted cell.
\\

\textbf{Description :} The method makes all changes to the board in order to make the turn.

\begin{table}[H]
    \begin{tabular}{|l|l|l|l|}
    \hline
    \rowcolor[HTML]{EFEFEF} 
    \cellcolor[HTML]{EFEFEF}\textbf{Method Name} & \textbf{Parameters}       & \textbf{Returned Type} & \textbf{Visibility} \\ \hline
    ValidSelection                               & int cell, bool firstClick & bool                   & Public              \\ \hline
    \end{tabular}
\end{table}

\textbf{Parameters :} Parameter cell represents the cell
where the player clicks and the parameter firstClick is a bool
that returns true if its the first click.
\\

\textbf{Description :} The method checks if the selection made
in the board is valid.

\begin{table}[H]
    \begin{tabular}{|l|l|l|l|}
    \hline
    \rowcolor[HTML]{EFEFEF} 
    \cellcolor[HTML]{EFEFEF}\textbf{Method Name} & \textbf{Parameters}     & \textbf{Returned Type} & \textbf{Visibility} \\ \hline
    HasPromotable                                & int target                    & bool                   & Public              \\ \hline
    \end{tabular}
\end{table}


\textbf{Parameters :} The parameter target represents the targeted cell for a piece move.
\textbf{Description :} This method returns true if the cell contains a promotable piece.

\begin{table}[H]
    \begin{tabular}{|l|l|l|l|}
    \hline
    \rowcolor[HTML]{EFEFEF} 
    \cellcolor[HTML]{EFEFEF}\textbf{Method Name} & \textbf{Parameters}     & \textbf{Returned Type} & \textbf{Visibility} \\ \hline
    Check                                        & none                    & bool                   & Public              \\ \hline
    \end{tabular}
\end{table}

\textbf{Description :} This method returns true if the current turn makes a check.

\begin{table}[H]
    \begin{tabular}{|l|l|l|l|}
    \hline
    \rowcolor[HTML]{EFEFEF} 
    \cellcolor[HTML]{EFEFEF}\textbf{Method Name} & \textbf{Parameters}     & \textbf{Returned Type} & \textbf{Visibility} \\ \hline
    Checkmate                                        & none                    & bool                   & Public              \\ \hline
    \end{tabular}
\end{table}

\textbf{Description :} This method returns true if the current turn makes a checkmate.

\begin{table}[H]
    \begin{tabular}{|l|l|l|l|}
    \hline
    \rowcolor[HTML]{EFEFEF} 
    \cellcolor[HTML]{EFEFEF}\textbf{Method Name} & \textbf{Parameters}     & \textbf{Returned Type} & \textbf{Visibility} \\ \hline
    Stalemate                                    & none                    & bool                   & Public              \\ \hline
    \end{tabular}
\end{table}

\textbf{Description :} This method returns true if the current turn makes a Stalemate.

\begin{table}[H]
    \begin{tabular}{|l|l|l|l|}
    \hline
    \rowcolor[HTML]{EFEFEF} 
    \cellcolor[HTML]{EFEFEF}\textbf{Method Name} & \textbf{Parameters}      & \textbf{Returned Type} & \textbf{Visibility} \\ \hline
    Castle                                       & int origin, int target   & bool                   & Public              \\ \hline
    \end{tabular}
\end{table}

\textbf{Description :} This method returns true if the move will make a castle.

\begin{table}[H]
    \begin{tabular}{|l|l|l|l|}
    \hline
    \rowcolor[HTML]{EFEFEF} 
    \cellcolor[HTML]{EFEFEF}\textbf{Method Name} & \textbf{Parameters}     & \textbf{Returned Type} & \textbf{Visibility} \\ \hline
    GetAsaillants                                & Colour colour           & List<int>                   & Public              \\ \hline
    \end{tabular}
\end{table}

\textbf{Parameters :} The parameter colour represents the piece colour specified.
\textbf{Description :} This method returns a list of all possible asaillants by the index of their cell.

\subsubsection{Properties}

\begin{table}[H]
    \begin{tabular}{|l|l|l|l|}
    \hline
    \rowcolor[HTML]{EFEFEF} 
    \cellcolor[HTML]{EFEFEF}\textbf{Property Name} & \textbf{Parameters}  & \textbf{Returned Type} & \textbf{Visibility} \\ \hline
    Current                                        & None                 & Colour                 & Public              \\ \hline
    \end{tabular}
\end{table}

\textbf{Description :} Gets or Sets the property \_current of a match (Colour white or black).

\newpage

%Board Class%
%%%%%%%%%%%%%

\subsection{Class Board}

This Class represents the model of a chess board wich compose the
match. It will return all changes to the match.

\begin{table}[H]
    \begin{tabular}{|l|}
    \hline
    \rowcolor[HTML]{C0C0C0} 
    \textbf{Board}                                             \\ \hline
    \rowcolor[HTML]{EFEFEF}                                    
    -\_cells : Cell[]                                          \\ \hline
    +ToString() : string                                       \\ \hline
    +Collision(int origin, int target) : bool                  \\ \hline
    +SameColour (int cell, Colour colour) : bool               \\ \hline
    +ValidMove(int origin, int target) : bool                  \\ \hline
    +MoveCellTo(int origin, int target) : void                 \\ \hline
    +GenerateBoard(string board) : void                        \\ \hline
    +IsEssetialExposed(Colour colour) : bool                   \\ \hline
    +HasPromotable(int target) : bool                          \\ \hline
    +GetAssaillants(Colour colour) : List<int>                 \\ \hline
    -GetEssentialPiece(Colour colour) : int                    \\ \hline
    -GetAttackingPieces(Colour colour, int target) : List<int> \\ \hline
    -HasAttackersAroundEssential(Colour colour) : bool          \\ \hline
    \end{tabular}
\end{table}

\subsubsection{Fields}

\begin{table}[H]
    \begin{tabular}{llllll}
    \hline
    \multicolumn{1}{|l|}{\cellcolor[HTML]{EFEFEF}\textbf{Field Name}} & \multicolumn{1}{l|}{\cellcolor[HTML]{EFEFEF}\textbf{Type}} & \multicolumn{1}{l|}{\cellcolor[HTML]{EFEFEF}\textbf{Visibility}} \\ \hline
    \multicolumn{1}{|l|}{\_cells}                                     & \multicolumn{1}{l|}{Cell[]}                                & \multicolumn{1}{l|}{Private}                                     \\ \hline
    \end{tabular}
\end{table}

\textbf{Description :} This field represent all 64 cells on a chess board. Each cell will contain a piece or be empty.

\subsubsection{Methods}

\begin{table}[H]
    \begin{tabular}{|l|l|l|l|}
    \hline
    \rowcolor[HTML]{EFEFEF} 
    \cellcolor[HTML]{EFEFEF}\textbf{Method Name} & \textbf{Parameters}    & \textbf{Returned Type} & \textbf{Visibility} \\ \hline
    Collision                                    & int origin, int target & bool                   & Public              \\ \hline
    \end{tabular}
\end{table}

\textbf{Parameters :} Parameter origin represents the cell
where the piece is before the move and target is the targeted
cell.
\\

\textbf{Description :} This method returns true if it detects a
possible collision in the path to the targeted cell.

\begin{table}[H]
    \begin{tabular}{|l|l|l|l|}
    \hline
    \rowcolor[HTML]{EFEFEF} 
    \cellcolor[HTML]{EFEFEF}\textbf{Method Name} & \textbf{Parameters}     & \textbf{Returned Type} & \textbf{Visibility} \\ \hline
    SameColour                                   & int cell, Colour colour & bool                   & Public              \\ \hline
    \end{tabular}
\end{table}

\textbf{Parameters :} Parameter cell represents the index of one of
the 64 cells that is selected and colour is the colour of the selected cell.
\\

\textbf{Description :} This method returns true the cell selected
contains a piece at the same colour of the current turn.

\begin{table}[H]
    \begin{tabular}{|l|l|l|l|}
    \hline
    \rowcolor[HTML]{EFEFEF} 
    \cellcolor[HTML]{EFEFEF}\textbf{Method Name} & \textbf{Parameters}     & \textbf{Returned Type} & \textbf{Visibility} \\ \hline
    ValidMove                                   & int origin, int target   & bool                   & Public              \\ \hline
    \end{tabular}
\end{table}

\textbf{Parameters :} Parameter origin represents the index of
the origin cell and the target parameter is the cell targeted.
\\

\textbf{Description :} This method returns true if the move is valid.

\begin{table}[H]
    \begin{tabular}{|l|l|l|l|}
    \hline
    \rowcolor[HTML]{EFEFEF} 
    \cellcolor[HTML]{EFEFEF}\textbf{Method Name} & \textbf{Parameters}     & \textbf{Returned Type} & \textbf{Visibility} \\ \hline
    MoveCellTo                                   & int origin, int target  & bool                   & Public              \\ \hline
    \end{tabular}
\end{table}

\textbf{Parameters :} Parameter origin represents the index of
the origin cell and the target parameter is the cell targeted.
\\

\textbf{Description :} This method swaps cells origin and target.

\begin{table}[H]
    \begin{tabular}{|l|l|l|l|}
    \hline
    \rowcolor[HTML]{EFEFEF} 
    \cellcolor[HTML]{EFEFEF}\textbf{Method Name} & \textbf{Parameters}     & \textbf{Returned Type} & \textbf{Visibility} \\ \hline
    GenerateBoard                              & string board            & void                   & Public              \\ \hline
    \end{tabular}
\end{table}

\textbf{Parameters :} Board parameter is a 64 char string that represent
a board state(tells where the pieces are supposed to be).
\\

\textbf{Description :} This method takes a string as board and transforms
each of the 64 char as the content of each of the 64 cells.

\begin{table}[H]
    \begin{tabular}{|l|l|l|l|}
    \hline
    \rowcolor[HTML]{EFEFEF} 
    \cellcolor[HTML]{EFEFEF}\textbf{Method Name} & \textbf{Parameters}     & \textbf{Returned Type} & \textbf{Visibility} \\ \hline
    IsEssentialExposed                           & Colour colour           & bool                   & Public              \\ \hline
    \end{tabular}
\end{table}

\textbf{Parameters :} The parameter colour represent the colour
to test.
\\

\textbf{Description :} Checks if the essential piece of the colour white or black
is exposed.

\begin{table}[H]
    \begin{tabular}{|l|l|l|l|}
    \hline
    \rowcolor[HTML]{EFEFEF} 
    \cellcolor[HTML]{EFEFEF}\textbf{Method Name} & \textbf{Parameters}     & \textbf{Returned Type} & \textbf{Visibility} \\ \hline
    HasPromotable                                & int target              & bool                   & Public              \\ \hline
    \end{tabular}
\end{table}

\textbf{Parameters :} The parameter target represent the targeted cell.
\\

\textbf{Description :} Checks if the targeted cell has the promotable attribute.

\begin{table}[H]
    \begin{tabular}{|l|l|l|l|}
    \hline
    \rowcolor[HTML]{EFEFEF} 
    \cellcolor[HTML]{EFEFEF}\textbf{Method Name} & \textbf{Parameters}     & \textbf{Returned Type} & \textbf{Visibility} \\ \hline
    GetAsaillants                                & Colour colour           & List<int>                   & Public              \\ \hline
    \end{tabular}
\end{table}

\textbf{Parameters :} The parameter colour represents the piece colour specified.
\textbf{Description :} This method returns a list of all possible asaillants by the index of their cell.

\begin{table}[H]
    \begin{tabular}{|l|l|l|l|}
    \hline
    \rowcolor[HTML]{EFEFEF} 
    \cellcolor[HTML]{EFEFEF}\textbf{Method Name} & \textbf{Parameters}     & \textbf{Returned Type} & \textbf{Visibility} \\ \hline
    GetEssentialPiece                            & Colour colour           & int                   & Public              \\ \hline
    \end{tabular}
\end{table}

\textbf{Parameters :} The parameter colour represents the piece colour specified.
\textbf{Description :} This method returns the index of the cell that contains the essential piece.

\begin{table}[H]
    \begin{tabular}{|l|l|l|l|}
    \hline
    \rowcolor[HTML]{EFEFEF} 
    \cellcolor[HTML]{EFEFEF}\textbf{Method Name} & \textbf{Parameters}        & \textbf{Returned Type} & \textbf{Visibility} \\ \hline
    GetAttackingPieces                           & Colour colour, int target  & List<int>                   & Public              \\ \hline
    \end{tabular}
\end{table}

\textbf{Parameters :} The parameter colour represents the piece colour specified and target is the targeted piece.
\textbf{Description :} This method returns a list of all attacking pieces.

\begin{table}[H]
    \begin{tabular}{|l|l|l|l|}
    \hline
    \rowcolor[HTML]{EFEFEF} 
    \cellcolor[HTML]{EFEFEF}\textbf{Method Name} & \textbf{Parameters}     & \textbf{Returned Type} & \textbf{Visibility} \\ \hline
    HasAttackersAroundEssential                  & Colour colour           & bool                   & Public              \\ \hline
    \end{tabular}
\end{table}

\textbf{Parameters :} The parameter colour represents the piece colour specified.
\textbf{Description :} This method returns true if there's attackers around the essential piece.

\newpage

%Cell Class%
%%%%%%%%%%%%%

\subsection{Class Cell}

This Class represents the model of a chess cell and its content.

\begin{table}[H]
    \begin{tabular}{|l|}
    \hline
    \rowcolor[HTML]{C0C0C0} 
    \textbf{Cell}                                    \\ \hline
    \rowcolor[HTML]{EFEFEF}
    -\_piece : Nullable<Piece>                        \\ \hline
    +IsEmpty() : bool                                 \\ \hline
    +HasCollision() : bool                            \\ \hline
    +HasPromotable() : bool                           \\ \hline
    +HasEssetial() : bool                             \\ \hline
    +ValidMove(int x1, int y1, int x2, int y2) : bool \\ \hline
    +Colour() : Colour                                \\ \hline
    \end{tabular}
\end{table}

\subsubsection{Fields}

\begin{table}[H]
    \begin{tabular}{llllll}
    \hline
    \multicolumn{1}{|l|}{\cellcolor[HTML]{EFEFEF}\textbf{Field Name}} & \multicolumn{1}{l|}{\cellcolor[HTML]{EFEFEF}\textbf{Type}} & \multicolumn{1}{l|}{\cellcolor[HTML]{EFEFEF}\textbf{Visibility}} \\ \hline
    \multicolumn{1}{|l|}{\_piece}                                     & \multicolumn{1}{l|}{Nullable<Piece>}                       & \multicolumn{1}{l|}{Private}                                     \\ \hline
    \end{tabular}
\end{table}

\textbf{Description :} This field represents de content of a cell if it is a piece
or null if it's empty.

\subsubsection{Methods}

\begin{table}[H]
    \begin{tabular}{|l|l|l|l|}
    \hline
    \rowcolor[HTML]{EFEFEF} 
    \cellcolor[HTML]{EFEFEF}\textbf{Method Name} & \textbf{Parameters}    & \textbf{Returned Type} & \textbf{Visibility} \\ \hline
    IsEmpty                                      & none                   & bool                   & Public              \\ \hline
    \end{tabular}
\end{table}

\textbf{Description :} This method returns true if the \_piece is null, so it's empty.

\begin{table}[H]
    \begin{tabular}{|l|l|l|l|}
    \hline
    \rowcolor[HTML]{EFEFEF} 
    \cellcolor[HTML]{EFEFEF}\textbf{Method Name} & \textbf{Parameters}     & \textbf{Returned Type} & \textbf{Visibility} \\ \hline
    HasCollision                                 & none                    & bool                   & Public              \\ \hline
    \end{tabular}
\end{table}

\textbf{Description :} This method returns true if the cell contains a piece that can collide.

\begin{table}[H]
    \begin{tabular}{|l|l|l|l|}
    \hline
    \rowcolor[HTML]{EFEFEF} 
    \cellcolor[HTML]{EFEFEF}\textbf{Method Name} & \textbf{Parameters}     & \textbf{Returned Type} & \textbf{Visibility} \\ \hline
    HasPromotable                                & none                    & bool                   & Public              \\ \hline
    \end{tabular}
\end{table}

\textbf{Description :} This method returns true if the cell contains a promotable piece.

\begin{table}[H]
    \begin{tabular}{|l|l|l|l|}
    \hline
    \rowcolor[HTML]{EFEFEF} 
    \cellcolor[HTML]{EFEFEF}\textbf{Method Name} & \textbf{Parameters}     & \textbf{Returned Type} & \textbf{Visibility} \\ \hline
    HasEssential                                 & none                    & bool                   & Public              \\ \hline
    \end{tabular}
\end{table}

\textbf{Description :} This method returns true if the cell contains a piece that is essential.

\begin{table}[H]
    \begin{tabular}{|l|l|l|l|}
    \hline
    \rowcolor[HTML]{EFEFEF} 
    \cellcolor[HTML]{EFEFEF}\textbf{Method Name} & \textbf{Parameters}            & \textbf{Returned Type} & \textbf{Visibility} \\ \hline
    ValidMove                                    & int x1, int y1, int x2, int y2 & bool                   & Public              \\ \hline
    \end{tabular}
\end{table}

\textbf{Parameters :} x1 ans y1 represent the coordinates of the position before a possible move
and x2 and y2 are the coordinates of the position after the move. These are all of Integer(16) type.
\\

\textbf{Description :} This method returns true if the move is valid.

\begin{table}[H]
    \begin{tabular}{|l|l|l|l|}
    \hline
    \rowcolor[HTML]{EFEFEF} 
    \cellcolor[HTML]{EFEFEF}\textbf{Method Name} & \textbf{Parameters}     & \textbf{Returned Type} & \textbf{Visibility} \\ \hline
    Colour                                       & none                    & Colour                   & Public              \\ \hline
    \end{tabular}
\end{table}

\textbf{Description :} This method returns the colour if the field \_piece is not null.

\newpage

%Player Class%
%%%%%%%%%%%%%%

\subsection{Class Player}

This Class represents a player in a chess game.

\begin{table}[H]
    \begin{tabular}{|l|}
    \hline
    \rowcolor[HTML]{C0C0C0} 
    \textbf{Player}          \\ \hline
    \rowcolor[HTML]{EFEFEF}
    -\_name : string        \\ \hline
    -\_points : int         \\ \hline
    \end{tabular}
\end{table}

\subsubsection{Fields}

\begin{table}[H]
    \begin{tabular}{llllll}
    \hline
    \multicolumn{1}{|l|}{\cellcolor[HTML]{EFEFEF}\textbf{Field Name}} & \multicolumn{1}{l|}{\cellcolor[HTML]{EFEFEF}\textbf{Type}} & \multicolumn{1}{l|}{\cellcolor[HTML]{EFEFEF}\textbf{Visibility}} \\ \hline
    \multicolumn{1}{|l|}{\_name}                                      & \multicolumn{1}{l|}{string}                                & \multicolumn{1}{l|}{Private}                                     \\ \hline
    \end{tabular}
\end{table}

\textbf{Description :} This field is the name of the player.

\begin{table}[H]
    \begin{tabular}{llllll}
    \hline
    \multicolumn{1}{|l|}{\cellcolor[HTML]{EFEFEF}\textbf{Field Name}} & \multicolumn{1}{l|}{\cellcolor[HTML]{EFEFEF}\textbf{Type}} & \multicolumn{1}{l|}{\cellcolor[HTML]{EFEFEF}\textbf{Visibility}} \\ \hline
    \multicolumn{1}{|l|}{\_points}                                      & \multicolumn{1}{l|}{int}                                 & \multicolumn{1}{l|}{Private}                                   \\ \hline
    \end{tabular}
\end{table}

\textbf{Description :} This field is the points that the player collects during a game.

\newpage

%NEW SECTION CONTROLLERS%
%%%%%%%%%%%%%%%%%%%%%%%%%

\section{Controllers}

%Chess Class%
%%%%%%%%%%%%%

\subsection{Class Chess}

\begin{table}[H]
    \begin{tabular}{|l|}
    \hline
    \rowcolor[HTML]{C0C0C0} 
    \textbf{Chess}                                             \\ \hline
    \rowcolor[HTML]{EFEFEF} 
    -\_listGames : List\textless{}GameController\textgreater{} \\ \hline
    +main() : void                                             \\ \hline
    +NewGame() : void                                          \\ \hline
    +StartGame(Player{[}2{]} players) : void                   \\ \hline
    +ManagePlayers() : void                                    \\ \hline
    +Exit() : void                                             \\ \hline
    \end{tabular}
\end{table}

\subsubsection{Fields}

\begin{table}[H]
    \begin{tabular}{llllll}
    \hline
    \multicolumn{1}{|l|}{\cellcolor[HTML]{EFEFEF}\textbf{Field Name}} & \multicolumn{1}{l|}{\cellcolor[HTML]{EFEFEF}\textbf{Type}} & \multicolumn{1}{l|}{\cellcolor[HTML]{EFEFEF}\textbf{Visibility}} \\ \hline
    \multicolumn{1}{|l|}{\_listGames}                                 & \multicolumn{1}{l|}{List<GameController>}                                & \multicolumn{1}{l|}{Private}                                     \\ \hline
    \end{tabular}
\end{table}

\textbf{Description :} This field is a list of the active games.

\subsubsection{Methods}

\begin{table}[H]
    \begin{tabular}{|l|l|l|l|}
    \hline
    \rowcolor[HTML]{EFEFEF} 
    \cellcolor[HTML]{EFEFEF}\textbf{Method Name} & \textbf{Parameters}    & \textbf{Returned Type} & \textbf{Visibility} \\ \hline
    main                                         & none                   & void                   & Public              \\ \hline
    \end{tabular}
\end{table}

\textbf{Description :} This is the entry point of the program.

\begin{table}[H]
    \begin{tabular}{|l|l|l|l|}
    \hline
    \rowcolor[HTML]{EFEFEF} 
    \cellcolor[HTML]{EFEFEF}\textbf{Method Name} & \textbf{Parameters}    & \textbf{Returned Type} & \textbf{Visibility} \\ \hline
    NewGame                                      & none                   & void                   & Public              \\ \hline
    \end{tabular}
\end{table}

\textbf{Description :} This method creates a new game with its own GameController, Match, Board etc\dots

\begin{table}[H]
    \begin{tabular}{|l|l|l|l|}
    \hline
    \rowcolor[HTML]{EFEFEF} 
    \cellcolor[HTML]{EFEFEF}\textbf{Method Name} & \textbf{Parameters}    & \textbf{Returned Type} & \textbf{Visibility} \\ \hline
    StartGame                                    & Player[2] players      & void                   & Public              \\ \hline
    \end{tabular}
\end{table}

\textbf{Description :} This method starts the game with the players given.

\begin{table}[H]
    \begin{tabular}{|l|l|l|l|}
    \hline
    \rowcolor[HTML]{EFEFEF} 
    \cellcolor[HTML]{EFEFEF}\textbf{Method Name} & \textbf{Parameters}    & \textbf{Returned Type} & \textbf{Visibility} \\ \hline
    ManagePlayers                                & none                   & void                   & Public              \\ \hline
    \end{tabular}
\end{table}

\textbf{Description :} This method makes the link between FormMenu and FormLeaderboard.

\begin{table}[H]
    \begin{tabular}{|l|l|l|l|}
    \hline
    \rowcolor[HTML]{EFEFEF} 
    \cellcolor[HTML]{EFEFEF}\textbf{Method Name} & \textbf{Parameters}    & \textbf{Returned Type} & \textbf{Visibility} \\ \hline
    Exit                                         & none                   & void                   & Public              \\ \hline
    \end{tabular}
\end{table}

\textbf{Description :} This is the exit point method.

\newpage

%GameController Class%
%%%%%%%%%%%%%%%%%%%%%%

\subsection{Class GameController}

\begin{table}[H]
    \begin{tabular}{|l|}
    \hline
    \rowcolor[HTML]{C0C0C0} 
    \textbf{GameController}              \\ \hline
    \rowcolor[HTML]{EFEFEF} 
    -\_main : Chess                      \\ \hline
    \rowcolor[HTML]{EFEFEF} 
    -\_selected : int                    \\ \hline
    \rowcolor[HTML]{EFEFEF} 
    -\_match : Match                     \\ \hline
    \rowcolor[HTML]{EFEFEF} 
    -\_playerA : Player                  \\ \hline
    \rowcolor[HTML]{EFEFEF} 
    -\_playerB : Player                  \\ \hline
    \rowcolor[HTML]{EFEFEF} 
    -\_tieCounter : int                  \\ \hline
    \rowcolor[HTML]{EFEFEF} 
    -\_view : FormMatch                  \\ \hline
    -Check() : bool                      \\ \hline
    -SelfCheck() : bool                  \\ \hline
    -Checkmate() : bool                  \\ \hline
    -Castle() : bool                     \\ \hline
    -Promotion() : bool                  \\ \hline
    -FiftyTurns() : bool                 \\ \hline
    -SameBoard() : bool                  \\ \hline
    +Turn(int origin, int target) : void \\ \hline
    +Selection(int cell) : void          \\ \hline
    +Resign() : void                     \\ \hline
    \end{tabular}
\end{table}



\newpage

%PlayerController Class%
%%%%%%%%%%%%%%%%%%%%%%%%

\subsection{Class PlayerController}

\begin{table}[H]
    \begin{tabular}{|l|}
    \hline
    \rowcolor[HTML]{C0C0C0} 
    \textbf{PlayerController}                       \\ \hline
    \rowcolor[HTML]{EFEFEF} 
    -\_main : Chess                               \\ \hline
    \rowcolor[HTML]{EFEFEF} 
    -\_list : List\textless{}Player\textgreater{} \\ \hline
    \rowcolor[HTML]{FFFFFF} 
    +Add() : void                                 \\ \hline
    \rowcolor[HTML]{FFFFFF} 
    +Remove() : void                              \\ \hline
    \end{tabular}
    \end{table}

\newpage

%NEW SECTION VIEWS%
%%%%%%%%%%%%%%%%%%%

\section{Views}

%FormSelection Class%
%%%%%%%%%%%%%%%%%%%%%

\subsection{Class FormSelection}

\begin{table}[H]
    \begin{tabular}{|l|}
    \hline
    \rowcolor[HTML]{C0C0C0} 
    \textbf{FormSelection}    \\ \hline
    \rowcolor[HTML]{EFEFEF} 
    -\_controller : Chess      \\ \hline
    \rowcolor[HTML]{FFFFFF} 
    +OpenLeaderboard() : void \\ \hline
    \rowcolor[HTML]{FFFFFF} 
    +Start() : Player{[}2{]}  \\ \hline
    \rowcolor[HTML]{FFFFFF} 
    +Cancel() : void          \\ \hline
    \end{tabular}
\end{table}



\newpage

%FormMenu Class%
%%%%%%%%%%%%%%%%

\subsection{Class FormMenu}

\begin{table}[H]
    \begin{tabular}{|l|}
    \hline
    \rowcolor[HTML]{C0C0C0} 
    \textbf{FormMenu}                                   \\ \hline
    \rowcolor[HTML]{EFEFEF} 
    -\_main : Chess                                          \\ \hline
    \rowcolor[HTML]{FFFFFF} 
    +Start(object sender, System.EventArgs e) : void         \\ \hline
    \rowcolor[HTML]{FFFFFF} 
    +Exit(object sender, System.EventArgs e) : void          \\ \hline
    \rowcolor[HTML]{FFFFFF} 
    +ManagePlayers(object sender, System.EventArgs e) : void \\ \hline
    \end{tabular}
\end{table}

\newpage

%FormPromotion Class%
%%%%%%%%%%%%%%%%%%%%%

\subsection{Class FormPromotion}

\begin{table}[H]
    \begin{tabular}{|l|}
    \hline
    \rowcolor[HTML]{C0C0C0} 
    \textbf{FormPromotion}                            \\ \hline
    \rowcolor[HTML]{EFEFEF} 
    -\_controller : GameController                    \\ \hline
    \rowcolor[HTML]{FFFFFF} 
    +Submit(object sender, System.EventArgs e) : void \\ \hline
    \end{tabular}
\end{table}



\newpage

%FormMatch Class%
%%%%%%%%%%%%%%%%%

\subsection{Class FormMatch}

\begin{table}[H]
    \begin{tabular}{|l|}
    \hline
    \rowcolor[HTML]{C0C0C0} 
    \textbf{FormMatch}                                   \\ \hline
    \rowcolor[HTML]{EFEFEF} 
    -\_controller : GameController                       \\ \hline
    \rowcolor[HTML]{FFFFFF} 
    +GridClick(object sender, System.EventArgs e) : void \\ \hline
    +DrawBoard(string board) : void                      \\ \hline
    +DrawSelection(int cell) : void                      \\ \hline
    +ShowMessage(string message) : void                  \\ \hline
    +VictoryMessage() : void                             \\ \hline
    \end{tabular}
\end{table}

\newpage

%FormLeaderboard Class%
%%%%%%%%%%%%%%%%%%%%%%%

\subsection{Class FormLeaderboard}

\begin{table}[H]
    \begin{tabular}{|l|}
    \hline
    \rowcolor[HTML]{C0C0C0} 
    \textbf{FormLeaderboard}                                    \\ \hline
    \rowcolor[HTML]{EFEFEF} 
    -\_controller : PlayerController                      \\ \hline
    \rowcolor[HTML]{FFFFFF} 
    +ShowList(List\textless{}Player\textgreater{}) : void \\ \hline
    +Add() : void                                         \\ \hline
    +Remove() : void                                      \\ \hline
    +Back() : void                                        \\ \hline
    \end{tabular}
\end{table}

\newpage

\end{document}