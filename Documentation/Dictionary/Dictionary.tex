\documentclass[12pt]{article}
\usepackage[T1]{fontenc}
\usepackage{booktabs}
\usepackage{float}
\usepackage[table,xcdraw]{xcolor}
\usepackage{graphicx}

\graphicspath{ {./images/} }

\title{Class Dictionary}
\author{Jean-Philippe Miguel-Gagnon, Jérémy Gaouette, Raphaël Rail}

\date{Thursday, 31th of march 2022}

\begin{document}

\begin{titlepage}
\maketitle
\includegraphics[width=\textwidth]{CHESS}
\begin{center}Presented to : Charles Jacob\end{center}
\end{titlepage}

\tableofcontents

\newpage

\section{Introduction}

The goal of this document is to inform the programmer about
classes used to create a C\#\ OOP Chess game.
\\

We'll go through this with the MVC model approch to make it
clearer for the programmer where to implement his code. 

\newpage

\section{Models}

\subsection{Classe `Piece'}

This is an abstract class for a base piece for the game.
\begin{table}[H]
    \begin{tabular}{|l|}
    \hline
    \cellcolor[HTML]{A9A9A9}\textbf{Piece}            \\ \hline
    \cellcolor[HTML]{E8E8E8}-\_colour : Colour        \\ \hline
    +CanCollide() : bool                                \\ \hline
    +ValidMove(int x1, int y1, int x2, int y2) : bool \\ \hline
    +ToString()                                       \\ \hline
    \end{tabular}
\end{table}

\subsection{Fields}

\begin{table}[H]
    \begin{tabular}{llllll}
    \hline
    \multicolumn{1}{|l|}{\cellcolor[HTML]{E8E8E8}\textbf{Field Name}} & \multicolumn{1}{l|}{\cellcolor[HTML]{E8E8E8}\textbf{Type}} & \multicolumn{1}{l|}{\cellcolor[HTML]{E8E8E8}\textbf{Visibility}} \\ \hline
    \multicolumn{1}{|l|}{\_colour}                                    & \multicolumn{1}{l|}{Colour}                                & \multicolumn{1}{l|}{Private}                                     \\ \hline
    \end{tabular}
\end{table}

\textbf{Description :} This field represent the colour a piece, wich can only be
black or white as it is for a regular chess game.


\subsection{Methods}

\begin{table}[H]
    \begin{tabular}{|l|l|l|l|}
    \hline
    \rowcolor[HTML]{EFEFEF} 
    \cellcolor[HTML]{EFEFEF}\textbf{Method Name} & \textbf{Parameters}  & \textbf{Returned Type} & \textbf{Visibility} \\ \hline
    ValidMove                          & x1, y1, x2, y2 : int & bool                   & Public              \\ \hline
    \end{tabular}
\end{table}

\textbf{Parameters :} x1 ans y1 represent the coordinates of the position before a possible move
and x2 and y2 are the coordinates of the position after the move. These are all of Integer(16) type.
\\
\textbf{Description :} The method returns true if the piece can move to the 2nd position.

\begin{table}[H]
    \begin{tabular}{|l|l|l|l|}
    \hline
    \rowcolor[HTML]{EFEFEF} 
    \cellcolor[HTML]{EFEFEF}\textbf{Method Name} & \textbf{Parameters}  & \textbf{Returned Type} & \textbf{Visibility} \\ \hline
    CanCollide                                   & None                 & bool                   & Public              \\ \hline
    \end{tabular}
\end{table}

\textbf{Description :} Returns true if the piece can't go over other pieces, otherwise
it returns false.

\begin{table}[H]
    \begin{tabular}{|l|l|l|l|}
    \hline
    \rowcolor[HTML]{EFEFEF} 
    \cellcolor[HTML]{EFEFEF}\textbf{Method Name} & \textbf{Parameters}  & \textbf{Returned Type} & \textbf{Visibility} \\ \hline
    ToString                                   & None                 & String                   & Public              \\ \hline
    \end{tabular}
\end{table}

\textbf{Description :} Returns true if the piece can't go over other pieces, otherwise
it returns false.

\subsection{Properties}

\begin{table}[H]
    \begin{tabular}{|l|l|l|l|}
    \hline
    \rowcolor[HTML]{EFEFEF} 
    \cellcolor[HTML]{EFEFEF}\textbf{Property Name} & \textbf{Parameters}  & \textbf{Returned Type} & \textbf{Visibility} \\ \hline
    Colour                                         & None                 & Colour                 & Public              \\ \hline
    \end{tabular}
\end{table}

\textbf{Description :} Returns de colour of a piece.


   
\end{document}